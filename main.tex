\documentclass[12pt,letterpaper]{article}
\usepackage{amitstyle}

\DeclareMathOperator{\cost}{cost}
\DeclareMathOperator{\dist}{dist}        % the other metric used in the paper
\newcommand{\dGT}[2]{\dist_{\mathrm{GT}}(#1,#2)} % Galois-transport metric

\newcommand{\vect}{\mathsf{vec}}         % category of finite-dimensional vector spaces
\newcommand{\Pos}{\mathbf{Pos}}          % category of posets

\DeclareMathOperator{\Match}{Match}      % set of matchings (as defined in text)
\DeclareMathOperator{\Bij}{Bij}          % set of bijections
\DeclareMathOperator{\Cone}{Cone}        % cone construction
\DeclareMathOperator{\Summands}{Summands}% set of direct summands

\newcommand{\fU}{\mathfrak{U}}           % fixed universe for the 2-category
\newcommand{\GT}{\mathrm{GT}}            % (optional) label for the name in prose

\newcommand{\todoilgreen}[1]{\textcolor{green}{TODO: #1}}
\newcommand{\frkU}{\mathfrak{U}}

\newcommand\Nname[1]{|[alias=#1]|}

\newcommand\al{\alpha}

\newcommand\be{\beta}

\newcommand\ep{\varepsilon}

\newcommand\Ga{\Gamma}

\newcommand\prj{\operatorname{prj}}

\newcommand\calC{{\mathcal C}}

\newcommand\calD{{\mathcal D}}

\newcommand\calP{{\mathcal P}}

\newcommand\bbN{\mathbb{N}}

\newcommand\bbR{\mathbb{R}}

\newcommand\iso{\cong}

\newcommand\ds{\oplus}

\newcommand\Ds{\bigoplus}

\newcommand\id{1\kern-.25em{\text{{\rm l}}}} 

\newcommand\isoto{\ \raise.8ex\hbox{$^{\sim}$}\kern-.7em\hbox{$\to$}\ }

\newcommand\down{_{\cdot}}

\newcommand\blank{\operatorname{-}}



\newcommand\bfP{\mathbf{P}}

\newcommand\bfQ{\mathbf{Q}}

\newcommand\bfX{\mathbf{X}}

\newcommand\bfY{\mathbf{Y}}

\newcommand\bfZ{\mathbf{Z}}

\newcommand\sfU{\mathsf{U}}

\newcommand\Fun{\operatorname{Fun}}

\newcommand\Cat{\mathbf{Cat}}

% Representable at x in P
\newcommand{\Yp}[1]{\bfP(#1,-)}
\DeclareMathOperator{\Hom}{Hom}






\begin{document}

\title{Title TBD}
\author{Hideto Asashiba\thanks{Supported by JSPS Grant-in-Aid for Scientific Research (C) 18K03207; 
JSPS Grant-in-Aid for Transformative Research Areas (A) (22A201); 
Osaka Central Advanced Mathematical Institute (MEXT Promotion of Distinctive Joint Research Center Program JPMXP0723833165).}
\and Amit K. Patel}


\maketitle

\begin{abstract}

\end{abstract}

\noindent\textbf{2020 Mathematics Subject Classification.} 16G20, 16G70, 55N31, 62R40.

\medskip

\noindent\textbf{Affiliations.}
Department of Mathematics, Faculty of Science, Shizuoka University,
836 Ohya, Suruga-ku, Shizuoka, 422-8529, Japan; Institute for Advanced Study, KUIAS, Kyoto University,
Yoshida Ushinomiya-cho, Sakyo-ku, Kyoto 606-8501, Japan; and
Osaka Central Advanced Mathematical Institute, 3-3-138 Sugimoto, Sumiyoshi-ku, Osaka, 558-8585, Japan. \par
Department of Mathematics, Colorado State University, Fort Collins, CO 80523, USA.

\medskip

\noindent\textbf{Emails.}
\href{mailto:asashiba.hideto@shizuoka.ac.jp}{asashiba.hideto@shizuoka.ac.jp}
\quad|\quad
\href{mailto:amit@akpatel.org}{amit@akpatel.org}


\tableofcontents

\section{Introduction}

%%%%%%%%%%%%%%

\subsection{Purposes}

%%%%%%%%%%%%%%%%%%%

\subsection{Our contributions}


\subsection{Organization}


\section{Preliminaries}
\label{sec:prelim}

We start with some categorical facts that will be used throughout.

\begin{lemma}
\label{lem:pres-proj}
Let $L:\calC\to\calD$ be a functor between abelian categories.
If $L$ has an exact right adjoint $R:\calD\to\calC$,
then $L$ sends projectives to projectives.
\end{lemma}

\begin{proof}
Let $A$ be projective in $\calC$. Then the functor $\calC(A,-)$ is exact, hence so is
\[
\calC(A,-)\circ R=\calC(A,R(-))\cong \calD(L(A),-).
\]
Thus $L(A)$ is projective.
\end{proof}

Let $\frkU$ be a universe. An element of $\frkU$ is called a $\frkU$-\emph{small} set. A category $\calC$ is $\frkU$-\emph{small} if its object set is $\frkU$-small and, for any objects $X,Y$, each hom-set $\calC(X,Y)$ is $\frkU$-small. We denote by $\Cat^\frkU$ the $2$-category whose objects are the $\frkU$-small categories, whose $1$-morphisms are functors, and whose $2$-morphisms are natural transformations.

\begin{proposition}
\label{prp:Yoneda-2-functor}
If $\calC$ is an object of $\Cat^\frkU$, then
\[
\Cat^\frkU(\mbox{-},\calC):(\Cat^\frkU)^{\mathrm{op}}\longrightarrow\Cat^\frkU
\]
is a $2$-functor.
\end{proposition}

\begin{proof}
See Proposition~4.5.4 in \cite{JY}.
\end{proof}

\begin{remark}
\label{rmk:contravariant-adj}
In particular, the $2$-functor $\Cat^\frkU(\mbox{-},\calC)$ sends an adjoint system $(f,g,\eta,\varepsilon)$ to the adjoint system $(g^{*},f^{*},\eta^{*},\varepsilon^{*})$ in a contravariant manner. Concretely, if $f\dashv g$ between categories $\calA$ and $\calB$, then precomposition yields $g^{*}\dashv f^{*}$ between the functor categories $\Cat^\frkU(\calB,\calC)$ and $\Cat^\frkU(\calA,\calC)$. We use this repeatedly with $\calC=\vect$ below.
\end{remark}

%%%%%%%%%%%%%%%%%%%%%%%%%%%%%%%%%%%%%%%%%%%%%%%%%%%%%%%%%%%%%%%%%%%
\subsection{Poset Modules}
\label{ssec:poset-mod}
%%%%%%%%%%%%%%%%%%%%%%%%%%%%%%%%%%%%%%%%%%%%%%%%%%%%%%%%%%%%%%%%%%%

Throughout, $\Bbbk$ is a field, and $\bfP=(\bfP,\le)$ is a poset. We regard $\bfP$ as a small (thin) category: there is a unique morphism $x\to y$ exactly when $x\le y$. This lets us identify $\bfP$-modules with functors $\bfP\to\vect$ and, equivalently, with left modules over the $\Bbbk$-linearization $\Bbbk\bfP$ (same objects, order relations as basis arrows, composition by transitivity). We assume $\bfP$ is finite except in Sect.\ \ref{sec:intlv-dist}.

Let $\vect$ denote the category of finite-dimensional $\Bbbk$-vector spaces, and write $\vect^\bfP=\Fun(\bfP,\vect)$ for the category of $\bfP$-modules. If $\bfP$ is finite, then $\vect^\bfP$ is a $\Bbbk$-linear, abelian, Krull–Schmidt category, and for each $M,N\in\vect^\bfP$ the $\Bbbk$-space $\vect^\bfP(M,N)$ is finite dimensional.

We write $\prj\bfP$ for the full subcategory of projective objects in $\vect^\bfP$.
For $x\in\bfP$, the \emph{representable} at $x$ is the functor
\[
\bfP(x,-)\in\vect^\bfP.
\]
It sends $y\in\bfP$ to $\bfP(x,-)(y)=\Bbbk$ if $x\le y$ and to $0$ otherwise; for a relation $y\le z$ it acts by the identity on $\Bbbk$ whenever $x\le y$ (and by $0$ otherwise). By the Yoneda lemma there is a natural isomorphism
\[
\Hom_{\vect^\bfP}(\bfP(x,-),M)\cong M(x)\qquad(M\in\vect^\bfP),
\]
showing that each $\bfP(x,-)$ is projective.

\begin{lemma}
\label{lem:ind-proj}
The family $\{\bfP(x,-)\mid x\in\bfP\}$ is a complete set of representatives of the isomorphism classes of indecomposable projective $\bfP$-modules. Hence each $M\in\prj \bfP$ decomposes uniquely as
\[
M \cong \bigoplus_{x\in\bfP} \bfP(x,-)^{\,a_x}
\]
for a uniquely determined family $(a_x)_{x\in\bfP}\in\bbN^\bfP$.
\end{lemma}

%%%%%%%%%%%%%%%%%%%%%%%%%%%%%%%%%%%%%%%%%%%%%%%%%%%%%%%%%%%%%%%%%%%
\subsection{Monotone Functions and Kan Extensions}
\label{ssec:mont-funct}
%%%%%%%%%%%%%%%%%%%%%%%%%%%%%%%%%%%%%%%%%%%%%%%%%%%%%%%%%%%%%%%%%%%

A map $f:\bfQ\to\bfP$ of posets is \emph{monotone} if $x\le y$ in $\bfQ$ implies $f(x)\le f(y)$ in $\bfP$. Viewing posets as thin categories, a functor $\bfQ\to\bfP$ is precisely a monotone map. Thus a monotone map $f:\bfQ\to\bfP$ induces the restriction (precomposition) functor
\[
f^{*}:\vect^\bfP\to\vect^\bfQ,\qquad f^{*}N = N\circ f .
\]

\begin{proposition}
\label{prop:Kan-posets}
For any monotone $f:\bfQ\to\bfP$, the functor $f^{*}$ is exact. Moreover, the left and right Kan extensions along $f$ exist and we denote them by
\[
f_{!}:=\Lan_{f}\qquad\text{and}\qquad f_{*}:=\Ran_{f}.
\]
By the defining property of conextensions there are natural adjunctions
\[
f_{!}\dashv f^{*}\dashv f_{*}.
\]
In particular, $f_{!}$ is right exact, $f_{*}$ is left exact, and $f^{*}$ is exact.
\end{proposition}

\[
\begin{tikzcd}[row sep=3em, column sep=6em]
\Nname{E}\vect^\bfQ & \Nname{C}\vect^\bfP
\arrow[from={E}, to={C}, "f_{!}", ""'{name=Fu}, bend left]
\arrow[from={C}, to={E},"f^{*}"{description, pos=.5, name=Fc}]
\arrow[from={E}, to={C}, "f_{*}"', ""'{name=Fd}, bend right]
\arrow[from={Fu}, to={Fc}, "\rotatebox{-90}{$\dashv$}" description, phantom]
\arrow[from={Fc}, to={Fd}, "\rotatebox{-90}{$\dashv$}" description, phantom]
\end{tikzcd}
\]

%%%%%%%%%%%%%%%%%%%%%%%%%%%%%%%%%%%%%%%%%%%%%%%%%%%%%%%%%%%%%%%%%%%
\subsection{Galois Connections}
%%%%%%%%%%%%%%%%%%%%%%%%%%%%%%%%%%%%%%%%%%%%%%%%%%%%%%%%%%%%%%%%%%%

We recall that Galois connections are adjunctions between posets and underlie the identifications among $f_{!}, f^{*}, f_{*}$ used later.

\begin{definition}
A \emph{Galois connection} between posets $\bfQ,\bfP$ consists of monotone maps $f:\bfQ\to\bfP$ and $g:\bfP\to\bfQ$ such that
\[
f(u)\le x \iff u\le g(x)\qquad(u\in \bfQ,\ x\in \bfP).
\]
Equivalently, viewing $\bfQ$ and $\bfP$ as thin categories, this means $f\dashv g$ as functors. We write $f:\bfQ\rightleftarrows \bfP:g$, with $f$ left adjoint and $g$ right adjoint.
\end{definition}

\begin{corollary}
If $f:\bfQ\rightleftarrows \bfP:g$ is a Galois connection, then $g^{*}\dashv f^{*}$.
\[
\begin{tikzcd}[row sep=3em]
\Nname{E}\vect^\bfP & \Nname{C} \vect^\bfQ 
\arrow[from={C}, to={E}, "g^{*}", ""'{name=Gs}, bend left]
\arrow[from={E}, to={C}, "f^{*}"{name=Fs}, bend left]
\arrow[from={Gs}, to={Fs}, "\rotatebox{-90}{$\dashv$}" description, phantom]
\end{tikzcd}.
\]
\end{corollary}

\begin{proof}
Immediate from the contravariant $2$-functoriality of $\Cat^\frkU(\mbox{-},\vect)$ (e.g.\ Proposition~4.5.4 in \cite{JY}).
\end{proof}

\begin{corollary}
\label{cor:3_adj}
In the same setting, there are natural isomorphisms $g^{*}\cong f_{!}$ and $f^{*}\cong g_{*}$. In particular, the adjoint pairs
\[
g^{*}\dashv f^{*},\qquad f_{!}\dashv f^{*},\qquad g^{*}\dashv g_{*}
\]
are compatible via these isomorphisms.
\[
\begin{tikzcd}[row sep=50pt, column sep = 70pt]
\Nname{E}\vect^\bfP & \Nname{C} \vect^\bfQ 
\arrow[from={C}, to={E}, "g^{*} \cong f_{!}"{description, pos=.5, name=gs}, bend left=15pt]
\arrow[from={E}, to={C}, "f^{*}\cong g_{*}"{description, pos=.5, name=fs}, bend left=15pt]
\arrow[from={E}, to={C}, "g_{!}"', ""{name=gL}, bend right=80pt]
\arrow[from={C}, to={E}, "f_{*}"', ""{name=fR}, bend right=80pt]
\arrow[from={gs}, to={fs}, "\rotatebox{-90}{$\dashv$}" description, phantom]
\arrow[from={gL}, to={gs}, "\rotatebox{-90}{$\dashv$}" description, phantom]
\arrow[from={fs}, to={fR}, "\rotatebox{-90}{$\dashv$}" description, phantom]
\end{tikzcd}.
\]
\end{corollary}

\begin{proof}
By uniqueness of adjoints in a $2$-category: $g^{*}$ and $f_{!}$ are both left adjoint to $f^{*}$, hence canonically isomorphic; dually for $f^{*}$ and $g_{*}$.
\end{proof}

\begin{corollary}
\label{cor:g-ast-ex}
If $f:\bfQ\rightleftarrows \bfP:g$ is a Galois connection, then $g_{*}:\vect^\bfP\to\vect^\bfQ$ is exact.
\end{corollary}

\begin{proof}
Using Corollary~\ref{cor:3_adj}, $g_{*}\cong f^{*}$, and $f^{*}$ is exact by Proposition~\ref{prop:Kan-posets}.
\end{proof}

Combining Corollary~\ref{cor:g-ast-ex} with Lemma~\ref{lem:pres-proj} yields:

\begin{proposition}\label{prop:pullback_preserves_projectives}
Let $f:\bfQ\rightleftarrows \bfP:g$ be a Galois connection. If $M\in\vect^\bfQ$ is projective, then $g^{*}(M)$ is projective in $\vect^\bfP$.\qed
\end{proposition}

\medskip

The next basic facts will be used tacitly; they are immediate from the adjunction $f\dashv g$ and we omit the proof.

\begin{lemma}\label{lem:galois-connection-basics}
Let $f:\bfQ\rightleftarrows \bfP:g$ be a Galois connection. Then:
\begin{enumerate}
\item The following are equivalent: $f$ is surjective; $g$ is injective; $f\circ g=\id_{\bfP}$.
\item For $x\in \bfP$,
\[
g(x)=\max\{\,u\in \bfQ\mid f(u)\le x\,\}.
\]
In particular, if $f\circ g=\id_{\bfP}$, then
\[
g(x)=\max\{\,u\in \bfQ\mid f(u)=x\,\}.
\]
\end{enumerate}
\end{lemma}



%%%%%%%%%%%%%%%%%%%%%%%%%%%%%%%%%%%%%%%%%%%%%%%%%%%%%%%%%%%%%%%%%
%%%%%%%%%%%%%%%%%%%%%%%%%%%%%%%%%%%%%%%%%%%%%%%%%%%%%%%%%%%%%%%%%
\section{Galois Transport Distance}
\label{sec:galois-transport}
%%%%%%%%%%%%%%%%%%%%%%%%%%%%%%%%%%%%%%%%%%%%%%%%%%%%%%%%%%%%%%%%%
%%%%%%%%%%%%%%%%%%%%%%%%%%%%%%%%%%%%%%%%%%%%%%%%%%%%%%%%%%%%%%%%%

Fix a finite poset $\bfP$ equipped with a metric $d_{\bfP}$. Motivated by optimal transport, we compare $\bfP$-modules by \emph{transporting} them through a common “apex” poset $\bfQ$ via \emph{Galois insertions}. 

\begin{definition}
\label{def:coupling-insertion}
Let $M,N\in\vect^{\bfP}$. A \emph{Galois coupling} of $(M,N, f \dashv g, h \dashv i, \Gamma)$ consists of a poset $\bfQ$, two Galois insertions
\[
f:\bfQ\rightleftarrows \bfP:g,
\qquad
h:\bfQ\rightleftarrows \bfP:i
\quad\text{with}\quad
f\circ g=\id_{\bfP}=h\circ i,
\]
and a module $\Gamma\in\vect^{\bfQ}$ such that $g^{*}\Gamma \cong M$ and $i^{*}\Gamma \cong N$. Equivalently (Corollary~\ref{cor:3_adj}), $M\cong f_{!}\Gamma$ and $N\cong h_{!}\Gamma$.
\end{definition}

\[
\begin{tikzcd}
& \bfQ \ar[dd,"\Gamma"] \ar[dr, bend left, "h"{name=H}] \ar[dl, bend right, "f"'{name=F}] & \\
\bfP \ar[rd,"M"'] \ar[ru, bend right, "g"'{name=G}] && \bfP \ar[ld,"N"] \ar[lu, bend left, "i"{name=I}] \\
& \vect &
% adjunction markers:
\arrow[phantom, from=F, to=G, "\rotatebox{-45}{$\dashv$}" description]
\arrow[phantom, from=H, to=I, "\rotatebox{-135}{$\dashv$}" description]
\end{tikzcd}
\]

\begin{definition}
\label{def:cost}
The \emph{cost} of a coupling $(\bfQ,f\dashv g,\ h\dashv i,\ \Gamma)$ is
\[
\cost(\Gamma):=\sup_{q\in\bfQ} d_{\bfP}\big(f(q),\,h(q)\big).
\]
\end{definition}

%%%%%%%%%%%%%%%%%%%%%%%%%%%%%%%%%%%%%%%%%%%%%%%%%%%%%%%%%%%%%%%%%
\subsection{Composition of Couplings}
\label{ssec:composition}
%%%%%%%%%%%%%%%%%%%%%%%%%%%%%%%%%%%%%%%%%%%%%%%%%%%%%%%%%%%%%%%%%

Now consider two Galois couplings $(M,N, f_1 \dashv g_1, h_1 \dashv i_1, \Gamma_1)$ and
$(N,O, f_2 \dashv g_2, h_2 \dashv i_2, \Gamma_2)$ below:

\[
\begin{tikzcd}
    && \bfR \ar[dr, bend left, "\pi_2"{name=piTwo}] \ar[dl, bend right, "\pi_1"'{name=piOne}]  && \\
    & \bfQ_1 \ar[dddr, "\Gamma_1", near end, bend right = 20] \ar[ur, bend right, "\iota_1"'{name=iotaOne}]  
        \ar[dl, bend right, "f_1"'{name=fOne}] \ar[dr, bend left, "h_1"{name=hOne}] & & 
      \bfQ_2 \ar[ddld, "\Gamma_2"', bend left = 20, near end] \ar[ul, bend left, "\iota_2"{name=iotaTwo}] 
        \ar[dl, bend right, "f_2"'{name=fTwo}] \ar[dr, bend left, "h_2"{name=hTwo}] & \\
    \bfP \ar[rrdd, "M"] \ar[ur, bend right, "g_1"'{name=gOne}] && 
    \bfP \ar[dd, "N"]  \ar[ul, bend left, "i_1"{name=iOne}] 
      \ar[ur, bend right, "g_2"'{name=gTwo}] && 
    \bfP \ar[ldld, "O"] \ar[ul, bend left, "i_2"{name=iTwo}] \\
    &  & &  & \\
    && \vect &&
    % adjunction markers:
    \arrow[phantom, from=piOne, to=iotaOne, "\rotatebox{-45}{$\dashv$}" description]
    \arrow[phantom, from=piTwo, to=iotaTwo, "\rotatebox{-135}{$\dashv$}" description]
    \arrow[phantom, from=fOne,  to=gOne,   "\rotatebox{-45}{$\dashv$}" description]
    \arrow[phantom, from=hOne,  to=iOne,   "\rotatebox{-135}{$\dashv$}" description]
    \arrow[phantom, from=fTwo,  to=gTwo,   "\rotatebox{-34}{$\dashv$}" description]
    \arrow[phantom, from=hTwo,  to=iTwo,   "\rotatebox{-135}{$\dashv$}" description]
\end{tikzcd}
\]
\vspace{-0.5em}

Let $\bfR$ be the pullback of $(h_1,f_2)$ in the category of posets (i.e.\ thin categories), with projections $\pi_1:\bfR\to\bfQ_1$ and $\pi_2:\bfR\to\bfQ_2$. Set
\[
\delta:=h_1\circ\pi_1=f_2\circ\pi_2:\bfR\to\bfP.
\]
Since $h_1$ and $f_2$ are the left adjoints in Galois insertions, $f_2\circ g_2=\id_{\bfP}$ and $h_1\circ i_1=\id_{\bfP}$. By the universal property of the pullback there are unique monotone maps
\[
\iota_1:\bfQ_1\to\bfR\quad\text{and}\quad \iota_2:\bfQ_2\to\bfR
\]
with $\pi_1\circ\iota_1=\id_{\bfQ_1}$ and $\pi_2\circ\iota_2=\id_{\bfQ_2}$. In particular, $\pi_j\dashv\iota_j$ is a Galois insertion (Lemma~\ref{lem:galois-connection-basics}(1)–(2)).

\begin{proposition}
\label{prop:compose-couplings}
With notation as above, there exists $\Psi\in\vect^{\bfR}$ and natural isomorphisms
\[
(\iota_1\circ g_1)^{*}\Psi \cong M,
\qquad
(\iota_2\circ i_2)^{*}\Psi \cong O,
\]
so that
\[
\big(M,O,\ f_1\circ\pi_1 \dashv \iota_1\circ g_1,\ \ h_2\circ\pi_2 \dashv \iota_2\circ i_2,\ \ \Psi\big)
\]
is a Galois coupling (the \emph{composite} of the two displayed couplings).
Moreover, this composite is unique up to unique isomorphism of couplings.
\end{proposition}

\begin{proof}
Consider the canonical isomorphisms in $\vect^{\bfP}$ coming from the two couplings,
\[
i_1^{*}\Gamma_1 \ \cong\ N \ \cong\ g_2^{*}\Gamma_2,
\]
and let $\varphi:i_1^{*}\Gamma_1\xrightarrow{\cong} g_2^{*}\Gamma_2$ be their composite. Whisker by $\delta^{*}$ to obtain an isomorphism
\[
\delta^{*}i_1^{*}\Gamma_1 \xrightarrow{\cong} \delta^{*}g_2^{*}\Gamma_2\quad\text{in }\vect^{\bfR}.
\]
The units of the insertions $\id\Rightarrow i_1h_1$ and $\id\Rightarrow g_2f_2$, transported by precomposition, give natural transformations
\[
\sigma_1:\ \pi_1^{*}\Gamma_1 \longrightarrow \delta^{*}i_1^{*}\Gamma_1,
\qquad
\sigma_2:\ \pi_2^{*}\Gamma_2 \longrightarrow \delta^{*}g_2^{*}\Gamma_2.
\]
Define $\Psi$ as the \emph{pullback} in the functor category $\vect^{\bfR}$:
\[
\Psi\ :=\ \mathrm{pullback}\left(
\pi_1^{*}\Gamma_1 \xrightarrow{\ \sigma_1\ } \delta^{*}i_1^{*}\Gamma_1
\ \xleftarrow[\ \cong\ ]{\ \delta^{*}\varphi\ }\ 
\delta^{*}g_2^{*}\Gamma_2 \xleftarrow{\ \sigma_2\ } \pi_2^{*}\Gamma_2
\right).
\]
Functor categories into $\vect$ admit limits computed pointwise, so this pullback exists canonically; the universal property immediately yields
$\iota_1^{*}\Psi\cong\Gamma_1$ and $\iota_2^{*}\Psi\cong\Gamma_2$, hence the endpoint isomorphisms. Uniqueness up to unique isomorphism follows from the universal property of the pullback.

By construction we have $\iota_1^{*}\Psi\cong\Gamma_1$ and $\iota_2^{*}\Psi\cong\Gamma_2$. 
Since precomposition is a (contravariant) $2$-functor, it is strictly functorial on composites:
\[
(\iota_1\circ g_1)^{*}=g_1^{*}\circ \iota_1^{*},
\qquad
(\iota_2\circ i_2)^{*}=i_2^{*}\circ \iota_2^{*}.
\]
Therefore
\[
(\iota_1\circ g_1)^{*}\Psi
= g_1^{*}(\iota_1^{*}\Psi)
\cong g_1^{*}\Gamma_1
\cong M,
\qquad
(\iota_2\circ i_2)^{*}\Psi
= i_2^{*}(\iota_2^{*}\Psi)
\cong i_2^{*}\Gamma_2
\cong O,
\]
where the last isomorphisms are those in the defining data of the two input couplings.
These composites are natural isomorphisms, being composites of natural isomorphisms.
\end{proof}

%%%%%%%%%%%%%%%%%%%%%%%%%%%%%%%%%%%%%%%%%%%%%%%%%%%%%%%%%%%%%%%%%
\subsection{Transport Distance}
\label{ssec:Galois_metric}
%%%%%%%%%%%%%%%%%%%%%%%%%%%%%%%%%%%%%%%%%%%%%%%%%%%%%%%%%%%%%%%%%

With composition available, we now define the transport distance and record its basic properties.

\begin{definition}
\label{def:transport-distance}
The \emph{Galois transport distance} between $M,N\in\vect^{\bfP}$ is
\[
\dGT(M,N):=\inf\{\ \cost(\Gamma)\ \mid\ \Gamma\ \text{is a Galois coupling of }(M,N)\ \}.
\]
If there is no Galois coupling between $M$ and $N$, set $\dGT(M,N)=\infty$.
\end{definition}

\begin{lemma}
\label{lem:cost-subadditive}
If $\Gamma_1$ is a coupling for $(M,N)$ and $\Gamma_2$ is a coupling for $(N,O)$, and $\Psi$ is their composite from Proposition~\ref{prop:compose-couplings}, then
\[
\cost(\Psi)\ \le\ \cost(\Gamma_1)\ +\ \cost(\Gamma_2).
\]
\end{lemma}

\begin{proof}
For $r\in\bfR$,
\[
\begin{aligned}
d_{\bfP}\big((f_1\pi_1)(r),\ (h_2\pi_2)(r)\big)
&\le d_{\bfP}\big((f_1\pi_1)(r),\ (h_1\pi_1)(r)\big) \\
&\quad + \underbrace{d_{\bfP}\big((h_1\pi_1)(r),\ (f_2\pi_2)(r)\big)}_{=\,0} \\
&\quad + d_{\bfP}\big((f_2\pi_2)(r),\ (h_2\pi_2)(r)\big),
\end{aligned}
\]
since $\delta=h_1\pi_1=f_2\pi_2$. Taking suprema and observing
\[
\sup_{r\in\bfR}d_{\bfP}\big(f_1\pi_1(r),h_1\pi_1(r)\big)=\sup_{q_1\in\bfQ_1}d_{\bfP}\big(f_1(q_1),h_1(q_1)\big)=\cost(\Gamma_1),
\]
(and similarly for $\Gamma_2$) yields the claim.
\end{proof}

\begin{proposition}
\label{prop:gt-pseudometric}
For a finite poset $\bfP$ with metric $d_{\bfP}$, the function
\[
\dGT:\ \mathrm{Ob}(\vect^{\bfP})\times \mathrm{Ob}(\vect^{\bfP})\longrightarrow [0,+\infty]
\]
is an extended pseudometric.
\end{proposition}

\begin{proof}
Nonnegativity is immediate. For any $M$, the identity coupling $\bfQ=\bfP$, $f=h=\id_{\bfP}$, $g=i=\id_{\bfP}$, $\Gamma=M$ has cost $0$, so $\dGT(M,M)=0$. Symmetry holds because swapping the two insertion legs $(f\dashv g,h\dashv i)$ of any coupling gives a coupling for $(N,M)$ with the same cost (the metric $d_{\bfP}$ is symmetric). The triangle inequality follows from Lemma~\ref{lem:cost-subadditive}.
\end{proof}

\begin{corollary}
\label{cor:gt-metric-on-iso}
On isomorphism classes, $\dGT$ is an extended metric: if $\dGT(M, N)=0$ then $M\cong N$.
\end{corollary}

\begin{proof}
Since $\bfP$ is finite, the set $\{d_{\bfP}(x,y)\mid x,y\in\bfP\}$ is finite; hence every coupling has cost in this finite set, and the infimum in the definition of $\dGT(M,N)$ is a \emph{minimum}. If $\dGT(M,N)=0$, there exists a coupling with $\cost(\Gamma)=0$, so $d_{\bfP}(f(q),h(q))=0$ for all $q\in\bfQ$, hence $f(q)=h(q)$ and therefore $f=h$ as maps $\bfQ\to\bfP$. Using Corollary~\ref{cor:3_adj},
\[
M\ \cong\ f_{!}\Gamma\ =\ h_{!}\Gamma\ \cong\ N,
\]
so $M\cong N$.
\end{proof}

\paragraph{Relation to interleavings.}
Over the totally ordered real line, the Galois transport distance coincides with the classical interleaving distance; see Appendix~\ref{app:interleaving}.



%%%%%%%%%%%%%%%%%%%%%%%%%%%%%%%%%%%%%%%%%%%%%%%%%%%%%%%%%%%%%%%%%
\subsection{Examples}
%%%%%%%%%%%%%%%%%%%%%%%%%%%%%%%%%%%%%%%%%%%%%%%%%%%%%%%%%%%%%%%%%

We now present two illustrative examples—one in the 1-parameter setting
and one in the 2-parameter setting.  
These will serve as running test cases throughout the paper for the 
Galois transport distance and its comparison with later constructions.

\begin{example}\label{ex:gtd-chain}
Let $\bfP=\{1<2<3<4\}$ with metric $d_{\bfP}(i,j)=|i-j|$.  
For $a<b$ write $I[a,b)$ for the interval $\bfP$-module supported on 
$\{a,a+1,\dots,b-1\}$ with identity maps.  
Define
\[
M:=I[1,3)\oplus I[3,4),\qquad 
N:=I[2,4).
\]

\smallskip
\noindent
To construct a low-cost coupling, take the apex poset 
$\bfQ:=\{0<1<2<3<4<5\}$ and define $f,h:\bfQ\to\bfP$ by
\[
\begin{array}{c|cccccc}
q & 0 & 1 & 2 & 3 & 4 & 5\\ \hline
f(q) & 1 & 1 & 2 & 3 & 4 & 4\\
h(q) & 1 & 2 & 2 & 3 & 4 & 4
\end{array}
\]
with right adjoints
\[
g(x):=\max\{q\in\bfQ \mid f(q)\le x\},\qquad
i(x):=\max\{q\in\bfQ \mid h(q)\le x\}.
\]
A short computation shows that $f\dashv g$ and $h\dashv i$ are Galois insertions.

Let
\[
\Gamma := I^{\bfQ}[1,3)\ \oplus\ I^{\bfQ}[3,4),
\]
the sum of the corresponding interval modules on $\bfQ$.  
Then 
\[
g^*\Gamma \cong M,\qquad i^*\Gamma \cong N,
\]
so $(\bfQ,f\dashv g,h\dashv i,\Gamma)$ is a coupling.

\smallskip
\noindent
The coupling cost is
\[
\cost(\Gamma)=\sup_{q\in\bfQ} d_{\bfP}(f(q),h(q))=1.
\]
Since $\dGT(M,N)$ assumes only integer values and $M\not\cong N$, 
we conclude
\[
\dGT(M,N)=1.
\]

\smallskip
\noindent
A summary of the data appears in Table~\ref{tab:gtd-chain}.
\begin{table}[h]
\centering
\caption{Coupling data for Example~\ref{ex:gtd-chain}.}
\label{tab:gtd-chain}
\begin{tabular}{c|c}
Quantity & Value \\ \hline
Modules & $M=I[1,3)\oplus I[3,4),\quad N=I[2,4)$ \\
Apex poset & $\bfQ=\{0,\dots,5\}$ \\
Adjoints & $f\dashv g,\ h\dashv i$ (defined above) \\
Apex module & $\Gamma=I^{\bfQ}[1,3)\oplus I^{\bfQ}[3,4)$ \\
Cost & $\cost(\Gamma)=1$ \\
GTD & $\dGT(M,N)=1$
\end{tabular}
\end{table}
\end{example}

\begin{example}\label{ex:gtd-2d}
Let $\bfP=\{1,\dots,8\}^2$ with the product order and 
$d_{\bfP}((i,j),(i',j'))=\max\{|i-i'|,|j-j'|\}$.  
For $(a_1,a_2)\le(b_1,b_2)$ let 
$J[(a_1,a_2),(b_1,b_2))$ denote the rectangle interval module.

Define two $2\times 2$ squares
\[
R_1:=J[(2,5),(4,7)],\qquad 
R_2:=J[(5,2),(7,4)],
\]
and set $M^{(2)}:=R_1\oplus R_2$.  
Define the $L$-shaped module
\[
A:=J[(1,4),(5,8)],\qquad
B:=J[(4,1),(8,5)],\qquad
S:=J[(4,4),(5,5)],\qquad
0\to S\xrightarrow{(\iota,-\iota)}A\oplus B\to N^{(2)}\to 0.
\]
The supports of $M^{(2)}$ and $N^{(2)}$ are shown in 
Figure~\ref{fig:gtd-2d-modules}.

\smallskip
\noindent
To couple these modules, take the apex 
$\bfQ:=\bfP_L\sqcup\bfP_R$ (two disjoint copies of $\bfP$) and
define
\[
f:\bfQ\rightleftarrows\bfP:g,\qquad 
h:\bfQ\rightleftarrows\bfP:i,
\]
with
\[
f(x)=x\ \text{on both copies},\qquad
h(x)=\begin{cases}
\min(x+(1,1),(8,8)), & x\in\bfP_L,\\
x, & x\in\bfP_R,
\end{cases}
\]
and $g,i$ the corresponding right adjoints.  
Set
\[
\Gamma|_{\bfP_L}=M^{(2)},\qquad \Gamma|_{\bfP_R}=N^{(2)}.
\]
Then 
\[
g^*\Gamma\cong M^{(2)},\qquad i^*\Gamma\cong N^{(2)}.
\]

Since points in $\bfP_L$ move by at most $(+1,+1)$ while $\bfP_R$ is fixed,
\[
\cost(\Gamma)=1.
\]
Thus
\[
\dGT^{\bfP}\bigl(M^{(2)},N^{(2)}\bigr)\le 1.
\]

\smallskip
\noindent
A summary of this data appears in Table~\ref{tab:gtd-2d}.

\begin{table}[h]
\centering
\caption{Coupling data for Example~\ref{ex:gtd-2d}.}
\label{tab:gtd-2d}
\begin{tabular}{c|c}
Quantity & Value \\ \hline
Modules & $M^{(2)}=R_1\oplus R_2,\quad N^{(2)}$ (L-shape) \\
Apex poset & $\bfQ=\bfP_L\sqcup\bfP_R$ \\
Adjoints & $f\dashv g,\ h\dashv i$ \\
Apex module & $\Gamma|_{\bfP_L}=M^{(2)},\ \Gamma|_{\bfP_R}=N^{(2)}$ \\
Cost & $1$ \\
GTD & $\le 1$ (in fact $=1$)
\end{tabular}
\end{table}
\end{example}



%%%%%%%%%%%%%%%%%%%%%%%%%%%%%%%%%%%%%%%%%%%%%%%%%%%
%%%%%%%%%%%%%%%%%%%%%%%%%%%%%%%%%%%%%%%%%%%%%%%%%%%
\section{Bottleneck Distance}
%%%%%%%%%%%%%%%%%%%%%%%%%%%%%%%%%%%%%%%%%%%%%%%%%%%
%%%%%%%%%%%%%%%%%%%%%%%%%%%%%%%%%%%%%%%%%%%%%%%%%%%

We adopt the following conventions. For $n\in\bbN$ write $[n]=\{1,\dots,n\}$. If $X \cong \bigoplus_{i=1}^{n} X_i$ is a finite direct sum of indecomposables, write $\Summands(X)=\{X_i \mid i\in [n]\}$; for finite sets $A,B$ with $|A|=|B|$, let $\Bij(A,B)$ be the set of bijections $A\to B$. Throughout, $(\bfP,d_{\bfP})$ is a finite metric poset.

For each $M\in\vect^{\bfP}$ fix once and for all a decomposition $M\cong\Ds_{i=1}^n M_i$ into indecomposables. Set $|M|:=n$ (the \emph{size}) and $\Summands(M):=\{M_i\mid i\in[n]\}$ (the \emph{summand set}); these are well defined up to isomorphism and the elements of $\Summands(M)$ are pairwise distinct. A minimal projective resolution of $M$ is denoted $P\down^M$ and is unique up to isomorphism of exact complexes. Let
\[
\Res(M)\quad\text{denote the set of all projective resolutions of }M.
\]
If $R\down=(R_i,\partial_i)_{i\ge 0}\in\Res(M)$, its \emph{size vector} is $|R\down|:=(|R_i|)_{i\ge 0}$. For indecomposable projectives $U,V$ (identified with representables $U\cong\bfP(x,-)$, $V\cong\bfP(y,-)$), put
\[
\dist(U,V):=d_{\bfP}(x,y).
\]

If $E$ is any projective object in $\vect^{\bfP}$, the mapping cone $\Cone(\id_E)$ is the two-term contractible complex
\[
\cdots \to 0 \to E \xrightarrow{\ \id\ } E \to 0 \to \cdots,
\]
concentrated in consecutive degrees; its shift $\Cone(\id_E)[a]$ places the two copies of $E$ in degrees $a$ and $a-1$. Direct-summing $\Cone(\id_E)[a]$ with a projective resolution leaves the resolved module unchanged and yields a chain-homotopy equivalent resolution (we call this \emph{padding by a contractible cone}).

\begin{lemma}
\label{lem:all-prj-resol}
Every projective resolution of $M$ is obtained from the minimal one by padding with contractible cones:
\[
\Res(M)
\ =\
\Bigl\{
\ P\down^M\ \ds\ \Ds_{i\in[n]}\,\Cone(\id_{E_i})[a_i]\
\Bigm|\ n\in\bbN,\ a_i\in\bbN,\ a_i\ge 1,\ E_i\in\prj\bfP\ 
\Bigr\}.
\]
\end{lemma}

\begin{remark}
Insert Remark here about Mobius homology.
\end{remark}

%%%%%%%%%%%%%%%%%%%%%%%%%%%%%%%%%%%%%%%%%%%%%%%%%%%%%%%%%%%%%%%%%%%%%%%%%%%%%%%%%%%%%%%%%%%%%%%%%%%%%%%%%%%%%%%%%%
\subsection{Matchings}
%%%%%%%%%%%%%%%%%%%%%%%%%%%%%%%%%%%%%%%%%%%%%%%%%%%%%%%%%%%%%%%%%%%%%%%%%%%%%%%%%%%%%%%%%%%%%%%%%%%%%%%%%%%%%%%%%%

We now define degreewise matchings between two resolutions. Given $M,N\in\vect^{\bfP}$, consider pairs of projective resolutions with the same size vector:
\[
\Res(P\down^M,P\down^N)
\ :=\
\bigl\{\, (E\down,F\down)\in\Res(M)\times\Res(N)\ \bigm|\ |E\down|=|F\down|\,\bigr\}.
\]

\begin{proposition}
\label{prop:common-padding}
If the size vectors of $P\down^M$ and $P\down^N$ are finitely supported and have the same alternating sum
\[
\sum_{i\ge 0}(-1)^i|P^M_i|\ =\ \sum_{i\ge 0}(-1)^i|P^N_i|,
\]
then $\Res(P\down^M,P\down^N)\neq\emptyset$.
\end{proposition}

\begin{proof}
Let $p=(|P^M_i|)$ and $q=(|P^N_i|)$. Adding $\Cone(\id_E)[a]$ with $a\ge 1$ increases the size vector by the elementary vector $e^{(a)}$ having $1$ in degrees $a$ and $a-1$ (and $0$ elsewhere). These $e^{(a)}$ generate, as an abelian group, the kernel of the alternating-sum map $\alpha:\bigoplus_{i\ge 0}\bbZ\to\bbZ$, $\alpha(r)=\sum_i(-1)^ir_i$. Since $\alpha(p)=\alpha(q)$, the difference $q-p$ lies in this kernel and can be written $T^+-T^-$ with $T^\pm$ finite nonnegative combinations of the $e^{(a)}$. Then $p+T^+=q+T^-$ is a common upper bound, realized by padding $P\down^M$ by the cones in $T^+$ and $P\down^N$ by those in $T^-$. Hence there exist $(E\down,F\down)$ with $|E\down|=|F\down|$.
\end{proof}

For $(E\down,F\down)\in\Res(P\down^M,P\down^N)$, a \emph{matching} is a family of degreewise bijections
\[
B=(B_i)_{i\ge 0},\qquad B_i\in\Bij\bigl(\Summands(E_i),\,\Summands(F_i)\bigr).
\]
Define its \emph{cost} as the $L^\infty$–type aggregate of the underlying poset metric,
\[
\cost(B)\ :=\
\sup\bigl\{\,\dist\bigl(U,\ B_i(U)\bigr)\ \bigm|\ i\ge 0,\ U\in\Summands(E_i)\,\bigr\}.
\]
(Equivalently, $\cost(B)=\sup\{\dist(B_i^{-1}(V),V)\mid i\ge 0,\ V\in\Summands(F_i)\}$.)  
We refer to the quantity
\[
\dist_R(E\down,F\down)\ :=\ \inf_{\,B\in\Match(E\down,F\down)}\ \cost(B)
\]
as the \emph{matching distance} for resolutions with a fixed size vector.


\begin{lemma}\label{lem:reg-triangle}
If $|E\down|=|F\down|=|G\down|$, then
\[
\dist_R(E\down,G\down)\ \le\ \dist_R(E\down,F\down)\ +\ \dist_R(F\down,G\down).
\]
\end{lemma}

%%%%%%%%%%%%%%%%%%%%%%%%%%%%%%%%%%%%%%%%%%%%%%%%%%%%%%%%%
%%%%%%%%%%%%%%%%%%%%%%%%%%%%%%%%%%%%%%%%%%%%%%%%%%%%%%%%%
\subsection{Matching Distance}
%%%%%%%%%%%%%%%%%%%%%%%%%%%%%%%%%%%%%%%%%%%%%%%%%%%%%%%%%
%%%%%%%%%%%%%%%%%%%%%%%%%%%%%%%%%%%%%%%%%%%%%%%%%%%%%%%%%

With matchings and \(\dist_R\) in hand, we define the global distance by allowing padding.

\begin{definition}
For minimal resolutions $P\down^M$ and $P\down^N$ (not necessarily of the same size), the \emph{bottleneck distance} is
\[
\dist_B\bigl(P\down^M,P\down^N\bigr)
\ :=\ 
\inf_{\ (E\down,F\down)\in\Res(P\down^M,P\down^N)}\ \dist_R(E\down,F\down).
\]
We adopt the extended-value convention that $\dist_B\bigl(P\down^M,P\down^N\bigr)=\infty$ if there is no compatible padding (i.e.\ $\Res(P\down^M,P\down^N)=\varnothing$).
\end{definition}

Proposition~\ref{prop:common-padding} gives a sufficient condition for finiteness (equality of alternating sums); in general, $\dist_B$ may be infinite.

\begin{proposition}
\label{prop:bneck-pseudometric}
On the class of minimal projective resolutions $\{P\down^M\mid M\in\vect^{\bfP}\}$, the function $\dist_B$ is an extended metric:
\[
\dist_B:\ \{P\down^M\}\times\{P\down^N\}\longrightarrow [0,+\infty].
\]
\end{proposition}

\begin{proof}
Nonnegativity is immediate. For any $M$, take the same padding on both sides and the identity matching degreewise to obtain $\dist_B(P\down^M,P\down^M)=0$. Symmetry holds because each matching family $B=(B_i)$ has an inverse family of the same cost. For the triangle inequality, given $P\down^M,P\down^N,P\down^O$, choose paddings producing $(E\down,F\down)$ and $(F'\down,G\down)$ with equal size vectors for $(M,N)$ and $(N,O)$ and with $\dist_R$ arbitrarily close to $\dist_B(M,N)$ and $\dist_B(N,O)$. Pad further to a common refinement so both pairs share the same size vector, compose the degreewise bijections, and apply Lemma~\ref{lem:reg-triangle}; taking infima over paddings yields the inequality. Finally, if $\dist_B(P\down^M,P\down^N)=0$, the costs lie in the finite set $\{d_{\bfP}(x,y)\mid x,y\in\bfP\}$, so the infimum is attained: there exist paddings and a matching of cost $0$. Thus each matched pair of summands has equal index in $\bfP$, the padded resolutions agree termwise up to isomorphism, and removing contractible cones yields $P\down^M\cong P\down^N$. Hence $\dist_B$ is an extended metric.
\end{proof}

%%%%%%%%%%%%%%%%%%%%%%%%%%%%%%%%%%%%%%%%%%%%%%%%%%%%%%%%%%%%%%%%%%%%
\subsection{Examples}
%%%%%%%%%%%%%%%%%%%%%%%%%%%%%%%%%%%%%%%%%%%%%%%%%%%%%%%%%%%%%%%%%%%%

We now compute bottleneck distances for our 1D and 2D running examples by
comparing minimal projective resolutions and equalizing degreewise sizes
via contractible cones.

\begin{example}\label{ex:bneck-chain}
Let $\bfP=\{1<2<3<4\}$ and consider
\[
M=I[1,3)\oplus I[3,4),\qquad 
N=I[2,4)
\]
as in Example~\ref{ex:gtd-chain}.

Minimal resolutions of the summands are
\[
\begin{aligned}
P\down^{I[1,3)}&:\ 0\to\bfP(3,-)\to\bfP(1,-)\to 0,\\
P\down^{I[3,4)}&:\ 0\to\bfP(4,-)\to\bfP(3,-)\to 0,\\
P\down^{I[2,4)}&:\ 0\to\bfP(4,-)\to\bfP(2,-)\to 0.
\end{aligned}
\]
Summing yields
\[
\begin{array}{c|c|c}
 & \text{degree }1 & \text{degree }0 \\ \hline
P\down^{M} & \bfP(3,-)\oplus\bfP(4,-) & \bfP(1,-)\oplus\bfP(3,-)\\
P\down^{N} & \bfP(4,-) & \bfP(2,-)
\end{array}
\]
so $|P\down^M|=(2,2,0,\dots)$ and $|P\down^N|=(1,1,0,\dots)$.
Since the alternating sums agree, padding is possible.

Pad $P\down^{N}$ by $\Cone(\id_{\bfP(3,-)})[1]$, adding $\bfP(3,-)$ in both
degrees $1$ and $0$:
\[
\begin{array}{c|c|c}
 & \text{degree }1 & \text{degree }0 \\ \hline
P\down^{N}\oplus\Cone(\id_{\bfP(3,-)})[1]
 & \bfP(4,-)\oplus\bfP(3,-) 
 & \bfP(2,-)\oplus\bfP(3,-)
\end{array}
\]
which matches the size vector of $P\down^{M}$.

A degreewise matching is given by
\[
\begin{aligned}
\text{degree }0:&\ 
\bfP(1,-)\longrightarrow \bfP(2,-),
\qquad\bfP(3,-)\longrightarrow \bfP(3,-),\\
\text{degree }1:&\ 
\bfP(4,-)\longrightarrow \bfP(4,-),
\qquad\bfP(3,-)\longrightarrow \bfP(3,-).
\end{aligned}
\]
All pairs except $\bfP(1,-)\to\bfP(2,-)$ have cost $0$; that pair has cost
$d_{\bfP}(1,2)=1$.  Thus
\[
\dist_B(P\down^{M},P\down^{N})=1.
\]

\begin{table}[h]
\centering
\caption{Degreewise structure in Example~\ref{ex:bneck-chain}.}
\label{tab:bneck-chain}
\begin{tabular}{c|c|c}
 & $P\down^{M}$ & $P\down^{N}$ (padded) \\ \hline
degree $0$ & $\bfP(1,-),\ \bfP(3,-)$ & $\bfP(2,-),\ \bfP(3,-)$\\
degree $1$ & $\bfP(3,-),\ \bfP(4,-)$ & $\bfP(4,-),\ \bfP(3,-)$ 
\end{tabular}
\end{table}

\end{example}

\begin{example}\label{ex:bneck-2d}
Let $\bfP=\{1,\dots,8\}^2$ and consider the modules
$M^{(2)},N^{(2)}\in\vect^{\bfP}$ from Example~\ref{ex:gtd-2d}.

A rectangle $J[(a_1,a_2),(b_1,b_2))$ has minimal resolution
\[
0\to 
\bfP((b_1,b_2),-)
\to
\bfP((a_1,b_2),-)\oplus\bfP((b_1,a_2),-)
\to
\bfP((a_1,a_2),-)
\to J[(a_1,a_2),(b_1,b_2))
\to 0.
\]

Thus each $2\times2$ square contributes  
degree $0$: $1$,\quad degree $1$: $2$,\quad degree $2$: $1$,
so
\[
|P\down^{M^{(2)}}|=
(\text{deg }0:2,\ \text{deg }1:4,\ \text{deg }2:2).
\]

For the L-shape (pushout of $A$ and $B$ along $S$), gluing resolutions yields
\[
|P\down^{N^{(2)}}|=
(\text{deg }0:2,\ \text{deg }1:3,\ \text{deg }2:1).
\]

Pad $P\down^{N^{(2)}}$ by $\Cone(\id_{\bfP((7,4),-)})[2]$, producing
\[
|\,P\down^{N^{(2)}}\oplus\Cone(\id_{\bfP((7,4),-)})[2]\,|
=(2,4,2),
\]
which matches $|P\down^{M^{(2)}}|$.

The matching pairs in each degree are recorded in
Table~\ref{tab:bneck-2d-arrows}, all with cost at most $1$.

\begin{table}[h]
\centering
\caption{Degreewise matching for Example~\ref{ex:bneck-2d}. 
Arrows indicate $P\down^{M^{(2)}}\to P\down^{N^{(2)}}$ (padded).}
\label{tab:bneck-2d-arrows}
\begin{tabular}{c|p{7.2cm}}
\textbf{degree} & \textbf{matching pairs} \\ \hline

0 
& $\bfP((2,5),-)\ \longrightarrow\ \bfP((1,4),-)$\\[-2pt]
& $\bfP((5,2),-)\ \longrightarrow\ \bfP((4,1),-)$
\\[6pt]

1
& $\bfP((4,5),-)\ \longrightarrow\ \bfP((4,5),-)$\\[-2pt]
& $\bfP((5,4),-)\ \longrightarrow\ \bfP((5,4),-)$\\[-2pt]
& $\bfP((2,7),-)\ \longrightarrow\ \bfP((1,8),-)$\\[-2pt]
& $\bfP((7,2),-)\ \longrightarrow\ \bfP((8,1),-)$
\\[6pt]

2
& $\bfP((4,7),-)\ \longrightarrow\ \bfP((5,8),-)$\\[-2pt]
& $\bfP((7,4),-)\ \longrightarrow\ \bfP((7,4),-)$ (cone)
\end{tabular}
\end{table}

Since all moves have cost $\le 1$ and at least one has cost $1$, we obtain
\[
\dist_B(P\down^{M^{(2)}},P\down^{N^{(2)}})=1.
\]
\end{example}



%%%%%%%%%%%%%%%%%%%%%%%%%%%%%%%%%%%%%%%%%%%%%%%%%%%%%%%%%%%
%%%%%%%%%%%%%%%%%%%%%%%%%%%%%%%%%%%%%%%%%%%%%%%%%%%%%%%%%%%
\section{Stability Theorem}
%%%%%%%%%%%%%%%%%%%%%%%%%%%%%%%%%%%%%%%%%%%%%%%%%%%%%%%%%%%
%%%%%%%%%%%%%%%%%%%%%%%%%%%%%%%%%%%%%%%%%%%%%%%%%%%%%%%%%%%

We now relate the two distances defined above. Informally: a Galois coupling of $M$ and $N$ controls, via restriction, a pair of projective resolutions whose degreewise summands can be matched with cost bounded by the coupling cost. Hence the bottleneck distance between minimal projective resolutions is at most the Galois transport distance.

The next lemma says that pulling back along the right adjoint of a Galois connection sends the indecomposable projective at $x\in\bfQ$ to the indecomposable projective at $f(x)\in\bfP$.

\begin{lemma}
\label{lem:rt-adj-prj-ind}
If $f : \bfQ \rightleftarrows \bfP : g$ is a Galois connection of posets, then
\[
g^\ast\bigl(\bfQ(x,\blank)\bigr)\ \cong\ \bfP\bigl(f(x), \blank\bigr)
\qquad\text{in }\vect^\bfP\ \text{for all }x\in\bfQ.
\]
\end{lemma}

\begin{proof}
By Corollary~\ref{cor:3_adj} we have a natural isomorphism $g^{*}\cong f_{!}$. For any $M\in\vect^{\bfP}$,
\[
\Hom_{\vect^{\bfP}}\bigl(\bfP(f(x),-),\,M\bigr)\ \cong\ M\bigl(f(x)\bigr)
\ \cong\ (f^{*}M)(x)
\ \cong\ \Hom_{\vect^{\bfQ}}\bigl(\bfQ(x,-),\,f^{*}M\bigr)
\ \cong\ \Hom_{\vect^{\bfP}}\bigl(f_{!}\bfQ(x,-),\,M\bigr).
\]
By Yoneda, $f_{!}\bfQ(x,-)\cong \bfP(f(x),-)$, hence $g^{*}\bfQ(x,-)\cong f_{!}\bfQ(x,-)\cong \bfP(f(x),-)$.
\end{proof}

\begin{theorem}[Stability]
\label{thm:triangle}
Let $(\bfP, d_\bfP)$ be a finite metric poset. Then for any $M, N \in \vect^\bfP$,
\[
\dist_B\bigl(P\down^M, P\down^N\bigr)\ \le\ \dGT(M, N).
\]
\end{theorem}

\begin{proof}
If there is no Galois coupling of $(M,N)$, then $\dGT(M,N)=\infty$ and the claim is tautological. Otherwise fix $\varepsilon>0$ and choose a coupling $(\bfQ, f \dashv g,\ h \dashv i,\ \Gamma)$ with
\[
g^\ast\Gamma\cong M,\qquad i^\ast\Gamma\cong N,\qquad \cost(\Gamma)\ \le\ \dGT(M,N)+\varepsilon.
\]
Let $R\down\to\Gamma$ be any projective resolution in $\vect^{\bfQ}$. Since precomposition is exact (Proposition~\ref{prop:Kan-posets}) and, for a Galois connection, preserves projectives (Proposition~\ref{prop:pullback_preserves_projectives}), the complexes
\[
E\down:=g^{*}R\down\quad\text{and}\quad F\down:=i^{*}R\down
\]
are projective resolutions of $M$ and $N$ in $\vect^{\bfP}$.

Write each degree $R_i$ as a finite direct sum of representables $R_i\cong\bigoplus_{x\in S_i}\bfQ(x,-)$ (Lemma~\ref{lem:ind-proj}). By Lemma~\ref{lem:rt-adj-prj-ind} (and the analogue for $h\dashv i$),
\[
E_i\ \cong\ \bigoplus_{x\in S_i}\bfP\bigl(f(x),-\bigr),
\qquad
F_i\ \cong\ \bigoplus_{x\in S_i}\bfP\bigl(h(x),-\bigr).
\]
Hence $|E\down|=|F\down|$ and $(E\down,F\down)\in\Res(P\down^M,P\down^N)$. Define the degreewise matching $B_i$ by the identity on indices $x\in S_i$:
\[
B_i:\ \bfP\bigl(f(x),-\bigr)\longmapsto \bfP\bigl(h(x),-\bigr)\qquad(x\in S_i).
\]
Then
\[
\dist_R(E\down,F\down)\ \le\ \cost(B)\ =\ \sup_{i}\ \sup_{x\in S_i} d_{\bfP}\bigl(f(x),h(x)\bigr)
\ \le\ \sup_{x\in\bfQ} d_{\bfP}\bigl(f(x),h(x)\bigr)\ =\ \cost(\Gamma).
\]
Taking the infimum over all compatible paddings yields
\[
\dist_B\bigl(P\down^M,P\down^N\bigr)\ \le\ \dist_R(E\down,F\down)\ \le\ \cost(\Gamma)\ \le\ \dGT(M,N)+\varepsilon.
\]
Letting $\varepsilon\to 0$ completes the proof.
\end{proof}


%%%%%%%%%%%%%%%%%%%%%%%%%%%%%%%%%%%%%%%%%%%%%%%%%%%%%%%%
\subsection{Examples}
%%%%%%%%%%%%%%%%%%%%%%%%%%%%%%%%%%%%%%%%%%%%%%%%%%%%%%%%

We now revisit our running 1D and 2D examples to illustrate the stability
inequality.  
In both cases the Galois transport distance and the bottleneck distance
coincide, showing that the bound in Theorem~\ref{thm:triangle} is sharp.


\begin{example}\label{ex:stability-chain}
From Examples~\ref{ex:gtd-chain} and~\ref{ex:bneck-chain} we have
\[
\dGT(M,N)=1,
\qquad
\dist_B(P\down^{M},P\down^{N})=1.
\]
Hence the stability inequality
\[
\dist_B(P\down^{M},P\down^{N})\ \le\ \dGT(M,N)
\]
holds with equality.

The transport computation detects a unit shift in the birth coordinate,
and the bottleneck computation recovers the same value after padding by a
single cone.  
Thus stability is numerically tight in this 1D case.
\end{example}


\begin{example}\label{ex:stability-2d}
For the 2D modules $M^{(2)}$ and $N^{(2)}$ from Example~\ref{ex:gtd-2d},
Example~\ref{ex:gtd-2d} gives
\[
\dGT^{\bfP}(M^{(2)},N^{(2)})\le 1,
\]
and Example~\ref{ex:bneck-2d} gives
\[
\dist_B(P\down^{M^{(2)}},P\down^{N^{(2)}})\le 1.
\]
Therefore
\[
\dist_B(P\down^{M^{(2)}},P\down^{N^{(2)}})
\ \le\
\dGT^{\bfP}(M^{(2)},N^{(2)}),
\]
and in this symmetric placement both sides equal \(1\).

Here the transport side shifts each square of $M^{(2)}$ by $(+1,+1)$ into
the arms of the L–shape, while on the resolution side, padding by a single
cone yields a matching of identical cost.  
Stability is therefore sharp in this 2D example as well.
\end{example}





%%%%%%%%%%%%%%%%%%%%%%%%%%%%%%%%%%%%%%%%%%%%%%%%%%%%%%%%%%%
%%%%%%%%%%%%%%%%%%%%%%%%%%%%%%%%%%%%%%%%%%%%%%%%%%%%%%%%%%%
\section{Application to Persistence}
%%%%%%%%%%%%%%%%%%%%%%%%%%%%%%%%%%%%%%%%%%%%%%%%%%%%%%%%%%%
%%%%%%%%%%%%%%%%%%%%%%%%%%%%%%%%%%%%%%%%%%%%%%%%%%%%%%%%%%%

We extract a persistence construction from a $\bfP$-module by passing to an interval poset and taking kernels of structure maps. The resulting “diagram” admits a stability inequality that recovers the classical bottleneck stability when $\bfP$ is totally ordered.

\begin{definition}
Let $(\bfP,d_{\bfP})$ be a finite metric poset with a top element $\top$.
We extend $d_{\bfP}$ to an \emph{extended} metric by
\[
d_{\bfP}(x,\top)=d_{\bfP}(\top,x)=+\infty\ \ (x\neq \top),\qquad d_{\bfP}(\top,\top)=0.
\]
Define the \emph{interval poset} $\Int\bfP$ to have objects the intervals $[x,y]$ with $x\le y\le \top$ in $\bfP$ and order
\[
[x_1,y_1]\ \le\ [x_2,y_2]\quad\Longleftrightarrow\quad x_1\le x_2\ \text{ and }\ y_1\le y_2.
\]
Equip $\Int\bfP$ with the product $L^\infty$ extended metric
\[
d_{\Int\bfP}\bigl([x_1,y_1],[x_2,y_2]\bigr)\ :=\ \max\{\,d_{\bfP}(x_1,x_2),\ d_{\bfP}(y_1,y_2)\,\}.
\]
\end{definition}

\begin{definition}
For $M\in\vect^{\bfP}$ define $K(M)\in\vect^{\Int\bfP}$ by
\[
K(M)([x,y])\ :=\
\begin{cases}
\ker\bigl(M(x\to y)\bigr), & y<\top,\\[2pt]
M(x), & y=\top,
\end{cases}
\]
and for a relation $[x_1,y_1]\le [x_2,y_2]$ let $K(M)([x_1,y_1]\to [x_2,y_2])$ be the map induced by $M(x_1\to x_2)$, which sends $\ker(M(x_1\to y_1))$ into $\ker(M(x_2\to y_2))$ when $y_2<\top$ by functoriality of $M$. This defines a functor $K:\vect^{\bfP}\to\vect^{\Int\bfP}$.
\end{definition}

We record two lemmas that will be used implicitly below.

\begin{lemma}\label{lem:Int-Galois}
If $f:\bfQ \rightleftarrows \bfP:g$ is a Galois connection, then so is
\[
\Int(f):\Int\bfQ \rightleftarrows \Int\bfP:\Int(g),
\qquad
\Int(f)[u,v]=[f(u),f(v)],\ \ \Int(g)[x,y]=[g(x),g(y)].
\]
\end{lemma}

\begin{proof}
Monotonicity of $\Int(f)$ and $\Int(g)$ follows from that of $f$ and $g$.
For $[u_1,v_1]\in\Int\bfQ$ and $[x_2,y_2]\in\Int\bfP$,
\[
\begin{aligned}
\Int(f)[u_1,v_1]\le [x_2,y_2]
&\iff \bigl(f(u_1)\le x_2\bigr)\ \text{and}\ \bigl(f(v_1)\le y_2\bigr)\\
&\iff \bigl(u_1\le g(x_2)\bigr)\ \text{and}\ \bigl(v_1\le g(y_2)\bigr)\\
&\iff [u_1,v_1]\le \Int(g)[x_2,y_2],
\end{aligned}
\]
using $f\dashv g$ coordinatewise. Thus $\Int(f)\dashv\Int(g)$.
\end{proof}

\begin{lemma}\label{lem:Int-Lipschitz}
For any monotone maps $f,h:\bfQ\to\bfP$,
\[
\sup_{[u,v]\in\Int\bfQ}
d_{\Int\bfP}\bigl(\Int(f)[u,v],\,\Int(h)[u,v]\bigr)
\ \le\
\sup_{u\in\bfQ} d_{\bfP}\bigl(f(u),\,h(u)\bigr).
\]
\end{lemma}

\begin{proof}
By the $L^\infty$ metric on $\Int\bfP$,
\[
d_{\Int\bfP}\bigl([f(u),f(v)],\,[h(u),h(v)]\bigr)
=\max\bigl\{\,d_{\bfP}(f(u),h(u)),\ d_{\bfP}(f(v),h(v))\,\bigr\}
\le \sup_{w\in\bfQ} d_{\bfP}(f(w),h(w)).
\]
Taking the supremum over all $[u,v]\in\Int\bfQ$ gives the claim.
\end{proof}

\begin{lemma}\label{lem:K-commute}
For any monotone $g:\bfP\to\bfQ$ there is a natural isomorphism of functors
\[
\Int(g)^{*}\circ K\ \ \cong\ \ K\circ g^{*}:\ \vect^{\bfQ}\longrightarrow \vect^{\Int\bfP}.
\]
\end{lemma}

\begin{proof}
Evaluate both composites at $M\in\vect^{\bfQ}$ and $[x,y]\in\Int\bfP$. If $y<\top$,
\[
(\Int(g)^{*}K(M))([x,y])\ =\ K(M)\bigl([g(x),g(y)]\bigr)\ =\ \ker\bigl(M(g(x)\to g(y))\bigr),
\]
\[
(K(g^{*}M))([x,y])\ =\ \ker\bigl((g^{*}M)(x\to y)\bigr)\ =\ \ker\bigl(M(g(x)\to g(y))\bigr).
\]
If $y=\top$, both sides equal $M(g(x))$. Naturality in $[x,y]$ follows from functoriality of $M$.
\end{proof}

\begin{proposition}\label{prop:K-Lipschitz}
For all $M,N\in\vect^{\bfP}$ one has
\[
\dGT^{\Int\bfP}\bigl(K(M),K(N)\bigr)\ \le\ \dGT^{\bfP}(M,N).
\]
\end{proposition}

\begin{proof}
Given a Galois coupling $(\bfQ,f\dashv g,\ h\dashv i,\ \Gamma)$ for $(M,N)$ in $\vect^{\bfP}$, Lemma~\ref{lem:Int-Galois} yields a Galois coupling
\(
\Int(f)\dashv\Int(g),\ \Int(h)\dashv\Int(i)
\)
on $\Int\bfQ\rightleftarrows\Int\bfP$. Take the apex module $K(\Gamma)\in\vect^{\Int\bfQ}$. By Lemma~\ref{lem:K-commute},
\[
\Int(g)^{*}K(\Gamma)\ \cong\ K(g^{*}\Gamma)\ \cong\ K(M),\qquad
\Int(i)^{*}K(\Gamma)\ \cong\ K(i^{*}\Gamma)\ \cong\ K(N),
\]
so this is a coupling for $(K(M),K(N))$. Its cost is bounded by Lemma~\ref{lem:Int-Lipschitz}. Taking infima gives the claim.
\end{proof}

\begin{definition}
For $M\in\vect^{\bfP}$, let $K\down^{M}$ denote a minimal projective resolution of $K(M)$ in $\vect^{\Int\bfP}$. We call the family of its degreewise indecomposable projective summands
\[
\bigl\{\Summands\bigl(K^{M}_i\bigr)\bigr\}_{i\ge 0}
\]
the \emph{persistence diagram} of $M$ over $\bfP$. (Equivalently, the isomorphism class of the minimal resolution $K\down^{M}$ encodes the diagram.)
\end{definition}

Padding a resolution by contractible cones $\Cone(\id_E)[a]$ adds one copy of $E$ in consecutive degrees $a$ and $a-1$ without changing the resolved module. In our bottleneck framework this plays the role of adding \emph{diagonal points} in classical matching: it equalizes degreewise sizes, and cones added symmetrically on both sides can be matched at zero cost. Thus diagonal padding is implemented categorically by homotopically trivial cones.

\begin{theorem}
For all $M,N\in\vect^{\bfP}$,
\[
\dist_B\bigl(K\down^{M},K\down^{N}\bigr)\ \le\ \dGT^{\Int\bfP}\bigl(K(M),K(N)\bigr)\ \le\ \dGT^{\bfP}(M,N).
\]
In particular, if $\bfP=\{1<\cdots<n\}$ with $d_{\bfP}(x,y)=|x-y|$ and $\top$ adjoined with $d_{\bfP}(\cdot,\top)=\infty$, this recovers the classical bottleneck stability inequality.
\end{theorem}

\begin{proof}
Apply the Stability Theorem with the base poset replaced by $\Int\bfP$ and the modules $K(M),K(N)$ to get the first inequality. The second inequality is Proposition~\ref{prop:K-Lipschitz}.
\end{proof}

%%%%%%%%%%%%%%%%%%%%%%%%%%%%%%%%%%%%%%%%%%%%%%%%%%%%%%%%%%%%%%%%
\subsection{Examples}
%%%%%%%%%%%%%%%%%%%%%%%%%%%%%%%%%%%%%%%%%%%%%%%%%%%%%%%%%%%%%%%%

We now compute persistence diagrams for our running 1D and 2D examples, and
verify stability at the level of minimal resolutions in $\vect^{\Int\bfP}$.

\begin{example}\label{ex:persistence-chain-correct}
Let $\bfP=\{1<2<3<4<\top\}$ with $d_{\bfP}(i,j)=|i-j|$ and
$d_{\bfP}(\cdot,\top)=\infty$.
Retain the modules
\[
M:=I[1,3)\oplus I[3,4),\qquad
N:=I[2,4)
\]
from Examples~\ref{ex:gtd-chain} and~\ref{ex:bneck-chain}.
We compute their kernel modules and minimal resolutions over $\Int\bfP$.

The kernel modules vanish at $[3,3]$ and $[4,4]$ for all three summands, and
$K(\cdot)$ is nonzero on the expected intervals extending upward or to~$\top$.

Let $\Yint{[x,y]}:=\Int\bfP([x,y],-)$.
The minimal generators of $K(M)$ occur at 
$\Yint{[1,3]}$, $\Yint{[2,3]}$, and $\Yint{[3,4]}$,
and the minimal relations occur at 
$\Yint{[3,3]}$ and $\Yint{[4,4]}$.
Thus a minimal resolution is
\[
0\longrightarrow
\Yint{[3,3]}\oplus \Yint{[4,4]}
\longrightarrow
\Yint{[1,3]}\oplus \Yint{[2,3]}\oplus \Yint{[3,4]}
\longrightarrow K(M)\longrightarrow 0.
\]

Similarly, $K(N)$ is minimally generated at $\Yint{[2,4]}$ with a single
relation at $\Yint{[4,4]}$, giving
\[
0\longrightarrow
\Yint{[4,4]}
\longrightarrow
\Yint{[2,4]}
\longrightarrow
K(N)\longrightarrow 0.
\]

The corresponding persistence diagrams are
\[
\mathrm{PD}(M):\ 
\begin{cases}
\text{degree }0:\ \Yint{[1,3]},\ \Yint{[2,3]},\ \Yint{[3,4]},\\
\text{degree }1:\ \Yint{[3,3]},\ \Yint{[4,4]},
\end{cases}
\qquad
\mathrm{PD}(N):\
\begin{cases}
\text{degree }0:\ \Yint{[2,4]},\\
\text{degree }1:\ \Yint{[4,4]}.
\end{cases}
\]

Cone padding (e.g.\ $\Cone(\id_{\Yint{[3,4]}})[1]$) implements the classical
“match to the diagonal’’ at zero cost.  
Since $\dGT^{\bfP}(M,N)=1$ (Example~\ref{ex:gtd-chain}) and 
$\dGT^{\Int\bfP}(K(M),K(N))\le 1$ (Proposition~\ref{prop:K-Lipschitz}),
the persistence stability theorem gives
\[
\dist_B(K\down^{M},K\down^{N})\le 1.
\]
\end{example}

\begin{table}[h]
\centering
\caption{Persistence diagrams for $M$ and $N$
over $\bfP=\{1<2<3<4<\top\}$.}
\label{tab:pd-1d}
\begin{tabular}{c|c|c}
 & $\mathrm{PD}(M)$ & $\mathrm{PD}(N)$ \\ \hline
degree $0$
 & $\Yint{[1,3]},\ \Yint{[2,3]},\ \Yint{[3,4]}$
 & $\Yint{[2,4]}$ \\[4pt]
degree $1$
 & $\Yint{[3,3]},\ \Yint{[4,4]}$
 & $\Yint{[4,4]}$
\end{tabular}
\end{table}


\begin{example}\label{ex:persistence-2d}
Let $\bfP=\{1,\dots,8\}^2$ with the product order and $L^\infty$ metric,
and retain $M^{(2)},N^{(2)}$ from Example~\ref{ex:gtd-2d}$.$  
Apply the kernel–interval functor $K:\vect^{\bfP}\to\vect^{\Int\bfP}$.

Each rectangle in $M^{(2)}$ contributes one degree-$0$ generator and two
degree-$1$ face relations.  
The L-shape contributes generators at $\Yint{[(1,4),(5,8)]}$ and
$\Yint{[(4,1),(8,5)]}$ and a single degree-$1$ corner relation at
$\Yint{[(4,4),(5,5)]}$.  
Thus $K\down^{M^{(2)}}$ and $K\down^{N^{(2)}}$ both have nonzero terms in
degrees $0$ and $1$.

After padding by contractible cones to equalize degreewise sizes, match the
degree-$0$ projectives using the $(+1,+1)$ shift:
\[
[(2,5),(4,7)]\longleftrightarrow [(1,4),(5,8)],\qquad
[(5,2),(7,4)]\longleftrightarrow [(4,1),(8,5)],
\]
each at cost $1$ in $d_{\Int\bfP}$.  
Match degree-$1$ relations by pairing the corner with its counterpart and the
remaining face relations with cone copies (cost $0$).  
Hence
\[
\dist_B(K\down^{M^{(2)}},K\down^{N^{(2)}})\le 1.
\]

By Proposition~\ref{prop:K-Lipschitz},
\[
\dGT^{\Int\bfP}(K(M^{(2)}),K(N^{(2)}))
\ \le\
\dGT^{\bfP}(M^{(2)},N^{(2)})=1,
\]
and so
\[
\dist_B(K\down^{M^{(2)}},K\down^{N^{(2)}})\le 1.
\]
\end{example}

\begin{table}[h]
\centering
\caption{Persistence diagrams for $M^{(2)}$ and $N^{(2)}$
in $\vect^{\Int\bfP}$ over $\bfP=\{1,\dots,8\}^2$.}
\label{tab:pd-2d}
\begin{tabular}{c|p{3.8cm}|p{3.8cm}}
 & $\mathrm{PD}(M^{(2)})$ & $\mathrm{PD}(N^{(2)})$ \\ \hline
degree $0$
 & $\Yint{[(2,5),(4,7)]}$\\[-3pt]
   $\Yint{[(5,2),(7,4)]}$
 & $\Yint{[(1,4),(5,8)]}$\\[-3pt]
   $\Yint{[(4,1),(8,5)]}$
 \\[6pt]
degree $1$
 & $\Yint{[(4,5),(5,7)]},\ \Yint{[(2,7),(4,8)]}$\\[-3pt]
   $\Yint{[(5,4),(7,5)]},\ \Yint{[(7,2),(8,4)]}$
 & $\Yint{[(4,4),(5,5)]}$
\end{tabular}
\end{table}


%%%%%%%%%%%%%%%%%%%%%%%%%%%%%%%%%%%%%%%%%%%%%%%%%%%%%%%%%%%%%%%%%%%
% References
%%%%%%%%%%%%%%%%%%%%%%%%%%%%%%%%%%%%%%%%%%%%%%%%%%

\bibliographystyle{alpha}
\bibliography{references}{}



\appendix

%%%%%%%%%%%%%%%%%%%%%%%%%%%%%%%%%%%%%%%%%%%%%%%%%%%%%%%%%%%%
%%%%%%%%%%%%%%%%%%%%%%%%%%%%%%%%%%%%%%%%%%%%%%%%%%%%%%%%%%%%
\section{Interleaving Distance on \texorpdfstring{$\mathbb{R}$}{R}}
\label{app:interleaving}
%%%%%%%%%%%%%%%%%%%%%%%%%%%%%%%%%%%%%%%%%%%%%%%%%%%%%%%%%%%%
%%%%%%%%%%%%%%%%%%%%%%%%%%%%%%%%%%%%%%%%%%%%%%%%%%%%%%%%%%%%

In the body of the paper we work with modules over a finite poset. Classical persistence uses the totally ordered real line
\[
\bfR=(\mathbb{R},\le),
\quad
d_{\bfR}(x,y)=|x-y|.
\]
Here we recall the interleaving distance on $\vect^{\bfR}$ and show it agrees with the Galois transport distance defined earlier (now interpreted over $\bfR$).


For $\varepsilon\ge 0$ let $T_\varepsilon:\vect^{\bfR}\to\vect^{\bfR}$ be the shift functor
\[
(T_\varepsilon M)(r):=M(r+\varepsilon),
\qquad
(T_\varepsilon\alpha)_r:=\alpha_{r+\varepsilon}.
\]

\begin{definition}\label{def:interleaving-distance}
An \emph{$\varepsilon$-interleaving} between $M,N\in \vect^{\bfR}$ is a pair of natural transformations
\[
\varphi:M\Rightarrow T_\varepsilon N,
\qquad
\psi:N\Rightarrow T_\varepsilon M
\]
such that the composites $M\xrightarrow{\ \varphi\ } T_\varepsilon N \xrightarrow{T_\varepsilon\psi} T_{2\varepsilon} M$ and
$N\xrightarrow{\ \psi\ } T_\varepsilon M \xrightarrow{T_\varepsilon\varphi} T_{2\varepsilon} N$
equal the canonical structure maps induced by $r\le r+2\varepsilon$.
The \emph{interleaving distance} is
\[
d_I(M,N):=\inf\{\ \varepsilon\ge 0\mid M,N\ \text{are $\varepsilon$-interleaved}\ \}.
\]
\end{definition}


\begin{proposition}\label{prop:dI-pseudometric}
For $M,N,O \in \vect^{\bfR}$,
\[
d_I(M,N)=d_I(N,M),\qquad
d_I(M,O)\le d_I(M,N)+d_I(N,O),\qquad
d_I(M,M)=0.
\]
Moreover, $d_I(M,N)=0$ iff $M\cong N$; hence $d_I$ is an extended metric on isomorphism classes.
\end{proposition}

\begin{proof}[Proof sketch]
Symmetry is clear by swapping $(\varphi,\psi)$. Triangle inequality is obtained by pasting interleavings. The case $d_I(M,M)=0$ is immediate; if $d_I(M,N)=0$, a $0$-interleaving gives mutually inverse isomorphisms.
\end{proof}


For $\varepsilon\ge 0$, the translations
\[
t_\varepsilon:\bfR\to\bfR,\quad t_\varepsilon(r)=r+\varepsilon,
\qquad
r_\varepsilon:\bfR\to\bfR,\quad r_\varepsilon(r)=r-\varepsilon
\]
satisfy \(t_\varepsilon\dashv r_\varepsilon\); in our convention this is a Galois insertion with left adjoint \(t_\varepsilon:\bfR\to\bfR\) and right adjoint \(r_\varepsilon:\bfR\to\bfR\).

\begin{proposition}\label{prop:int-to-galois-R}
If $M,N\in \vect^{\bfR}$ are $\varepsilon$-interleaved, there exists a
Galois coupling $(\bfQ,f\dashv g,\ h\dashv i,\ \Gamma)$ for $(M,N)$ with
\[
\bfQ:=\bfR_L\sqcup \bfR_R,
\quad
f|_{\bfR_L}=\mathrm{id},\ f|_{\bfR_R}=t_\varepsilon,
\quad
h|_{\bfR_L}=t_\varepsilon,\ h|_{\bfR_R}=\mathrm{id},
\]
and \(\cost(\Gamma)=\varepsilon\).
\end{proposition}

\begin{proof}
Order the disjoint union $\bfR_L\sqcup\bfR_R$ so that cross-inequalities encode the shift (e.g.\ $a_L\le b_R\iff a+\varepsilon\le b$ and symmetrically). Define $\Gamma(a_L):=M(a)$ and $\Gamma(a_R):=N(a)$, with structure maps on cross arrows induced by the interleaving data. Then $g^{*}\Gamma\cong M$, $i^{*}\Gamma\cong N$, and $\sup_{x\in\bfQ}|f(x)-h(x)|=\varepsilon$.
\end{proof}

\begin{proposition}\label{prop:coupling-implies-int-R}
Let $(\bfQ,f\dashv g,\ h\dashv i,\ \Gamma)$ be a Galois coupling for
$(M,N)\in \vect^{\bfR}$ with \(\sup_{x\in \bfQ}|f(x)-h(x)|\le\varepsilon\).
Then $M$ and $N$ are $\varepsilon$-interleaved.
\end{proposition}

\begin{proof}[Idea]
For $r\in\mathbb{R}$ set $x=g(r)$. Then $f(x)\le r$ and $h(x)\le r+\varepsilon$. Using the unit/counit of $h\dashv i$ and $f\dashv g$, define
\[
\varphi_r: M(r)\xrightarrow{\ \cong\ }\Gamma(x)\to \Gamma(i(h(x)))\xrightarrow{\ \cong\ }N(h(x))\to N(r+\varepsilon),
\]
and symmetrically \(\psi_r:N(r)\to M(r+\varepsilon)\). Naturality and the interleaving identities follow from the adjunction triangle identities and the cost bound.
\end{proof}


\begin{theorem}\label{thm:GT-equals-dI-R}
For all $M,N\in \vect^{\bfR}$, the Galois transport distance equals
the interleaving distance:
\[
\dGT^{\bfR}(M,N)=d_I(M,N).
\]
\end{theorem}

\begin{proof}
By Proposition~\ref{prop:int-to-galois-R}, any $\varepsilon$-interleaving yields a coupling of cost $\varepsilon$, so \(\dGT^{\bfR}(M,N)\le d_I(M,N)\).
Conversely, by Proposition~\ref{prop:coupling-implies-int-R}, any coupling of cost $\varepsilon$ yields an $\varepsilon$-interleaving, so \(d_I(M,N)\le \dGT^{\bfR}(M,N)\).
\end{proof}











\end{document}