\documentclass[12pt,letterpaper]{article}
\usepackage{amitstyle}

\DeclareMathOperator{\cost}{cost}
\DeclareMathOperator{\dist}{dist}        % the other metric used in the paper
\newcommand{\dGT}[2]{\dist_{\mathrm{GT}}(#1,#2)} % Galois-transport metric

\newcommand{\vect}{\mathsf{vec}}         % category of finite-dimensional vector spaces
\newcommand{\Pos}{\mathbf{Pos}}          % category of posets

\DeclareMathOperator{\Match}{Match}      % set of matchings (as defined in text)
\DeclareMathOperator{\Bij}{Bij}          % set of bijections
\DeclareMathOperator{\Cone}{Cone}        % cone construction
\DeclareMathOperator{\Summands}{Summands}% set of direct summands

\newcommand{\fU}{\mathfrak{U}}           % fixed universe for the 2-category
\newcommand{\GT}{\mathrm{GT}}            % (optional) label for the name in prose

\newcommand{\todoilgreen}[1]{\textcolor{green}{TODO: #1}}
\newcommand{\frkU}{\mathfrak{U}}

\newcommand\Nname[1]{|[alias=#1]|}

\newcommand\al{\alpha}

\newcommand\be{\beta}

\newcommand\ep{\varepsilon}

\newcommand\Ga{\Gamma}

\newcommand\prj{\operatorname{prj}}

\newcommand\calC{{\mathcal C}}

\newcommand\calD{{\mathcal D}}

\newcommand\calP{{\mathcal P}}

\newcommand\bbN{\mathbb{N}}

\newcommand\bbR{\mathbb{R}}

\newcommand\iso{\cong}

\newcommand\ds{\oplus}

\newcommand\Ds{\bigoplus}

\newcommand\id{1\kern-.25em{\text{{\rm l}}}} 

\newcommand\isoto{\ \raise.8ex\hbox{$^{\sim}$}\kern-.7em\hbox{$\to$}\ }

\newcommand\down{_{\cdot}}

\newcommand\blank{\operatorname{-}}



\newcommand\bfP{\mathbf{P}}

\newcommand\bfQ{\mathbf{Q}}

\newcommand\bfX{\mathbf{X}}

\newcommand\bfY{\mathbf{Y}}

\newcommand\bfZ{\mathbf{Z}}

\newcommand\sfU{\mathsf{U}}

\newcommand\Fun{\operatorname{Fun}}

\newcommand\Cat{\mathbf{Cat}}

% Representable at x in P
\newcommand{\Yp}[1]{\bfP(#1,-)}
\DeclareMathOperator{\Hom}{Hom}






\begin{document}

\title{Title TBD}
\author{Hideto Asashiba\thanks{Supported by JSPS Grant-in-Aid for Scientific Research (C) 18K03207; 
JSPS Grant-in-Aid for Transformative Research Areas (A) (22A201); 
Osaka Central Advanced Mathematical Institute (MEXT Promotion of Distinctive Joint Research Center Program JPMXP0723833165).}
\and Amit K. Patel}


\maketitle

\begin{abstract}

\end{abstract}

\noindent\textbf{2020 Mathematics Subject Classification.} 16G20, 16G70, 55N31, 62R40.

\medskip

\noindent\textbf{Affiliations.}
Department of Mathematics, Faculty of Science, Shizuoka University,
836 Ohya, Suruga-ku, Shizuoka, 422-8529, Japan; Institute for Advanced Study, KUIAS, Kyoto University,
Yoshida Ushinomiya-cho, Sakyo-ku, Kyoto 606-8501, Japan; and
Osaka Central Advanced Mathematical Institute, 3-3-138 Sugimoto, Sumiyoshi-ku, Osaka, 558-8585, Japan. \par
Department of Mathematics, Colorado State University, Fort Collins, CO 80523, USA.

\medskip

\noindent\textbf{Emails.}
\href{mailto:asashiba.hideto@shizuoka.ac.jp}{asashiba.hideto@shizuoka.ac.jp}
\quad|\quad
\href{mailto:Amit.Patel@colostate.edu}{Amit.Patel@colostate.edu}


\tableofcontents

\section{Introduction}

%%%%%%%%%%%%%%

\subsection{Purposes}

%%%%%%%%%%%%%%%%%%%

\subsection{Our contributions}


\subsection{Organization}


\section{Preliminaries}
\label{sec:prelim}

We start with some categorical facts that will be used throughout.

\begin{lemma}
\label{lem:pres-proj}
Let $L:\calC\to\calD$ be a functor between abelian categories.
If $L$ has an exact right adjoint $R:\calD\to\calC$,
then $L$ sends projectives to projectives.
\end{lemma}

\begin{proof}
Let $A$ be projective in $\calC$. Then the functor $\calC(A,-)$ is exact, hence so is
\[
\calC(A,-)\circ R=\calC(A,R(-))\cong \calD(L(A),-).
\]
Thus $L(A)$ is projective.
\end{proof}

Let $\frkU$ be a universe. An element of $\frkU$ is called a $\frkU$-\emph{small} set. A category $\calC$ is $\frkU$-\emph{small} if its object set is $\frkU$-small and, for any objects $X,Y$, each hom-set $\calC(X,Y)$ is $\frkU$-small. We denote by $\Cat^\frkU$ the $2$-category whose objects are the $\frkU$-small categories, whose $1$-morphisms are functors, and whose $2$-morphisms are natural transformations.

\begin{proposition}
\label{prp:Yoneda-2-functor}
If $\calC$ is an object of $\Cat^\frkU$, then
\[
\Cat^\frkU(\mbox{-},\calC):(\Cat^\frkU)^{\mathrm{op}}\longrightarrow\Cat^\frkU
\]
is a $2$-functor.
\end{proposition}

\begin{proof}
See Proposition~4.5.4 in \cite{JY}.
\end{proof}

\begin{remark}
\label{rmk:contravariant-adj}
In particular, the $2$-functor $\Cat^\frkU(\mbox{-},\calC)$ sends an adjoint system $(f,g,\eta,\varepsilon)$ to the adjoint system $(g^{*},f^{*},\eta^{*},\varepsilon^{*})$ in a contravariant manner. Concretely, if $f\dashv g$ between categories $\calA$ and $\calB$, then precomposition yields $g^{*}\dashv f^{*}$ between the functor categories $\Cat^\frkU(\calB,\calC)$ and $\Cat^\frkU(\calA,\calC)$. We use this repeatedly with $\calC=\vect$ below.
\end{remark}

%%%%%%%%%%%%%%%%%%%%%%%%%%%%%%%%%%%%%%%%%%%%%%%%%%%%%%%%%%%%%%%%%%%
\subsection{Poset Modules}
\label{ssec:poset-mod}
%%%%%%%%%%%%%%%%%%%%%%%%%%%%%%%%%%%%%%%%%%%%%%%%%%%%%%%%%%%%%%%%%%%

Throughout, $\Bbbk$ is a field, and $\bfP=(\bfP,\le)$ is a poset. We regard $\bfP$ as a small (thin) category: there is a unique morphism $x\to y$ exactly when $x\le y$. This lets us identify $\bfP$-modules with functors $\bfP\to\vect$ and, equivalently, with left modules over the $\Bbbk$-linearization $\Bbbk\bfP$ (same objects, order relations as basis arrows, composition by transitivity). We assume $\bfP$ is finite except in Sect.\ \ref{sec:intlv-dist}.

Let $\vect$ denote the category of finite-dimensional $\Bbbk$-vector spaces, and write $\vect^\bfP=\Fun(\bfP,\vect)$ for the category of $\bfP$-modules. If $\bfP$ is finite, then $\vect^\bfP$ is a $\Bbbk$-linear, abelian, Krull–Schmidt category, and for each $M,N\in\vect^\bfP$ the $\Bbbk$-space $\vect^\bfP(M,N)$ is finite dimensional.

We write $\prj\bfP$ for the full subcategory of projective objects in $\vect^\bfP$.
For $x\in\bfP$, the \emph{representable} at $x$ is the functor
\[
\bfP(x,-)\in\vect^\bfP.
\]
It sends $y\in\bfP$ to $\bfP(x,-)(y)=\Bbbk$ if $x\le y$ and to $0$ otherwise; for a relation $y\le z$ it acts by the identity on $\Bbbk$ whenever $x\le y$ (and by $0$ otherwise). By the Yoneda lemma there is a natural isomorphism
\[
\Hom_{\vect^\bfP}(\bfP(x,-),M)\cong M(x)\qquad(M\in\vect^\bfP),
\]
showing that each $\bfP(x,-)$ is projective.

\begin{lemma}
\label{lem:ind-proj}
The family $\{\bfP(x,-)\mid x\in\bfP\}$ is a complete set of representatives of the isomorphism classes of indecomposable projective $\bfP$-modules. Hence each $M\in\prj \bfP$ decomposes uniquely as
\[
M \cong \bigoplus_{x\in\bfP} \bfP(x,-)^{\,a_x}
\]
for a uniquely determined family $(a_x)_{x\in\bfP}\in\bbN^\bfP$.
\end{lemma}

%%%%%%%%%%%%%%%%%%%%%%%%%%%%%%%%%%%%%%%%%%%%%%%%%%%%%%%%%%%%%%%%%%%
\subsection{Monotone Functions and Kan Extensions}
\label{ssec:mont-funct}
%%%%%%%%%%%%%%%%%%%%%%%%%%%%%%%%%%%%%%%%%%%%%%%%%%%%%%%%%%%%%%%%%%%

A map $f:\bfP\to\bfQ$ of posets is \emph{monotone} if $x\le y$ in $\bfP$ implies $f(x)\le f(y)$ in $\bfQ$. Viewing posets as thin categories, a functor $\bfP\to\bfQ$ is precisely a monotone map. Thus a monotone map $f:\bfP\to\bfQ$ induces the restriction (precomposition) functor
\[
f^{*}:\vect^\bfQ\to\vect^\bfP,\qquad f^{*}N = N\circ f .
\]

\begin{proposition}
\label{prop:Kan-posets}
For any monotone $f:\bfP\to\bfQ$, the functor $f^{*}$ is exact. Moreover, the left and right Kan extensions along $f$ exist and we denote them by
\[
f_{!}:=\Lan_{f}\qquad\text{and}\qquad f_{*}:=\Ran_{f}.
\]
By the defining property of conextensions there are natural adjunctions
\[
f_{!}\dashv f^{*}\dashv f_{*}.
\]
In particular, $f_{!}$ is right exact, $f_{*}$ is left exact, and $f^{*}$ is exact.
\end{proposition}

\[
\begin{tikzcd}[row sep=3em, column sep=6em]
\Nname{E}\vect^\bfP & \Nname{C}\vect^\bfQ
\arrow[from={E}, to={C}, "f_{!}", ""'{name=Fu}, bend left]
\arrow[from={C}, to={E},"f^{*}"{description, pos=.5, name=Fc}]
\arrow[from={E}, to={C}, "f_{*}"', ""'{name=Fd}, bend right]
\arrow[from={Fu}, to={Fc}, "\rotatebox{-90}{$\dashv$}" description, phantom]
\arrow[from={Fc}, to={Fd}, "\rotatebox{-90}{$\dashv$}" description, phantom]
\end{tikzcd}
\]

%%%%%%%%%%%%%%%%%%%%%%%%%%%%%%%%%%%%%%%%%%%%%%%%%%%%%%%%%%%%%%%%%%%
\subsection{Galois Connections}
%%%%%%%%%%%%%%%%%%%%%%%%%%%%%%%%%%%%%%%%%%%%%%%%%%%%%%%%%%%%%%%%%%%

We recall that Galois connections are adjunctions between posets and underlie the identifications among $f_{!}, f^{*}, f_{*}$ used later.

\begin{definition}
A \emph{Galois connection} between posets $\bfQ,\bfP$ consists of monotone maps $f:\bfQ\to\bfP$ and $g:\bfP\to\bfQ$ such that
\[
f(u)\le x \iff u\le g(x)\qquad(u\in \bfQ,\ x\in \bfP).
\]
Equivalently, viewing $\bfQ$ and $\bfP$ as thin categories, this means $f\dashv g$ as functors. We write $f:\bfQ\rightleftarrows \bfP:g$, with $f$ left adjoint and $g$ right adjoint.
\end{definition}

\begin{corollary}
If $f:\bfQ\rightleftarrows \bfP:g$ is a Galois connection, then $g^{*}\dashv f^{*}$.
\[
\begin{tikzcd}[row sep=3em]
\Nname{E}\vect^\bfP & \Nname{C} \vect^\bfQ 
\arrow[from={C}, to={E}, "g^{*}", ""'{name=Gs}, bend left]
\arrow[from={E}, to={C}, "f^{*}"{name=Fs}, bend left]
\arrow[from={Gs}, to={Fs}, "\rotatebox{-90}{$\dashv$}" description, phantom]
\end{tikzcd}.
\]
\end{corollary}

\begin{proof}
Immediate from the contravariant $2$-functoriality of $\Cat^\frkU(\mbox{-},\vect)$ (e.g.\ Proposition~4.5.4 in \cite{JY}).
\end{proof}

\begin{corollary}
\label{cor:3_adj}
In the same setting, there are natural isomorphisms $g^{*}\cong f_{!}$ and $f^{*}\cong g_{*}$. In particular, the adjoint pairs
\[
g^{*}\dashv f^{*},\qquad f_{!}\dashv f^{*},\qquad g^{*}\dashv g_{*}
\]
are compatible via these isomorphisms.
\[
\begin{tikzcd}[row sep=50pt, column sep = 70pt]
\Nname{E}\vect^\bfP & \Nname{C} \vect^\bfQ 
\arrow[from={C}, to={E}, "g^{*} \cong f_{!}"{description, pos=.5, name=gs}, bend left=15pt]
\arrow[from={E}, to={C}, "f^{*}\cong g_{*}"{description, pos=.5, name=fs}, bend left=15pt]
\arrow[from={E}, to={C}, "g_{!}"', ""{name=gL}, bend right=80pt]
\arrow[from={C}, to={E}, "f_{*}"', ""{name=fR}, bend right=80pt]
\arrow[from={gs}, to={fs}, "\rotatebox{-90}{$\dashv$}" description, phantom]
\arrow[from={gL}, to={gs}, "\rotatebox{-90}{$\dashv$}" description, phantom]
\arrow[from={fs}, to={fR}, "\rotatebox{-90}{$\dashv$}" description, phantom]
\end{tikzcd}.
\]
\end{corollary}

\begin{proof}
By uniqueness of adjoints in a $2$-category: $g^{*}$ and $f_{!}$ are both left adjoint to $f^{*}$, hence canonically isomorphic; dually for $f^{*}$ and $g_{*}$.
\end{proof}

\begin{corollary}
\label{cor:g-ast-ex}
If $f:\bfQ\rightleftarrows \bfP:g$ is a Galois connection, then $g_{*}:\vect^\bfP\to\vect^\bfQ$ is exact.
\end{corollary}

\begin{proof}
Using Corollary~\ref{cor:3_adj}, $g_{*}\cong f^{*}$, and $f^{*}$ is exact by Proposition~\ref{prop:Kan-posets}.
\end{proof}

Combining Corollary~\ref{cor:g-ast-ex} with Lemma~\ref{lem:pres-proj} yields:

\begin{proposition}\label{prop:pullback_preserves_projectives}
Let $f:\bfQ\rightleftarrows \bfP:g$ be a Galois connection. If $M\in\vect^\bfQ$ is projective, then $g^{*}(M)$ is projective in $\vect^\bfP$.\qed
\end{proposition}

\medskip

The next basic facts will be used tacitly; they are immediate from the adjunction $f\dashv g$ and we omit the proof.

\begin{lemma}\label{lem:galois-connection-basics}
Let $f:\bfQ\rightleftarrows \bfP:g$ be a Galois connection. Then:
\begin{enumerate}
\item The following are equivalent: $f$ is surjective; $g$ is injective; $f\circ g=\id_{\bfP}$.
\item For $x\in \bfP$,
\[
g(x)=\max\{\,u\in \bfQ\mid f(u)\le x\,\}.
\]
In particular, if $f\circ g=\id_{\bfP}$, then
\[
g(x)=\max\{\,u\in \bfQ\mid f(u)=x\,\}.
\]
\end{enumerate}
\end{lemma}



%%%%%%%%%%%%%%%%%%%%%%%%%%%%%%%%%%%%%%%%%%%%%%%%%%%%%%%%%%%%%%%%%
%%%%%%%%%%%%%%%%%%%%%%%%%%%%%%%%%%%%%%%%%%%%%%%%%%%%%%%%%%%%%%%%%
\section{Galois Transport Distance}
\label{sec:galois-transport}
%%%%%%%%%%%%%%%%%%%%%%%%%%%%%%%%%%%%%%%%%%%%%%%%%%%%%%%%%%%%%%%%%
%%%%%%%%%%%%%%%%%%%%%%%%%%%%%%%%%%%%%%%%%%%%%%%%%%%%%%%%%%%%%%%%%

Fix a finite poset $\bfP$ equipped with a metric $d_{\bfP}$. Motivated by optimal transport, we compare $\bfP$-modules by \emph{transporting} them through a common “apex” poset $\bfQ$ via \emph{Galois insertions}. 

\begin{definition}
\label{def:coupling-insertion}
Let $M,N\in\vect^{\bfP}$. A \emph{Galois coupling} of $(M,N, f \dashv g, h \dashv i, \Gamma)$ consists of a poset $\bfQ$, two Galois insertions
\[
f:\bfQ\rightleftarrows \bfP:g,
\qquad
h:\bfQ\rightleftarrows \bfP:i
\quad\text{with}\quad
f\circ g=\id_{\bfP}=h\circ i,
\]
and a module $\Gamma\in\vect^{\bfQ}$ such that $g^{*}\Gamma \cong M$ and $i^{*}\Gamma \cong N$. Equivalently (Corollary~\ref{cor:3_adj}), $M\cong f_{!}\Gamma$ and $N\cong h_{!}\Gamma$.
\end{definition}

\[
\begin{tikzcd}
& \bfQ \ar[dd,"\Gamma"] \ar[dr, bend left, "h"{name=H}] \ar[dl, bend right, "f"'{name=F}] & \\
\bfP \ar[rd,"M"'] \ar[ru, bend right, "g"'{name=G}] && \bfP \ar[ld,"N"] \ar[lu, bend left, "i"{name=I}] \\
& \vect &
% adjunction markers:
\arrow[phantom, from=F, to=G, "\rotatebox{-45}{$\dashv$}" description]
\arrow[phantom, from=H, to=I, "\rotatebox{-135}{$\dashv$}" description]
\end{tikzcd}
\]

\begin{definition}
\label{def:cost}
The \emph{cost} of a coupling $(\bfQ,f\dashv g,\ h\dashv i,\ \Gamma)$ is
\[
\cost(\Gamma):=\sup_{q\in\bfQ} d_{\bfP}\big(f(q),\,h(q)\big).
\]
\end{definition}

%%%%%%%%%%%%%%%%%%%%%%%%%%%%%%%%%%%%%%%%%%%%%%%%%%%%%%%%%%%%%%%%%
\subsection{Composition of Couplings}
\label{ssec:composition}
%%%%%%%%%%%%%%%%%%%%%%%%%%%%%%%%%%%%%%%%%%%%%%%%%%%%%%%%%%%%%%%%%

Now consider two Galois couplings $(M,N, f_1 \dashv g_1, h_1 \dashv i_1, \Gamma_1)$ and
$(N,O, f_2 \dashv g_2, h_2 \dashv i_2, \Gamma_2)$ below:

\[
\begin{tikzcd}
    && \bfR \ar[dr, bend left, "\pi_2"{name=piTwo}] \ar[dl, bend right, "\pi_1"'{name=piOne}]  && \\
    & \bfQ_1 \ar[dddr, "\Gamma_1", near end, bend right = 20] \ar[ur, bend right, "\iota_1"'{name=iotaOne}]  
        \ar[dl, bend right, "f_1"'{name=fOne}] \ar[dr, bend left, "h_1"{name=hOne}] & & 
      \bfQ_2 \ar[ddld, "\Gamma_2"', bend left = 20, near end] \ar[ul, bend left, "\iota_2"{name=iotaTwo}] 
        \ar[dl, bend right, "f_2"'{name=fTwo}] \ar[dr, bend left, "h_2"{name=hTwo}] & \\
    \bfP \ar[rrdd, "M"] \ar[ur, bend right, "g_1"'{name=gOne}] && 
    \bfP \ar[dd, "N"]  \ar[ul, bend left, "i_1"{name=iOne}] 
      \ar[ur, bend right, "g_2"'{name=gTwo}] && 
    \bfP \ar[ldld, "O"] \ar[ul, bend left, "i_2"{name=iTwo}] \\
    &  & &  & \\
    && \vect &&
    % adjunction markers:
    \arrow[phantom, from=piOne, to=iotaOne, "\rotatebox{-45}{$\dashv$}" description]
    \arrow[phantom, from=piTwo, to=iotaTwo, "\rotatebox{-135}{$\dashv$}" description]
    \arrow[phantom, from=fOne,  to=gOne,   "\rotatebox{-45}{$\dashv$}" description]
    \arrow[phantom, from=hOne,  to=iOne,   "\rotatebox{-135}{$\dashv$}" description]
    \arrow[phantom, from=fTwo,  to=gTwo,   "\rotatebox{-34}{$\dashv$}" description]
    \arrow[phantom, from=hTwo,  to=iTwo,   "\rotatebox{-135}{$\dashv$}" description]
\end{tikzcd}
\]
\vspace{-0.5em}

Let $\bfR$ be the pullback of $(h_1,f_2)$ in the category of posets (i.e.\ thin categories), with projections $\pi_1:\bfR\to\bfQ_1$ and $\pi_2:\bfR\to\bfQ_2$. Set
\[
\delta:=h_1\circ\pi_1=f_2\circ\pi_2:\bfR\to\bfP.
\]
Since $h_1$ and $f_2$ are the left adjoints in Galois insertions, $f_2\circ g_2=\id_{\bfP}$ and $h_1\circ i_1=\id_{\bfP}$. By the universal property of the pullback there are unique monotone maps
\[
\iota_1:\bfQ_1\to\bfR\quad\text{and}\quad \iota_2:\bfQ_2\to\bfR
\]
with $\pi_1\circ\iota_1=\id_{\bfQ_1}$ and $\pi_2\circ\iota_2=\id_{\bfQ_2}$. In particular, $\pi_j\dashv\iota_j$ is a Galois insertion (Lemma~\ref{lem:galois-connection-basics}(1)–(2)).

\begin{proposition}
\label{prop:compose-couplings}
With notation as above, there exists $\Psi\in\vect^{\bfR}$ and natural isomorphisms
\[
(\iota_1\circ g_1)^{*}\Psi \cong M,
\qquad
(\iota_2\circ i_2)^{*}\Psi \cong O,
\]
so that
\[
\big(M,O,\ f_1\circ\pi_1 \dashv \iota_1\circ g_1,\ \ h_2\circ\pi_2 \dashv \iota_2\circ i_2,\ \ \Psi\big)
\]
is a Galois coupling (the \emph{composite} of the two displayed couplings).
Moreover, this composite is unique up to unique isomorphism of couplings.
\end{proposition}

\begin{proof}
Consider the canonical isomorphisms in $\vect^{\bfP}$ coming from the two couplings,
\[
i_1^{*}\Gamma_1 \ \cong\ N \ \cong\ g_2^{*}\Gamma_2,
\]
and let $\varphi:i_1^{*}\Gamma_1\xrightarrow{\cong} g_2^{*}\Gamma_2$ be their composite. Whisker by $\delta^{*}$ to obtain an isomorphism
\[
\delta^{*}i_1^{*}\Gamma_1 \xrightarrow{\cong} \delta^{*}g_2^{*}\Gamma_2\quad\text{in }\vect^{\bfR}.
\]
The units of the insertions $\id\Rightarrow i_1h_1$ and $\id\Rightarrow g_2f_2$, transported by precomposition, give natural transformations
\[
\sigma_1:\ \pi_1^{*}\Gamma_1 \longrightarrow \delta^{*}i_1^{*}\Gamma_1,
\qquad
\sigma_2:\ \pi_2^{*}\Gamma_2 \longrightarrow \delta^{*}g_2^{*}\Gamma_2.
\]
Define $\Psi$ as the \emph{pullback} in the functor category $\vect^{\bfR}$:
\[
\Psi\ :=\ \mathrm{pullback}\left(
\pi_1^{*}\Gamma_1 \xrightarrow{\ \sigma_1\ } \delta^{*}i_1^{*}\Gamma_1
\ \xleftarrow[\ \cong\ ]{\ \delta^{*}\varphi\ }\ 
\delta^{*}g_2^{*}\Gamma_2 \xleftarrow{\ \sigma_2\ } \pi_2^{*}\Gamma_2
\right).
\]
Functor categories into $\vect$ admit limits computed pointwise, so this pullback exists canonically; the universal property immediately yields
$\iota_1^{*}\Psi\cong\Gamma_1$ and $\iota_2^{*}\Psi\cong\Gamma_2$, hence the endpoint isomorphisms. Uniqueness up to unique isomorphism follows from the universal property of the pullback.

By construction we have $\iota_1^{*}\Psi\cong\Gamma_1$ and $\iota_2^{*}\Psi\cong\Gamma_2$. 
Since precomposition is a (contravariant) $2$-functor, it is strictly functorial on composites:
\[
(\iota_1\circ g_1)^{*}=g_1^{*}\circ \iota_1^{*},
\qquad
(\iota_2\circ i_2)^{*}=i_2^{*}\circ \iota_2^{*}.
\]
Therefore
\[
(\iota_1\circ g_1)^{*}\Psi
= g_1^{*}(\iota_1^{*}\Psi)
\cong g_1^{*}\Gamma_1
\cong M,
\qquad
(\iota_2\circ i_2)^{*}\Psi
= i_2^{*}(\iota_2^{*}\Psi)
\cong i_2^{*}\Gamma_2
\cong O,
\]
where the last isomorphisms are those in the defining data of the two input couplings.
These composites are natural isomorphisms, being composites of natural isomorphisms.
\end{proof}

%%%%%%%%%%%%%%%%%%%%%%%%%%%%%%%%%%%%%%%%%%%%%%%%%%%%%%%%%%%%%%%%%
\subsection{Transport Distance}
\label{ssec:Galois_metric}
%%%%%%%%%%%%%%%%%%%%%%%%%%%%%%%%%%%%%%%%%%%%%%%%%%%%%%%%%%%%%%%%%

With composition available, we now define the transport distance and record its basic properties.

\begin{definition}
\label{def:transport-distance}
The \emph{Galois transport distance} between $M,N\in\vect^{\bfP}$ is
\[
\dGT(M,N):=\inf\{\ \cost(\Gamma)\ \mid\ \Gamma\ \text{is a Galois coupling of }(M,N)\ \}.
\]
If there is no Galois coupling between $M$ and $N$, set $\dGT(M,N)=\infty$.
\end{definition}

\begin{lemma}
\label{lem:cost-subadditive}
If $\Gamma_1$ is a coupling for $(M,N)$ and $\Gamma_2$ is a coupling for $(N,O)$, and $\Psi$ is their composite from Proposition~\ref{prop:compose-couplings}, then
\[
\cost(\Psi)\ \le\ \cost(\Gamma_1)\ +\ \cost(\Gamma_2).
\]
\end{lemma}

\begin{proof}
For $r\in\bfR$,
\[
\begin{aligned}
d_{\bfP}\big((f_1\pi_1)(r),\ (h_2\pi_2)(r)\big)
&\le d_{\bfP}\big((f_1\pi_1)(r),\ (h_1\pi_1)(r)\big) \\
&\quad + \underbrace{d_{\bfP}\big((h_1\pi_1)(r),\ (f_2\pi_2)(r)\big)}_{=\,0} \\
&\quad + d_{\bfP}\big((f_2\pi_2)(r),\ (h_2\pi_2)(r)\big),
\end{aligned}
\]
since $\delta=h_1\pi_1=f_2\pi_2$. Taking suprema and observing
\[
\sup_{r\in\bfR}d_{\bfP}\big(f_1\pi_1(r),h_1\pi_1(r)\big)=\sup_{q_1\in\bfQ_1}d_{\bfP}\big(f_1(q_1),h_1(q_1)\big)=\cost(\Gamma_1),
\]
(and similarly for $\Gamma_2$) yields the claim.
\end{proof}

\begin{proposition}
\label{prop:gt-pseudometric}
For a finite poset $\bfP$ with metric $d_{\bfP}$, the function
\[
\dGT:\ \mathrm{Ob}(\vect^{\bfP})\times \mathrm{Ob}(\vect^{\bfP})\longrightarrow [0,+\infty]
\]
is an extended pseudometric.
\end{proposition}

\begin{proof}
Nonnegativity is immediate. For any $M$, the identity coupling $\bfQ=\bfP$, $f=h=\id_{\bfP}$, $g=i=\id_{\bfP}$, $\Gamma=M$ has cost $0$, so $\dGT(M,M)=0$. Symmetry holds because swapping the two insertion legs $(f\dashv g,h\dashv i)$ of any coupling gives a coupling for $(N,M)$ with the same cost (the metric $d_{\bfP}$ is symmetric). The triangle inequality follows from Lemma~\ref{lem:cost-subadditive}.
\end{proof}

\begin{corollary}
\label{cor:gt-metric-on-iso}
On isomorphism classes, $\dGT$ is an extended metric: if $\dGT(M, N)=0$ then $M\cong N$.
\end{corollary}

\begin{proof}
Since $\bfP$ is finite, the set $\{d_{\bfP}(x,y)\mid x,y\in\bfP\}$ is finite; hence every coupling has cost in this finite set, and the infimum in the definition of $\dGT(M,N)$ is a \emph{minimum}. If $\dGT(M,N)=0$, there exists a coupling with $\cost(\Gamma)=0$, so $d_{\bfP}(f(q),h(q))=0$ for all $q\in\bfQ$, hence $f(q)=h(q)$ and therefore $f=h$ as maps $\bfQ\to\bfP$. Using Corollary~\ref{cor:3_adj},
\[
M\ \cong\ f_{!}\Gamma\ =\ h_{!}\Gamma\ \cong\ N,
\]
so $M\cong N$.
\end{proof}

\begin{example}
Relate the interleaving distance to the Galois transport distance for finite totally ordered posets.
\end{example}
%%%%%%%%%%%%%%%%%%%%%%%%%%%%%%%%%%%%%%%%%%%%%%%%%%%
%%%%%%%%%%%%%%%%%%%%%%%%%%%%%%%%%%%%%%%%%%%%%%%%%%%
\section{Bottleneck Distance}
%%%%%%%%%%%%%%%%%%%%%%%%%%%%%%%%%%%%%%%%%%%%%%%%%%%
%%%%%%%%%%%%%%%%%%%%%%%%%%%%%%%%%%%%%%%%%%%%%%%%%%%

We adopt the following conventions. For $n\in\bbN$ write $[n]=\{1,\dots,n\}$. If $X \cong \bigoplus_{i=1}^{n} X_i$ is a finite direct sum of indecomposables, write $\Summands(X)=\{X_i \mid i\in [n]\}$; for finite sets $A,B$ with $|A|=|B|$, let $\Bij(A,B)$ be the set of bijections $A\to B$. Throughout, $(\bfP,d_{\bfP})$ is a finite metric poset.

For each $M\in\vect^{\bfP}$ fix once and for all a decomposition $M\cong\Ds_{i=1}^n M_i$ into indecomposables. Set $|M|:=n$ (the \emph{size}) and $\Summands(M):=\{M_i\mid i\in[n]\}$ (the \emph{summand set}); these are well defined up to isomorphism and the elements of $\Summands(M)$ are pairwise distinct. A minimal projective resolution of $M$ is denoted $P\down^M$ and is unique up to isomorphism of exact complexes. Let
\[
\Res(M)\quad\text{denote the set of all projective resolutions of }M.
\]
If $R\down=(R_i,\partial_i)_{i\ge 0}\in\Res(M)$, its \emph{size vector} is $|R\down|:=(|R_i|)_{i\ge 0}$. For indecomposable projectives $U,V$ (identified with representables $U\cong\bfP(x,-)$, $V\cong\bfP(y,-)$), put
\[
\dist(U,V):=d_{\bfP}(x,y).
\]

If $E$ is any projective object in $\vect^{\bfP}$, the mapping cone $\Cone(\id_E)$ is the two-term contractible complex
\[
\cdots \to 0 \to E \xrightarrow{\ \id\ } E \to 0 \to \cdots,
\]
concentrated in consecutive degrees; its shift $\Cone(\id_E)[a]$ places the two copies of $E$ in degrees $a$ and $a-1$. Direct-summing $\Cone(\id_E)[a]$ with a projective resolution leaves the resolved module unchanged and yields a chain-homotopy equivalent resolution (we call this \emph{padding by a contractible cone}).

\begin{lemma}
\label{lem:all-prj-resol}
Every projective resolution of $M$ is obtained from the minimal one by padding with contractible cones:
\[
\Res(M)
\ =\
\Bigl\{
\ P\down^M\ \ds\ \Ds_{i\in[n]}\,\Cone(\id_{E_i})[a_i]\
\Bigm|\ n\in\bbN,\ a_i\in\bbN,\ a_i\ge 1,\ E_i\in\prj\bfP\ 
\Bigr\}.
\]
\end{lemma}

%%%%%%%%%%%%%%%%%%%%%%%%%%%%%%%%%%%%%%%%%%%%%%%%%%%%%%%%%%%%%%%%%%%%%%%%%%%%%%%%%%%%%%%%%%%%%%%%%%%%%%%%%%%%%%%%%%
\subsection{Matchings}
%%%%%%%%%%%%%%%%%%%%%%%%%%%%%%%%%%%%%%%%%%%%%%%%%%%%%%%%%%%%%%%%%%%%%%%%%%%%%%%%%%%%%%%%%%%%%%%%%%%%%%%%%%%%%%%%%%

We now define degreewise matchings between two resolutions. Given $M,N\in\vect^{\bfP}$, consider pairs of projective resolutions with the same size vector:
\[
\Res(P\down^M,P\down^N)
\ :=\
\bigl\{\, (E\down,F\down)\in\Res(M)\times\Res(N)\ \bigm|\ |E\down|=|F\down|\,\bigr\}.
\]

\begin{proposition}
\label{prop:common-padding}
If the size vectors of $P\down^M$ and $P\down^N$ are finitely supported and have the same alternating sum
\[
\sum_{i\ge 0}(-1)^i|P^M_i|\ =\ \sum_{i\ge 0}(-1)^i|P^N_i|,
\]
then $\Res(P\down^M,P\down^N)\neq\emptyset$.
\end{proposition}

\begin{proof}
Let $p=(|P^M_i|)$ and $q=(|P^N_i|)$. Adding $\Cone(\id_E)[a]$ with $a\ge 1$ increases the size vector by the elementary vector $e^{(a)}$ having $1$ in degrees $a$ and $a-1$ (and $0$ elsewhere). These $e^{(a)}$ generate, as an abelian group, the kernel of the alternating-sum map $\alpha:\bigoplus_{i\ge 0}\bbZ\to\bbZ$, $\alpha(r)=\sum_i(-1)^ir_i$. Since $\alpha(p)=\alpha(q)$, the difference $q-p$ lies in this kernel and can be written $T^+-T^-$ with $T^\pm$ finite nonnegative combinations of the $e^{(a)}$. Then $p+T^+=q+T^-$ is a common upper bound, realized by padding $P\down^M$ by the cones in $T^+$ and $P\down^N$ by those in $T^-$. Hence there exist $(E\down,F\down)$ with $|E\down|=|F\down|$.
\end{proof}

For $(E\down,F\down)\in\Res(P\down^M,P\down^N)$, a \emph{matching} is a family of degreewise bijections
\[
B=(B_i)_{i\ge 0},\qquad B_i\in\Bij\bigl(\Summands(E_i),\,\Summands(F_i)\bigr).
\]
Define its \emph{cost} as the $L^\infty$–type aggregate of the underlying poset metric,
\[
\cost(B)\ :=\
\sup\bigl\{\,\dist\bigl(U,\ B_i(U)\bigr)\ \bigm|\ i\ge 0,\ U\in\Summands(E_i)\,\bigr\}.
\]
(Equivalently, $\cost(B)=\sup\{\dist(B_i^{-1}(V),V)\mid i\ge 0,\ V\in\Summands(F_i)\}$.)  
We refer to the quantity
\[
\dist_R(E\down,F\down)\ :=\ \inf_{\,B\in\Match(E\down,F\down)}\ \cost(B)
\]
as the \emph{matching distance} for resolutions with a fixed size vector.

\begin{lemma}[Triangle inequality for $\dist_R$ on a fixed size vector]
\label{lem:reg-triangle}
If $|E\down|=|F\down|=|G\down|$, then
\[
\dist_R(E\down,G\down)\ \le\ \max\bigl\{\dist_R(E\down,F\down),\ \dist_R(F\down,G\down)\bigr\}.
\]
\end{lemma}

\noindent\emph{Proof sketch.}
Compose nearly optimal degreewise bijections and take suprema across degrees.

%%%%%%%%%%%%%%%%%%%%%%%%%%%%%%%%%%%%%%%%%%%%%%%%%%%%%%%%%
%%%%%%%%%%%%%%%%%%%%%%%%%%%%%%%%%%%%%%%%%%%%%%%%%%%%%%%%%
\subsection{Matching Distance}
%%%%%%%%%%%%%%%%%%%%%%%%%%%%%%%%%%%%%%%%%%%%%%%%%%%%%%%%%
%%%%%%%%%%%%%%%%%%%%%%%%%%%%%%%%%%%%%%%%%%%%%%%%%%%%%%%%%

With matchings and \(\dist_R\) in hand, we define the global distance by allowing padding.

\begin{definition}
For minimal resolutions $P\down^M$ and $P\down^N$ (not necessarily of the same size), the \emph{bottleneck distance} is
\[
\dist_B\bigl(P\down^M,P\down^N\bigr)
\ :=\ 
\inf_{\ (E\down,F\down)\in\Res(P\down^M,P\down^N)}\ \dist_R(E\down,F\down).
\]
We adopt the extended-value convention that $\dist_B\bigl(P\down^M,P\down^N\bigr)=\infty$ if there is no compatible padding (i.e.\ $\Res(P\down^M,P\down^N)=\varnothing$).
\end{definition}

Proposition~\ref{prop:common-padding} gives a sufficient condition for finiteness (equality of alternating sums); in general, $\dist_B$ may be infinite.

\begin{proposition}
\label{prop:bneck-pseudometric}
On the class of minimal projective resolutions $\{P\down^M\mid M\in\vect^{\bfP}\}$, the function $\dist_B$ is an extended metric:
\[
\dist_B:\ \{P\down^M\}\times\{P\down^N\}\longrightarrow [0,+\infty].
\]
\end{proposition}

\begin{proof}
Nonnegativity is immediate. For any $M$, take the same padding on both sides and the identity matching degreewise to obtain $\dist_B(P\down^M,P\down^M)=0$. Symmetry holds because each matching family $B=(B_i)$ has an inverse family of the same cost. For the triangle inequality, given $P\down^M,P\down^N,P\down^O$, choose paddings producing $(E\down,F\down)$ and $(F'\down,G\down)$ with equal size vectors for $(M,N)$ and $(N,O)$ and with $\dist_R$ arbitrarily close to $\dist_B(M,N)$ and $\dist_B(N,O)$. Pad further to a common refinement so both pairs share the same size vector, compose the degreewise bijections, and apply Lemma~\ref{lem:reg-triangle}; taking infima over paddings yields the inequality. Finally, if $\dist_B(P\down^M,P\down^N)=0$, the costs lie in the finite set $\{d_{\bfP}(x,y)\mid x,y\in\bfP\}$, so the infimum is attained: there exist paddings and a matching of cost $0$. Thus each matched pair of summands has equal index in $\bfP$, the padded resolutions agree termwise up to isomorphism, and removing contractible cones yields $P\down^M\cong P\down^N$. Hence $\dist_B$ is an extended metric.
\end{proof}




%%%%%%%%%%%%%%%%%%%%%%%%%%%%%%%%%%%%%%%%%%%%%%%%%%%%%%%%%%%
%%%%%%%%%%%%%%%%%%%%%%%%%%%%%%%%%%%%%%%%%%%%%%%%%%%%%%%%%%%
\section{Stability Theorem}
%%%%%%%%%%%%%%%%%%%%%%%%%%%%%%%%%%%%%%%%%%%%%%%%%%%%%%%%%%%
%%%%%%%%%%%%%%%%%%%%%%%%%%%%%%%%%%%%%%%%%%%%%%%%%%%%%%%%%%%

We now relate the two distances defined above. Informally: a Galois coupling of $M$ and $N$ controls, via restriction, a pair of projective resolutions whose degreewise summands can be matched with cost bounded by the coupling cost. Hence the bottleneck distance between minimal projective resolutions is at most the Galois transport distance.

The next lemma says that pulling back along the right adjoint of a Galois connection sends the indecomposable projective at $x\in\bfQ$ to the indecomposable projective at $f(x)\in\bfP$.

\begin{lemma}
\label{lem:rt-adj-prj-ind}
If $f : \bfQ \rightleftarrows \bfP : g$ is a Galois connection of posets, then
\[
g^\ast\bigl(\bfQ(x,\blank)\bigr)\ \cong\ \bfP\bigl(f(x), \blank\bigr)
\qquad\text{in }\vect^\bfP\ \text{for all }x\in\bfQ.
\]
\end{lemma}

\begin{proof}
By Corollary~\ref{cor:3_adj} we have a natural isomorphism $g^{*}\cong f_{!}$. For any $M\in\vect^{\bfP}$,
\[
\Hom_{\vect^{\bfP}}\bigl(\bfP(f(x),-),\,M\bigr)\ \cong\ M\bigl(f(x)\bigr)
\ \cong\ (f^{*}M)(x)
\ \cong\ \Hom_{\vect^{\bfQ}}\bigl(\bfQ(x,-),\,f^{*}M\bigr)
\ \cong\ \Hom_{\vect^{\bfP}}\bigl(f_{!}\bfQ(x,-),\,M\bigr).
\]
By Yoneda, $f_{!}\bfQ(x,-)\cong \bfP(f(x),-)$, hence $g^{*}\bfQ(x,-)\cong f_{!}\bfQ(x,-)\cong \bfP(f(x),-)$.
\end{proof}

\begin{theorem}[Stability]
Let $(\bfP, d_\bfP)$ be a finite metric poset. Then for any $M, N \in \vect^\bfP$,
\[
\dist_B\bigl(P\down^M, P\down^N\bigr)\ \le\ \dGT(M, N).
\]
\end{theorem}

\begin{proof}
If there is no Galois coupling of $(M,N)$, then $\dGT(M,N)=\infty$ and the claim is tautological. Otherwise fix $\varepsilon>0$ and choose a coupling $(\bfQ, f \dashv g,\ h \dashv i,\ \Gamma)$ with
\[
g^\ast\Gamma\cong M,\qquad i^\ast\Gamma\cong N,\qquad \cost(\Gamma)\ \le\ \dGT(M,N)+\varepsilon.
\]
Let $R\down\to\Gamma$ be any projective resolution in $\vect^{\bfQ}$. Since precomposition is exact (Proposition~\ref{prop:Kan-posets}) and, for a Galois connection, preserves projectives (Proposition~\ref{prop:pullback_preserves_projectives}), the complexes
\[
E\down:=g^{*}R\down\quad\text{and}\quad F\down:=i^{*}R\down
\]
are projective resolutions of $M$ and $N$ in $\vect^{\bfP}$.

Write each degree $R_i$ as a finite direct sum of representables $R_i\cong\bigoplus_{x\in S_i}\bfQ(x,-)$ (Lemma~\ref{lem:ind-proj}). By Lemma~\ref{lem:rt-adj-prj-ind} (and the analogue for $h\dashv i$),
\[
E_i\ \cong\ \bigoplus_{x\in S_i}\bfP\bigl(f(x),-\bigr),
\qquad
F_i\ \cong\ \bigoplus_{x\in S_i}\bfP\bigl(h(x),-\bigr).
\]
Hence $|E\down|=|F\down|$ and $(E\down,F\down)\in\Res(P\down^M,P\down^N)$. Define the degreewise matching $B_i$ by the identity on indices $x\in S_i$:
\[
B_i:\ \bfP\bigl(f(x),-\bigr)\longmapsto \bfP\bigl(h(x),-\bigr)\qquad(x\in S_i).
\]
Then
\[
\dist_R(E\down,F\down)\ \le\ \cost(B)\ =\ \sup_{i}\ \sup_{x\in S_i} d_{\bfP}\bigl(f(x),h(x)\bigr)
\ \le\ \sup_{x\in\bfQ} d_{\bfP}\bigl(f(x),h(x)\bigr)\ =\ \cost(\Gamma).
\]
Taking the infimum over all compatible paddings yields
\[
\dist_B\bigl(P\down^M,P\down^N\bigr)\ \le\ \dist_R(E\down,F\down)\ \le\ \cost(\Gamma)\ \le\ \dGT(M,N)+\varepsilon.
\]
Letting $\varepsilon\to 0$ completes the proof.
\end{proof}

\begin{example}
Insert example we constructed in Fort Collins
\end{example}


%%%%%%%%%%%%%%%%%%%%%%%%%%%%%%%%%%%%%%%%%%%%%%%%%%%%%%%%%%%
%%%%%%%%%%%%%%%%%%%%%%%%%%%%%%%%%%%%%%%%%%%%%%%%%%%%%%%%%%%
\section{Application to Persistence}
%%%%%%%%%%%%%%%%%%%%%%%%%%%%%%%%%%%%%%%%%%%%%%%%%%%%%%%%%%%
%%%%%%%%%%%%%%%%%%%%%%%%%%%%%%%%%%%%%%%%%%%%%%%%%%%%%%%%%%%

We extract a persistence construction from a $\bfP$-module by passing to an interval poset and taking kernels of structure maps. The resulting “diagram” admits a stability inequality that recovers the classical bottleneck stability when $\bfP$ is totally ordered.

\begin{definition}
Let $(\bfP,d_{\bfP})$ be a finite metric poset with a top element $\top$.
We extend $d_{\bfP}$ to an \emph{extended} metric by
\[
d_{\bfP}(x,\top)=d_{\bfP}(\top,x)=+\infty\ \ (x\neq \top),\qquad d_{\bfP}(\top,\top)=0.
\]
Define the \emph{interval poset} $\Int\bfP$ to have objects the intervals $[x,y]$ with $x\le y\le \top$ in $\bfP$ and order
\[
[x_1,y_1]\ \le\ [x_2,y_2]\quad\Longleftrightarrow\quad x_1\le x_2\ \text{ and }\ y_1\le y_2.
\]
Equip $\Int\bfP$ with the product $L^\infty$ extended metric
\[
d_{\Int\bfP}\bigl([x_1,y_1],[x_2,y_2]\bigr)\ :=\ \max\{\,d_{\bfP}(x_1,x_2),\ d_{\bfP}(y_1,y_2)\,\}.
\]
\end{definition}

\begin{definition}
For $M\in\vect^{\bfP}$ define $K(M)\in\vect^{\Int\bfP}$ by
\[
K(M)([x,y])\ :=\
\begin{cases}
\ker\bigl(M(x\to y)\bigr), & y<\top,\\[2pt]
M(x), & y=\top,
\end{cases}
\]
and for a relation $[x_1,y_1]\le [x_2,y_2]$ let $K(M)([x_1,y_1]\to [x_2,y_2])$ be the map induced by $M(x_1\to x_2)$, which sends $\ker(M(x_1\to y_1))$ into $\ker(M(x_2\to y_2))$ when $y_2<\top$ by functoriality of $M$. This defines a functor $K:\vect^{\bfP}\to\vect^{\Int\bfP}$.
\end{definition}

We record two lemmas that will be used implicitly below.

\begin{lemma}\label{lem:Int-Galois}
If $f:\bfQ \rightleftarrows \bfP:g$ is a Galois connection, then so is
\[
\Int(f):\Int\bfQ \rightleftarrows \Int\bfP:\Int(g),
\qquad
\Int(f)[u,v]=[f(u),f(v)],\ \ \Int(g)[x,y]=[g(x),g(y)].
\]
\end{lemma}

\begin{proof}
Monotonicity of $\Int(f)$ and $\Int(g)$ follows from that of $f$ and $g$.
For $[u_1,v_1]\in\Int\bfQ$ and $[x_2,y_2]\in\Int\bfP$,
\[
\begin{aligned}
\Int(f)[u_1,v_1]\le [x_2,y_2]
&\iff \bigl(f(u_1)\le x_2\bigr)\ \text{and}\ \bigl(f(v_1)\le y_2\bigr)\\
&\iff \bigl(u_1\le g(x_2)\bigr)\ \text{and}\ \bigl(v_1\le g(y_2)\bigr)\\
&\iff [u_1,v_1]\le \Int(g)[x_2,y_2],
\end{aligned}
\]
using $f\dashv g$ coordinatewise. Thus $\Int(f)\dashv\Int(g)$.
\end{proof}

\begin{lemma}\label{lem:Int-Lipschitz}
For any monotone maps $f,h:\bfQ\to\bfP$,
\[
\sup_{[u,v]\in\Int\bfQ}
d_{\Int\bfP}\bigl(\Int(f)[u,v],\,\Int(h)[u,v]\bigr)
\ \le\
\sup_{u\in\bfQ} d_{\bfP}\bigl(f(u),\,h(u)\bigr).
\]
\end{lemma}

\begin{proof}
By the $L^\infty$ metric on $\Int\bfP$,
\[
d_{\Int\bfP}\bigl([f(u),f(v)],\,[h(u),h(v)]\bigr)
=\max\bigl\{\,d_{\bfP}(f(u),h(u)),\ d_{\bfP}(f(v),h(v))\,\bigr\}
\le \sup_{w\in\bfQ} d_{\bfP}(f(w),h(w)).
\]
Taking the supremum over all $[u,v]\in\Int\bfQ$ gives the claim.
\end{proof}

\begin{lemma}\label{lem:K-commute}
For any monotone $g:\bfP\to\bfQ$ there is a natural isomorphism of functors
\[
\Int(g)^{*}\circ K\ \ \cong\ \ K\circ g^{*}:\ \vect^{\bfQ}\longrightarrow \vect^{\Int\bfP}.
\]
\end{lemma}

\begin{proof}
Evaluate both composites at $M\in\vect^{\bfQ}$ and $[x,y]\in\Int\bfP$. If $y<\top$,
\[
(\Int(g)^{*}K(M))([x,y])\ =\ K(M)\bigl([g(x),g(y)]\bigr)\ =\ \ker\bigl(M(g(x)\to g(y))\bigr),
\]
\[
(K(g^{*}M))([x,y])\ =\ \ker\bigl((g^{*}M)(x\to y)\bigr)\ =\ \ker\bigl(M(g(x)\to g(y))\bigr).
\]
If $y=\top$, both sides equal $M(g(x))$. Naturality in $[x,y]$ follows from functoriality of $M$.
\end{proof}

\begin{proposition}\label{prop:K-Lipschitz}
For all $M,N\in\vect^{\bfP}$ one has
\[
\dGT^{\Int\bfP}\bigl(K(M),K(N)\bigr)\ \le\ \dGT^{\bfP}(M,N).
\]
\end{proposition}

\begin{proof}
Given a Galois coupling $(\bfQ,f\dashv g,\ h\dashv i,\ \Gamma)$ for $(M,N)$ in $\vect^{\bfP}$, Lemma~\ref{lem:Int-Galois} yields a Galois coupling
\(
\Int(f)\dashv\Int(g),\ \Int(h)\dashv\Int(i)
\)
on $\Int\bfQ\rightleftarrows\Int\bfP$. Take the apex module $K(\Gamma)\in\vect^{\Int\bfQ}$. By Lemma~\ref{lem:K-commute},
\[
\Int(g)^{*}K(\Gamma)\ \cong\ K(g^{*}\Gamma)\ \cong\ K(M),\qquad
\Int(i)^{*}K(\Gamma)\ \cong\ K(i^{*}\Gamma)\ \cong\ K(N),
\]
so this is a coupling for $(K(M),K(N))$. Its cost is bounded by Lemma~\ref{lem:Int-Lipschitz}. Taking infima gives the claim.
\end{proof}

\begin{definition}
For $M\in\vect^{\bfP}$, let $K\down^{M}$ denote a minimal projective resolution of $K(M)$ in $\vect^{\Int\bfP}$. We call the family of its degreewise indecomposable projective summands
\[
\bigl\{\Summands\bigl(K^{M}_i\bigr)\bigr\}_{i\ge 0}
\]
the \emph{persistence diagram} of $M$ over $\bfP$. (Equivalently, the isomorphism class of the minimal resolution $K\down^{M}$ encodes the diagram.)
\end{definition}

Padding a resolution by contractible cones $\Cone(\id_E)[a]$ adds one copy of $E$ in consecutive degrees $a$ and $a-1$ without changing the resolved module. In our bottleneck framework this plays the role of adding \emph{diagonal points} in classical matching: it equalizes degreewise sizes, and cones added symmetrically on both sides can be matched at zero cost. Thus diagonal padding is implemented categorically by homotopically trivial cones.

\begin{theorem}
For all $M,N\in\vect^{\bfP}$,
\[
\dist_B\bigl(K\down^{M},K\down^{N}\bigr)\ \le\ \dGT^{\Int\bfP}\bigl(K(M),K(N)\bigr)\ \le\ \dGT^{\bfP}(M,N).
\]
In particular, if $\bfP=\{1<\cdots<n\}$ with $d_{\bfP}(x,y)=|x-y|$ and $\top$ adjoined with $d_{\bfP}(\cdot,\top)=\infty$, this recovers the classical bottleneck stability inequality.
\end{theorem}

\begin{proof}
Apply the Stability Theorem with the base poset replaced by $\Int\bfP$ and the modules $K(M),K(N)$ to get the first inequality. The second inequality is Proposition~\ref{prop:K-Lipschitz}.
\end{proof}

\begin{example}
Do a classical 1-parameter persistence example. Demonstrate matching.
\end{example}

\begin{example}
Create a 2-parameter persistence example.
\end{example}



%%%%%%%%%%%%%%%%%%%%%%%%%%%%%%%%%%%%%%%%%%%%%%%%%%%%%%%%%%%
\appendix
%%%%%%%%%%%%%%%%%%%%%%%%%%%%%%%%%%%%%%%%%%%%%%%%%%%%%%%%%%%

%%%%%%%%%%%%%%%%%%%%%%%%%%%%%%%%%%%%%%%%%%%%%%%%%%%%%%%%%%%%
%%%%%%%%%%%%%%%%%%%%%%%%%%%%%%%%%%%%%%%%%%%%%%%%%%%%%%%%%%%%
%\section{Interleaving Distance}
%\label{sec:intlv-dist}
%%%%%%%%%%%%%%%%%%%%%%%%%%%%%%%%%%%%%%%%%%%%%%%%%%%%%%%%%%%%
%%%%%%%%%%%%%%%%%%%%%%%%%%%%%%%%%%%%%%%%%%%%%%%%%%%%%%%%%%%%
%
%The Galois transport distance can be viewed as an adaptation of the classical interleaving distance from persistence theory. In this paper we work with modules over a finite poset $\bfP$, whereas classical persistence uses the totally ordered real line $(\mathbb{R},\le)$. Although persistence modules over $\mathbb{R}$ are not required to be finite, in practice one often restricts attention to a finite totally ordered subposet (e.g.\ determined by selected parameter values or critical times) and extends from there. This places interleavings and Galois transports in a common finite setting for comparison.
%
%
%\subsection*{Definition}
%
%We recall the standard categorical definition using the shift endofunctor
%(see \cite{ChazalCohenSteinerGlisseGuibasOudot2009,Lesnick2015}).
%
%\begin{definition}[Interleaving distance]\label{def:interleaving-distance}
%Let $\bfP=(\mathbb{R},\le)$. For $\varepsilon\ge 0$, set
%$T_\varepsilon:\vect^\bbR \to\vect^\bbR$ by
%$(T_\varepsilon M)(r)=M(r+\varepsilon)$. An \emph{$\varepsilon$-interleaving}
%between $M,N\in \vect^\bbR$ is a pair of natural transformations
%$\varphi:M\Rightarrow T_\varepsilon N$ and $\psi:N\Rightarrow T_\varepsilon M$
%such that the composites $M\to T_{2\varepsilon}M$ and
%$N\to T_{2\varepsilon}N$ equal the canonical structure maps.  The
%\emph{interleaving distance} is
%\[
%d_I(M,N):=\inf\{\ \varepsilon\ge 0\mid M,N\ \text{are $\varepsilon$-interleaved}\ \}.
%\]
%\end{definition}
%
%\subsection*{Two basic facts}
%
%\begin{proposition}[Tail agreement implies finiteness]\label{prop:tail-agree-vanish}
%If $M,N \in \vect^\bbR$ satisfy $M(r)=N(r)=0$ for all $r\le b$ and
%$M|_{[R,\infty)}\cong N|_{[R,\infty)}$, then $d_I(M,N)\le R-b$.
%\end{proposition}
%
%\begin{proof}
%Let $\varepsilon=R-b$.  For $r>b$, define $\varphi_r$ by following
%$M(r)\to M(r+\varepsilon)\xrightarrow{\theta_{r+\varepsilon}}N(r+\varepsilon)$,
%and symmetrically define $\psi_r$.  For $r\le b$ both $M(r)$ and $N(r)$ vanish,
%so the maps are zero.  One checks that these assignments are natural, and that
%the interleaving equations hold: beyond $R$ the isomorphism $\theta$ cancels,
%and before $b$ everything is zero.  Hence $d_I(M,N)\le\varepsilon$.
%\end{proof}
%
%\begin{proposition}[Pseudo-metric and separation]\label{prop:dI-pseudometric}
%For $M,N,O \in \vect^\bbR$,
%\[
%d_I(M,N)=d_I(N,M),\qquad
%d_I(M,O)\le d_I(M,N)+d_I(N,O),\qquad
%d_I(M,M)=0.
%\]
%Moreover, $d_I(M,N)=0$ iff $M\cong N$. Hence $d_I$ is an extended
%metric on isomorphism classes.
%\end{proposition}
%
%\begin{proof}
%Symmetry is clear by swapping $\varphi,\psi$.  
%For the triangle inequality, if $M,N$ are $\varepsilon$-interleaved and $N,O$ are $\delta$-interleaved, then by pasting the data one obtains an $(\varepsilon+\delta)$-interleaving of $M,O$; the details are a standard exercise.  
%Finally, $d_I(M,M)=0$ via the identity maps.  
%If $d_I(M,N)=0$ then a $0$-interleaving exists, giving mutually inverse isomorphisms $M\cong N$.
%\end{proof}
%
%\subsection*{Relating interleavings and Galois couplings on $\mathbb{R}$}
%
%For this subsection fix $\bfP=(\mathbb{R},\le)$ with metric $d(x,y)=|x-y|$.
%For $\varepsilon\ge 0$, the translations
%$t_\varepsilon(x)=x+\varepsilon$ and $r_\varepsilon(x)=x-\varepsilon$
%satisfy $t_\varepsilon\dashv r_\varepsilon$, so $t_\varepsilon\dashv r_\varepsilon$ is a Galois insertion.
%
%\begin{proposition}[Interleaving $\Rightarrow$ coupling]\label{prop:int-to-galois-R}
%If $M,N\in \vect^\bbR$ are $\varepsilon$-interleaved, there exists a
%Galois coupling $(\bfX,f\dashv g,\ h\dashv i,\ \Gamma)$ for $(M,N)$ with
%$\bfX=\mathbb{R}_L\sqcup\mathbb{R}_R$ and $\operatorname{cost}(\Gamma)=\varepsilon$.
%\end{proposition}
%
%\begin{proof}
%Take two copies of $\mathbb{R}$, ordered so that cross-inequalities encode the shift:
%$a_L\le b_R$ iff $a+\varepsilon\le b$, and symmetrically for $R$ to $L$.
%Let $f,h:\bfX\to\mathbb{R}$ be the evident projections, with $f(a_L)=a$ and $f(a_R)=a+\varepsilon$, etc.
%Define $\Gamma$ by $\Gamma(a_L)=M(a)$ and $\Gamma(a_R)=N(a)$, using the interleaving maps to define $\Gamma$ on cross arrows.  
%By construction $g^\ast\Gamma\cong M$, $i^\ast\Gamma\cong N$, and the cost is $\sup|f(x)-h(x)|=\varepsilon$.  
%Functoriality of $\Gamma$ on mixed zigzags is ensured by the interleaving identities; the reader can check this directly.
%\end{proof}
%
%\begin{proposition}[Coupling $\Rightarrow$ interleaving]\label{prop:coupling-implies-int-R}
%Let $(\bfX,f\dashv g,\ h\dashv i,\ \Gamma)$ be a Galois coupling for
%$(M,N)\in \vect^\bbR$ with $\sup_{x\in \bfX}|f(x)-h(x)|\le\varepsilon$.
%Then $M$ and $N$ are $\varepsilon$-interleaved.
%\end{proposition}
%
%\begin{proof}
%Fix $r\in\mathbb{R}$.  Using $x=g(r)$, we have $f(x)=r$ and $h(x)\le r+\varepsilon$.
%Define $\varphi_r:M(r)\to N(r+\varepsilon)$ by
%\[
%M(r)\xrightarrow[\cong]{\alpha_r}\Gamma(x)\xrightarrow{\Gamma(\eta^h_x)}\Gamma(i(h(x)))
%\xrightarrow[\cong]{\beta_{h(x)}}N(h(x))\to N(r+\varepsilon).
%\]
%Here $\eta^h$ is the unit of $h\dashv i$.  Similarly, using $y=i(r)$ one defines
%\[
%\psi_r:\ N(r)\to M(r+\varepsilon).
%\]
%Naturality follows from functoriality of $\Gamma$ and the seam maps.  
%For the interleaving identities, one composes $\varphi$ and $\psi$ and uses the triangle identities of the adjunctions, together with the cost condition ensuring $h(g(r))\le r+\varepsilon$, etc.  
%This reduces exactly to the canonical structure maps $M(r)\to M(r+2\varepsilon)$ and $N(r)\to N(r+2\varepsilon)$.  
%Routine but slightly tedious diagram chases are left to the reader.
%\end{proof}
%
%\subsection*{Equality of distances on $\mathbb{R}$}
%
%\begin{theorem}\label{thm:GT-equals-dI-R}
%For all $M,N\in \vect^\bbR$, the Galois transport distance equals
%the interleaving distance:
%\[
%\dGT(M,N)=d_I(M,N).
%\]
%\end{theorem}
%
%\begin{proof}
%By Proposition~\ref{prop:int-to-galois-R}, any $\varepsilon$-interleaving yields a coupling of cost $\varepsilon$, so $\GT\le d_I$.  
%Conversely, by Proposition~\ref{prop:coupling-implies-int-R}, any coupling of cost $\varepsilon$ yields an $\varepsilon$-interleaving, so $d_I\le \GT$.  
%Thus the two distances coincide.
%\end{proof}
%






\bibliographystyle{plain}

\begin{thebibliography}{10}
\bibitem{MR4402576}
Hideto Asashiba, Micka\"{e}l Buchet, Emerson~G. Escolar, Ken Nakashima, and Michio Yoshiwaki.
\newblock On interval decomposability of 2{D} persistence modules.
\newblock {\em Computational Geometry}, 105/106:Paper No. 101879, 33, 2022.
\newblock \url{https://doi.org/10.1016/j.comgeo.2022.101879}.

\bibitem{ASASHIBA2023107397}
Hideto Asashiba, Emerson~G. Escolar, Ken Nakashima, and Michio Yoshiwaki.
\newblock Approximation by interval-decomposables and interval resolutions of persistence modules.
\newblock {\em Journal of Pure and Applied Algebra}, 227(10):107397, 2023.
\newblock \url{https://doi.org/10.1016/j.jpaa.2023.107397}.

\bibitem{ASASHIBA2023100007}
Hideto Asashiba, Emerson~G. Escolar, Ken Nakashima, and Michio Yoshiwaki.
\newblock On approximation of 2{D} persistence modules by interval-decomposables.
\newblock {\em Journal of Computational Algebra}, 6--7:100007, 2023.
\newblock \url{https://doi.org/10.1016/j.jaca.2023.100007}.

\bibitem{aoki2024bipath}
Toshitaka Aoki, Emerson~G. Escolar, and Shunsuke Tada.
\newblock Bipath persistence.
\newblock {\em arXiv preprint}, arXiv:2404.02536, 2024.
\newblock \url{https://doi.org/10.48550/arXiv.2404.02536}.

\bibitem{aoki2023summand}
Toshitaka Aoki, Emerson~G. Escolar, and Shunsuke Tada.
\newblock Summand-injectivity of interval covers and monotonicity of interval resolution global dimensions.
\newblock {\em arXiv preprint}, arXiv:2308.14979, 2023.
\newblock \url{https://doi.org/10.48550/arXiv.2308.14979}.

\bibitem{asashiba2024interval}
Hideto Asashiba, Etienne Gauthier, and Enhao Liu.
\newblock Interval replacements of persistence modules.
\newblock {\em arXiv preprint}, arXiv:2403.08308, 2024.
\newblock \url{https://doi.org/10.48550/arXiv.2403.08308}.

\bibitem{Asashiba2017}
Hideto Asashiba, Ken Nakashima, and Michio Yoshiwaki.
\newblock Decomposition theory of modules: the case of Kronecker algebra.
\newblock {\em Japan Journal of Industrial and Applied Mathematics}, 34(2):489--507, Aug 2017.
\newblock \url{https://doi.org/10.1007/s13160-017-0247-y}.

\bibitem{MR4091895}
Ulrich Bauer, Magnus~B. Botnan, Steffen Oppermann, and Johan Steen.
\newblock Cotorsion torsion triples and the representation theory of filtered hierarchical clustering.
\newblock {\em Advances in Mathematics}, 369:107171, 51, 2020.
\newblock \url{https://doi.org/10.1016/j.aim.2020.107171}.

\bibitem{BBH2024approximations}
Benjamin Blanchette, Thomas Br{\"u}stle, and Eric~J. Hanson.
\newblock Homological approximations in persistence theory.
\newblock {\em Canadian Journal of Mathematics}, 76(1):66--103, 2024.
\newblock \url{https://doi.org/10.4153/s0008414x22000657}.

\bibitem{dey2022fast}
Tamal~K Dey and Tao Hou.
\newblock Fast computation of zigzag persistence.
\newblock In {\em 30th Annual European Symposium on Algorithms (ESA 2022)}. Schloss Dagstuhl-Leibniz-Zentrum f{\"u}r Informatik, 2022.
\newblock \url{https://doi.org/10.4230/LIPIcs.ESA.2022.43}.

\bibitem{dey2023computing}
Tamal~K Dey, Woojin Kim, and Facundo M{\'e}moli.
\newblock Computing generalized rank invariant for 2-parameter persistence modules via zigzag persistence and its applications.
\newblock {\em Discrete {$\&$} Computational Geometry}, pages 1--28, 2023.
\newblock \url{https://doi.org/10.1007/s00454-023-00584-z}.

\bibitem{MR3824276}
Tamal~K. Dey and Cheng Xin.
\newblock Computing bottleneck distance for 2-{D} interval decomposable modules.
\newblock In {\em 34th {I}nternational {S}ymposium on {C}omputational {G}eometry}, volume~99 of {\em LIPIcs. Leibniz Int. Proc. Inform.}, pages Art. No. 32, 15. Schloss Dagstuhl. Leibniz-Zent. Inform., Wadern, 2018.
\newblock \url{https://doi.org/10.4230/LIPIcs.SoCG.2018.32}.

\bibitem{edelsbrunner2010computational}
Herbert Edelsbrunner and John L. Harer.
{\em Computational {T}opology: {A}n {I}ntroduction}. 
American Mathematical Society, Providence, 2010.
\newblock \url{https://doi.org/10.1090/mbk/069}.

\bibitem{MR1949898}
Herbert Edelsbrunner, David Letscher, and Afra Zomorodian.
\newblock Topological persistence and simplification.
\newblock {\em Discrete {$\&$} Computational Geometry}, 28(4):511--533, 2002.
\newblock \url{https://doi.org/10.1007/s00454-002-2885-2}.

\bibitem{gabriel2006auslander}
Peter Gabriel.
\newblock Auslander-Reiten sequences and representation-finite algebras.
\newblock In {\em Representation Theory I: Proceedings of the Workshop on the Present Trends in Representation Theory, Ottawa, Carleton University, August 13--18, 1979}, pages 1--71. Springer, 2006.
\newblock \url{https://doi.org/10.1007/BFb0089778}.

\bibitem{hiraoka2023refinement}
Yasuaki Hiraoka, Ken Nakashima, Ippei Obayashi, and Chenguang Xu.
\newblock Refinement of interval approximations for fully commutative quivers.
\newblock {\em arXiv preprint arXiv:2310.03649}, 2023.
\newblock \url{https://doi.org/10.48550/arXiv.2310.03649}.

\bibitem{JY}
Johnson; Donald Yau:
{\it 2-dimensional categories}%%%%%%% give a complete one

\bibitem{kim2021generalized}
Woojin Kim and Facundo M{\'e}moli.
\newblock Generalized persistence diagrams for persistence modules over posets.
\newblock {\em Journal of Applied and Computational Topology}, 5(4):533--581, 2021.
\newblock \url{https://doi.org/10.1007/s41468-021-00075-1}.

\bibitem{MR3728284}
James R. Munkres.
{\em Topology} second edition. 
Prentice Hall, Inc., Upper Saddle River, NJ, 2000.

\bibitem{MR2121296}
Afra Zomorodian and Gunnar Carlsson.
\newblock Computing persistent homology.
\newblock {\em Discrete $\&$ Computational Geometry}, 33(2):249--274, 2005.
\newblock \url{https://doi.org/10.1007/s00454-004-1146-y}.

\bibitem[CCSG+09]{ChazalCohenSteinerGlisseGuibasOudot2009} Fr{\'e}d{\'e}ric Chazal, David Cohen--Steiner, Marc Glisse, Leonidas J. Guibas, and Steve Y.
Oudot. {\it Proximity of persistence modules and their diagrams}. In Proceedings of the
25th Annual Symposium on Computational Geometry (SoCG), pages 237--246. ACM,
2009.

\bibitem[Cur18]{curry2018}
Justin Michael Curry: {\it Dualities between cellular sheaves and cosheaves}. Journal of
Pure and Applied Algebra, 222(4):966--993, 2018.

\bibitem[EP24]{ElchesenPatel}
Alex Elchesen and Amit Patel:
{\it A categorical approach to m{\"o}bius inversion via de-
rived functors}, 2024.

\bibitem[Les15]{Lesnick2015}
Michael Lesnick:
{\it The theory of the interleaving distance on multidimensional persistence modules}, Foundations of Computational Mathematics, 15(3):613--650, 2015.







\end{thebibliography}

\end{document}