\documentclass[12pt,letterpaper]{article}
\usepackage{amitstyle}

\DeclareMathOperator{\cost}{cost}
\DeclareMathOperator{\dist}{dist}        % the other metric used in the paper
\newcommand{\dGT}[2]{\dist_{\mathrm{GT}}(#1,#2)} % Galois-transport metric

\newcommand{\vect}{\mathsf{vec}}         % category of finite-dimensional vector spaces
\newcommand{\Pos}{\mathbf{Pos}}          % category of posets

\DeclareMathOperator{\Match}{Match}      % set of matchings (as defined in text)
\DeclareMathOperator{\Bij}{Bij}          % set of bijections
\DeclareMathOperator{\Cone}{Cone}        % cone construction
\DeclareMathOperator{\Summands}{Summands}% set of direct summands

\newcommand{\fU}{\mathfrak{U}}           % fixed universe for the 2-category
\newcommand{\GT}{\mathrm{GT}}            % (optional) label for the name in prose

\newcommand{\todoilgreen}[1]{\textcolor{green}{TODO: #1}}
\newcommand{\frkU}{\mathfrak{U}}

\newcommand\Nname[1]{|[alias=#1]|}

\newcommand\al{\alpha}

\newcommand\be{\beta}

\newcommand\ep{\varepsilon}

\newcommand\Ga{\Gamma}

\newcommand\prj{\operatorname{prj}}

\newcommand\calC{{\mathcal C}}

\newcommand\calD{{\mathcal D}}

\newcommand\calP{{\mathcal P}}

\newcommand\bbN{\mathbb{N}}

\newcommand\bbR{\mathbb{R}}

\newcommand\iso{\cong}

\newcommand\ds{\oplus}

\newcommand\Ds{\bigoplus}

\newcommand\id{1\kern-.25em{\text{{\rm l}}}} 

\newcommand\isoto{\ \raise.8ex\hbox{$^{\sim}$}\kern-.7em\hbox{$\to$}\ }

\newcommand\down{_{\cdot}}

\newcommand\blank{\operatorname{-}}



\newcommand\bfP{\mathbf{P}}

\newcommand\bfQ{\mathbf{Q}}

\newcommand\bfX{\mathbf{X}}

\newcommand\bfY{\mathbf{Y}}

\newcommand\bfZ{\mathbf{Z}}

\newcommand\sfU{\mathsf{U}}

\newcommand\Fun{\operatorname{Fun}}

\newcommand\Cat{\mathbf{Cat}}

% Representable at x in P
\newcommand{\Yp}[1]{\bfP(#1,-)}
\DeclareMathOperator{\Hom}{Hom}






\begin{document}

\title{Stability of minimal projective resolutions}
\author{Hideto Asashiba\thanks{Supported by JSPS Grant-in-Aid for Scientific Research (C) 18K03207; 
JSPS Grant-in-Aid for Transformative Research Areas (A) (22A201); 
Osaka Central Advanced Mathematical Institute (MEXT Promotion of Distinctive Joint Research Center Program JPMXP0723833165).}
\and Amit K. Patel}


\maketitle

\begin{abstract}

\end{abstract}

\noindent\textbf{2020 Mathematics Subject Classification.} 16G20, 16G70, 55N31, 62R40.

\medskip

\noindent\textbf{Affiliations.}
Department of Mathematics, Faculty of Science, Shizuoka University,
836 Ohya, Suruga-ku, Shizuoka, 422-8529, Japan; Institute for Advanced Study, KUIAS, Kyoto University,
Yoshida Ushinomiya-cho, Sakyo-ku, Kyoto 606-8501, Japan; and
Osaka Central Advanced Mathematical Institute, 3-3-138 Sugimoto, Sumiyoshi-ku, Osaka, 558-8585, Japan. \par
Department of Mathematics, Colorado State University, Fort Collins, CO 80523, USA.

\medskip

\noindent\textbf{Emails.}
\href{mailto:asashiba.hideto@shizuoka.ac.jp}{asashiba.hideto@shizuoka.ac.jp}
\quad|\quad
\href{mailto:Amit.Patel@colostate.edu}{Amit.Patel@colostate.edu}


\tableofcontents

\section{Introduction}

%%%%%%%%%%%%%%

\subsection{Purposes}

%%%%%%%%%%%%%%%%%%%

\subsection{Our contributions}


\subsection{Organization}


\section{Preliminaries}
\label{sec:prelim}

We start with some categorical facts that will be used throughout.

\begin{lemma}
\label{lem:pres-proj}
Let $L:\calC\to\calD$ be a functor between abelian categories.
If $L$ has an exact right adjoint $R:\calD\to\calC$,
then $L$ sends projectives to projectives.
\end{lemma}

\begin{proof}
Let $A$ be projective in $\calC$. Then the functor $\calC(A,-)$ is exact, hence so is
\[
\calC(A,-)\circ R=\calC(A,R(-))\cong \calD(L(A),-).
\]
Thus $L(A)$ is projective.
\end{proof}

Let $\frkU$ be a universe. An element of $\frkU$ is called a $\frkU$-\emph{small} set. A category $\calC$ is $\frkU$-\emph{small} if its object set is $\frkU$-small and, for any objects $X,Y$, each hom-set $\calC(X,Y)$ is $\frkU$-small. We denote by $\Cat^\frkU$ the $2$-category whose objects are the $\frkU$-small categories, whose $1$-morphisms are functors, and whose $2$-morphisms are natural transformations.

\begin{proposition}
\label{prp:Yoneda-2-functor}
If $\calC$ is an object of $\Cat^\frkU$, then
\[
\Cat^\frkU(\mbox{-},\calC):(\Cat^\frkU)^{\mathrm{op}}\longrightarrow\Cat^\frkU
\]
is a $2$-functor.
\end{proposition}

\begin{proof}
See Proposition~4.5.4 in \cite{JY}.
\end{proof}

\begin{remark}
\label{rmk:contravariant-adj}
In particular, the $2$-functor $\Cat^\frkU(\mbox{-},\calC)$ sends an adjoint system $(f,g,\eta,\varepsilon)$ to the adjoint system $(g^{*},f^{*},\eta^{*},\varepsilon^{*})$ in a contravariant manner. Concretely, if $f\dashv g$ between categories $\calA$ and $\calB$, then precomposition yields $g^{*}\dashv f^{*}$ between the functor categories $\Cat^\frkU(\calB,\calC)$ and $\Cat^\frkU(\calA,\calC)$. We use this repeatedly with $\calC=\vect$ below.
\end{remark}

%%%%%%%%%%%%%%%%%%%%%%%%%%%%%%%%%%%%%%%%%%%%%%%%%%%%%%%%%%%%%%%%%%%
\subsection{Poset Modules}
\label{ssec:poset-mod}
%%%%%%%%%%%%%%%%%%%%%%%%%%%%%%%%%%%%%%%%%%%%%%%%%%%%%%%%%%%%%%%%%%%

Throughout, $\Bbbk$ is a field, and $\bfP=(\bfP,\le)$ is a poset. We regard $\bfP$ as a small (thin) category: there is a unique morphism $x\to y$ exactly when $x\le y$. This lets us identify $\bfP$-modules with functors $\bfP\to\vect$ and, equivalently, with left modules over the $\Bbbk$-linearization $\Bbbk\bfP$ (same objects, order relations as basis arrows, composition by transitivity). We assume $\bfP$ is finite except in Sect.\ \ref{sec:intlv-dist}.

Let $\vect$ denote the category of finite-dimensional $\Bbbk$-vector spaces, and write $\vect^\bfP=\Fun(\bfP,\vect)$ for the category of $\bfP$-modules. If $\bfP$ is finite, then $\vect^\bfP$ is a $\Bbbk$-linear, abelian, Krull–Schmidt category, and for each $M,N\in\vect^\bfP$ the $\Bbbk$-space $\vect^\bfP(M,N)$ is finite dimensional.

We write $\prj\bfP$ for the full subcategory of projective objects in $\vect^\bfP$.
For $x\in\bfP$, the \emph{representable} at $x$ is the functor
\[
\Yp{x}:=\bfP(x,-)\in\vect^\bfP.
\]
It sends $y\in\bfP$ to $\Yp{x}(y)=\Bbbk$ if $x\le y$ and to $0$ otherwise; for a relation $y\le z$ it acts by the identity on $\Bbbk$ whenever $x\le y$ (and by $0$ otherwise). By the Yoneda lemma there is a natural isomorphism
\[
\Hom_{\vect^\bfP}(\Yp{x},M)\cong M(x)\qquad(M\in\vect^\bfP),
\]
showing that each $\Yp{x}$ is projective.

\begin{lemma}
\label{lem:ind-proj}
The family $\{\Yp{x}\mid x\in\bfP\}$ is a complete set of representatives of the isomorphism classes of indecomposable projective $\bfP$-modules. Hence each $M\in\prj \bfP$ decomposes uniquely as
\[
M \cong \bigoplus_{x\in\bfP} \Yp{x}^{\,a_x}
\]
for a uniquely determined family $(a_x)_{x\in\bfP}\in\bbN^\bfP$.
\end{lemma}

%%%%%%%%%%%%%%%%%%%%%%%%%%%%%%%%%%%%%%%%%%%%%%%%%%%%%%%%%%%%%%%%%%%
\subsection{Monotone Functions and Conextensions}
\label{ssec:mont-funct}
%%%%%%%%%%%%%%%%%%%%%%%%%%%%%%%%%%%%%%%%%%%%%%%%%%%%%%%%%%%%%%%%%%%

A map $f:\bfP\to\bfQ$ of posets is \emph{monotone} if $x\le y$ in $\bfP$ implies $f(x)\le f(y)$ in $\bfQ$. Viewing posets as thin categories, a functor $\bfP\to\bfQ$ is precisely a monotone map. Thus a monotone map $f:\bfP\to\bfQ$ induces the restriction (precomposition) functor
\[
f^{*}:\vect^\bfQ\to\vect^\bfP,\qquad f^{*}N = N\circ f .
\]

\begin{proposition}
\label{prop:Kan-posets}
For any monotone $f:\bfP\to\bfQ$, the functor $f^{*}$ is exact. Moreover, the left and right Kan extensions along $f$ exist and we denote them by
\[
f_{!}:=\Lan_{f}\qquad\text{and}\qquad f_{*}:=\Ran_{f}.
\]
By the defining property of conextensions there are natural adjunctions
\[
f_{!}\dashv f^{*}\dashv f_{*}.
\]
In particular, $f_{!}$ is right exact, $f_{*}$ is left exact, and $f^{*}$ is exact.
\end{proposition}

\[
\begin{tikzcd}[row sep=3em, column sep=6em]
\Nname{E}\vect^\bfP & \Nname{C}\vect^\bfQ
\arrow[from={E}, to={C}, "f_{!}", ""'{name=Fu}, bend left]
\arrow[from={C}, to={E},"f^{*}"{description, pos=.5, name=Fc}]
\arrow[from={E}, to={C}, "f_{*}"', ""'{name=Fd}, bend right]
\arrow[from={Fu}, to={Fc}, "\rotatebox{-90}{$\dashv$}" description, phantom]
\arrow[from={Fc}, to={Fd}, "\rotatebox{-90}{$\dashv$}" description, phantom]
\end{tikzcd}
\]


%%%%%%%%%%%%%%%%%%%%%%%%%%%%%%%%%%%%%%%%%%%%%%%%%%%%%%%%%%%%%%%%%%%
\subsection{Galois Connections}
%%%%%%%%%%%%%%%%%%%%%%%%%%%%%%%%%%%%%%%%%%%%%%%%%%%%%%%%%%%%%%%%%%%

We recall that Galois connections are adjunctions between posets and underlie the identifications among $f_{!}, f^{*}, f_{*}$ used later.

\begin{definition}
A \emph{Galois connection} between posets $\bfP,\bfQ$ consists of monotone maps $f:\bfP\to\bfQ$ and $g:\bfQ\to\bfP$ such that
\[
f(x)\le q \iff x\le g(q)\qquad(x\in \bfP,\ q\in \bfQ).
\]
Equivalently, viewing $\bfP$ and $\bfQ$ as thin categories, this means $f\dashv g$ as functors. We write $f:\bfP\rightleftarrows \bfQ:g$, with $f$ left adjoint and $g$ right adjoint.
\end{definition}

\begin{corollary}
If $f:\bfP\rightleftarrows \bfQ:g$ is a Galois connection, then $g^{*}\dashv f^{*}$.
\[
\begin{tikzcd}[row sep=3em]
\Nname{E}\vect^\bfP & \Nname{C}\vect^\bfQ 
\arrow[from={E}, to={C}, "g^{*}", ""'{name=Fd}, bend left]
\arrow[from={C}, to={E}, "f^{*}"{description, pos=.5, name=Gu}, bend left]
\arrow[from={Fd}, to={Gu}, "\rotatebox{-90}{$\dashv$}" description, phantom]
\end{tikzcd}.
\]
\end{corollary}

\begin{proof}
Immediate from the contravariant $2$-functoriality of $\Cat^\frkU(\mbox{-},\vect)$ (e.g.\ Proposition~4.5.4 in \cite{JY}).
\end{proof}

\begin{corollary}
\label{cor:3_adj}
In the same setting, there are natural isomorphisms $g^{*}\cong f_{!}$ and $f^{*}\cong g_{*}$. In particular, the adjoint pairs
\[
g^{*}\dashv f^{*},\qquad f_{!}\dashv f^{*},\qquad g^{*}\dashv g_{*}
\]
are compatible via these isomorphisms.
\[
\begin{tikzcd}[row sep=50pt, column sep = 70pt]
\Nname{E}\vect^\bfP & \Nname{C}\vect^\bfQ 
\arrow[from={E}, to={C}, "g^{*} \cong f_{!}"{description, pos=.5, name=Fu}, ""'{name=Fd}, bend left=15pt]
\arrow[from={C}, to={E}, "f^{*}\cong g_{*}"{description, pos=.5, name=Gd}, ""'{name=Gu}, bend left=15pt]
\arrow[from={C}, to={E}, "g_{!}"', ""{name=gd}, bend right=80pt]
\arrow[from={E}, to={C}, "f_{*}"', ""{name=fu}, bend right=80pt]
\arrow[from={Fd}, to={Gu}, "\rotatebox{-90}{$\dashv$}" description, phantom]
\arrow[from={gd}, to={Fu}, "\rotatebox{-90}{$\dashv$}" description, phantom]
\arrow[from={Gd}, to={fu}, "\rotatebox{-90}{$\dashv$}" description, phantom]
\end{tikzcd}.
\]
\end{corollary}

\begin{proof}
By uniqueness of adjoints in a $2$-category: $g^{*}$ and $f_{!}$ are both left adjoint to $f^{*}$, hence canonically isomorphic; dually for $f^{*}$ and $g_{*}$.
\end{proof}

\begin{corollary}
\label{cor:g-ast-ex}
If $f:\bfP\rightleftarrows \bfQ:g$ is a Galois connection, then $g_{*}:\vect^\bfQ\to\vect^\bfP$ is exact.
\end{corollary}

\begin{proof}
Using Corollary~\ref{cor:3_adj}, $g_{*}\cong f^{*}$, and $f^{*}$ is exact by Proposition~\ref{prop:Kan-posets}.
\end{proof}

Combining Corollary~\ref{cor:g-ast-ex} with Lemma~\ref{lem:pres-proj} yields:

\begin{proposition}\label{prop:pullback_preserves_projectives}
Let $f:\bfP\rightleftarrows \bfQ:g$ be a Galois connection. If $M\in\vect^\bfP$ is projective, then $g^{*}(M)$ is projective in $\vect^\bfQ$.\qed
\end{proposition}

\medskip

The next basic facts will be used tacitly; they are immediate from the adjunction $f\dashv g$ and we omit the proof.

\begin{lemma}\label{lem:galois-connection-basics}
Let $f:\bfP\rightleftarrows \bfQ:g$ be a Galois connection. Then:
\begin{enumerate}
\item The following are equivalent: $f$ is surjective; $g$ is injective; $f\circ g=\id_{\bfQ}$.
\item For $q\in \bfQ$,
\[
g(q)=\max\{\,x\in \bfP\mid f(x)\le q\,\}.
\]
In particular, if $f\circ g=\id_{\bfQ}$, then
\[
g(q)=\max\{\,x\in \bfP\mid f(x)=q\,\}.
\]
\end{enumerate}
\end{lemma}

%%%%%%%%%%%%%%%%%%%%%%%%%%%%%%%%%%%%%%%%%%%%%%%%%%%%%%%%%%%%%%%%%
%%%%%%%%%%%%%%%%%%%%%%%%%%%%%%%%%%%%%%%%%%%%%%%%%%%%%%%%%%%%%%%%%



\section{Galois Transport Distance}
\label{sec:galois-transport}

Fix a finite poset $\bfP$ equipped with a metric $d_{\bfP}$. Motivated by optimal transport, we compare $\bfP$-modules by \emph{transporting} them through a common “apex” poset $\bfQ$ via \emph{Galois insertions}. 

%%%%%%%%%%%%%%%%%%%%%%%%%%%%%%%%%%%%%%%%%%%%%%%%%%%%%%%%%%%%%%%%%%%
\subsection{Couplings from Galois insertions and the induced distance}
%%%%%%%%%%%%%%%%%%%%%%%%%%%%%%%%%%%%%%%%%%%%%%%%%%%%%%%%%%%%%%%%%%%

\begin{definition}[Galois coupling with insertions]
\label{def:coupling-insertion}
Let $M,N\in\vect^{\bfP}$. A \emph{Galois coupling} of $(M,N)$ consists of a poset $\bfQ$, two Galois insertions
\[
f:\bfQ\rightleftarrows \bfP:g,
\qquad
h:\bfQ\rightleftarrows \bfP:i
\quad\text{with}\quad
f\circ g=\id_{\bfP}=h\circ i,
\]
and a module $\Gamma\in\vect^{\bfQ}$ such that $g^{*}\Gamma \cong M$ and $i^{*}\Gamma \cong N$. Equivalently (Corollary~\ref{cor:3_adj}), $M\cong f_{!}\Gamma$ and $N\cong h_{!}\Gamma$.
\end{definition}


\begin{definition}
\label{def:cost-distance}
The \emph{cost} of a coupling $(\bfQ,f\dashv g,\ h\dashv i,\ \Gamma)$ is
\[
\cost(\Gamma):=\sup_{q\in\bfQ} d_{\bfP}\big(f(q),\,h(q)\big).
\]
The \emph{Galois transport distance} between $M,N\in\vect^{\bfP}$ is
\[
\dGT{M}{N}:=\inf\{\ \cost(\Gamma)\ \mid\ \Gamma\ \text{is a Galois coupling of }(M,N)\ \}.
\]
\end{definition}

Basic properties are immediate: $\dGT(M,M)=0$ (take $\bfQ=\bfP$ and the identity insertions), symmetry (swap the legs), and isomorphism invariance.

%%%%%%%%%%%%%%%%%%%%%%%%%%%%%%%%%%%%%%%%%%%%%%%%%%%%%%%%%%%%%%%%%%%
\subsection{Composition of couplings and the triangle inequality}
%%%%%%%%%%%%%%%%%%%%%%%%%%%%%%%%%%%%%%%%%%%%%%%%%%%%%%%%%%%%%%%%%%%

We now prove the triangle inequality by gluing couplings along a pullback apex. The proof is categorical: we use Kan–extension formulas, adjunctions, and objectwise pullbacks in $\vect$—no heavy base change.

\begin{theorem}[Triangle inequality]
\label{thm:triangle}
For all $M,N,O\in\vect^{\bfP}$,
\[
\dGT{M}{O}\ \le\ \dGT{M}{N}\ +\ \dGT{N}{O}.
\]
\end{theorem}

\begin{proof}
Let $(\bfQ,f\dashv g,\ h\dashv i,\ \Gamma)$ be a coupling of $(M,N)$ and $(\bfQ',f'\dashv g',\ h'\dashv i',\ \Lambda)$ a coupling of $(N,O)$, with $f\circ g=h\circ i=f'\circ g'=h'\circ i'=\id_{\bfP}$ and
\[
g^{*}\Gamma \cong M,\quad i^{*}\Gamma \cong N,\qquad
g'^{*}\Lambda \cong N,\quad i'^{*}\Lambda \cong O.
\]

\noindent\emph{Step 1 (pullback apex).}
Form the pullback in $\Pos$:
\[
\begin{tikzcd}[column sep=4.6em, row sep=3.8em]
& \Nname{Q}\bfQ \arrow[dl, "f"'] \arrow[dr, "h"] & \\
\Nname{P}\bfP & & \Nname{Ptwo}\bfP \\
& \Nname{Qp}\bfQ' \arrow[ul, "f'"'] \arrow[ur, "h'"] & \\
& \Nname{PB}{\bfQ\times_{\bfP}\bfQ'}
  \arrow[from={PB}, to={Q}, "\pi_{1}"']
  \arrow[from={PB}, to={Qp}, "\pi_{2}"]
  \arrow[from={PB}, to={P}, "F=f\circ \pi_{1}"', bend left=14]
  \arrow[from={PB}, to={Ptwo}, "H=h'\circ \pi_{2}", bend left=14]
\end{tikzcd}
\]
so that $p:=h\circ\pi_{1}=f'\circ\pi_{2}$. The maps $F$ and $H$ are the new left legs that determine cost.

\noindent\emph{Step 2 (glued mediator $\Psi$ in $\vect^{\bfQ\times_{\bfP}\bfQ'}$).}
Using the identifications $i^{*}\Gamma\cong N$ and $g'^{*}\Lambda\cong N$, together with the units/counits for $h\dashv i$ and $f'\dashv g'$, there are canonical natural transformations
\[
\Gamma\circ \pi_{1}\ \longrightarrow\ N\circ p,
\qquad
\Lambda\circ \pi_{2}\ \longrightarrow\ N\circ p.
\]
Define $\Psi$ as their objectwise pullback in $\vect$:
\[
\begin{tikzcd}[row sep=large, column sep=large]
\Psi \arrow[r] \arrow[d] & \Lambda\circ \pi_{2} \arrow[d] \\
\Gamma\circ \pi_{1} \arrow[r] & N\circ p
\end{tikzcd}
\qquad(\text{pullback in }\vect).
\]

\noindent\emph{Step 3 ($F_{!}\Psi\cong M$ and $H_{!}\Psi\cong O$).}
For $F=f\circ\pi_{1}$, consider the comma category $(F\downarrow x)$ at $x\in\bfP$. Since $f\dashv g$ and $f\circ g=\id_{\bfP}$, the object
\[
G(x)\ :=\ \big(g(x),\ g'\big(h(g(x))\big)\big)\ \in\ \bfQ\times_{\bfP}\bfQ'
\]
is terminal in $(F\downarrow x)$. By the Kan formula specialized to thin categories,
\[
(F_{!}\Psi)(x)\ \cong\ \Psi\big(G(x)\big).
\]
Evaluating the pullback square at $G(x)$ and using $g'^{*}\Lambda\cong N$ identifies the fiber product with $\Gamma\big(g(x)\big)$, and then $g^{*}\Gamma\cong M$ gives $\Gamma\big(g(x)\big)\cong M(x)$, naturally in $x$. Hence $F_{!}\Psi\cong M$. By the same argument (interchanging $(f,h,g,i)$ with $(f',h',g',i')$) we obtain $H_{!}\Psi\cong O$.

\noindent\emph{Step 4 (cost bound).}
For any $(q,q')$ in the pullback apex, the seam identity $h(\pi_{1}(q,q'))=f'(\pi_{2}(q,q'))=p(q,q')$ gives, by the metric triangle inequality on $\bfP$,
\[
d_{\bfP}\big(F(q,q'),\,H(q,q')\big)
= d_{\bfP}\big(f(q),\,h'(q')\big)
\le d_{\bfP}\big(f(q),\,h(q)\big)+d_{\bfP}\big(f'(q'),\,h'(q')\big).
\]
Taking suprema yields $\cost(\Psi)\le \cost(\Gamma)+\cost(\Lambda)$.

\noindent\emph{Conclusion.}
Given $\varepsilon>\dGT{M}{N}$ and $\delta>\dGT{N}{O}$, choose couplings of costs $<\varepsilon$ and $<\delta$; the construction above produces a coupling of $(M,O)$ with cost $<\varepsilon+\delta$. Taking infima proves the claim.
\end{proof}

\begin{remark}
On $(\mathbb{R},\le)$ with metric $|x-y|$, the transport distance agrees with the classical interleaving distance: translations supply the insertion legs, interleaving maps assemble into a coupling of the same cost, and conversely. A brief appendix records this dictionary; the main text works with finite posets.
\end{remark}

%%%%%%%%%%%%%%%%%%%%%%%%%%%%%%%%%%%%%%%%%%%%%%%%%%%%%%%%%%%%%%%%%%%
\subsection{Metric properties}
%%%%%%%%%%%%%%%%%%%%%%%%%%%%%%%%%%%%%%%%%%%%%%%%%%%%%%%%%%%%%%%%%%%

We record the basic metric behavior of the Galois transport distance. Recall that “extended” means the value $+\infty$ is allowed (when no finite‑cost coupling exists).

\begin{proposition}
\label{prop:gt-pseudometric}
For a finite poset $\bfP$ with metric $d_{\bfP}$, the function
\[
\dGT{-}{-}:\ \mathrm{Ob}(\vect^{\bfP})\times \mathrm{Ob}(\vect^{\bfP})\longrightarrow [0,+\infty]
\]
is an extended pseudometric: $\dGT{M}{M}=0$ for all $M$, $\dGT{M}{N}=\dGT{N}{M}$ for all $M,N$, and the triangle inequality of Theorem~\ref{thm:triangle} holds.
\end{proposition}

\begin{proof}
Nonnegativity is immediate from the definition of cost. The diagonal vanishes by the identity coupling $\bfQ=\bfP$, $f=h=\id_{\bfP}$, $g=i=\id_{\bfP}$, $\Gamma=M$, which has cost $0$. Symmetry holds by exchanging the two insertion legs of any coupling. The triangle inequality is Theorem~\ref{thm:triangle}.
\end{proof}

\begin{corollary}
\label{cor:gt-metric-on-iso}
On isomorphism classes, $\dGT{-}{-}$ is an extended metric: if $\dGT{M}{N}=0$ then $M\cong N$.
\end{corollary}

\begin{proof}
Since $\bfP$ is finite, the set of values $\{d_{\bfP}(x,y)\mid x,y\in\bfP\}$ is finite; consequently every coupling has cost in this finite set, and the infimum of achievable costs is a \emph{minimum}. Thus $\dGT{M}{N}=0$ implies the existence of a coupling with $\cost(\Gamma)=0$. For such a coupling we have $d_{\bfP}(f(q),h(q))=0$ for all $q\in\bfQ$, hence $f(q)=h(q)$ (metric $T_0$ property), so $f=h$ as maps $\bfQ\to\bfP$. Therefore
\[
M\ \cong\ f_{!}\Gamma\ =\ h_{!}\Gamma\ \cong\ N,
\]
which shows $M\cong N$.
\end{proof}

\begin{remark}
The “extended” qualifier is essential in general: for some pairs $(M,N)$ there may be no coupling of finite cost (hence $\dGT{M}{N}=+\infty$). All results above remain valid with $+\infty$ admitted.
\end{remark}



%%%%%%%%%%%%%%%%%%%%%%%%%%%%%%%%%%%%%%%%%%%%%%%%%%%%%%
%%%%%%%%%%%%%%%%%%%%%%%%%%%%%%%%%%%%%%%%%%%%%%%%%%%%%%
\section{Galois Transport Distance}
%%%%%%%%%%%%%%%%%%%%%%%%%%%%%%%%%%%%%%%%%%%%%%%%%%%%%%
%%%%%%%%%%%%%%%%%%%%%%%%%%%%%%%%%%%%%%%%%%%%%%%%%%%%%%

Let $\bfP$ be a finite poset with a metric $d_\bfP : \bfP \times \bfP \to \mathbb{R}_{\ge 0}$.

%%%%%%%%%%%%%%%%%%%%%%%%%%%%%%%%%%%%%%%%%%%%%%%%%%%%%%%%%
\subsection{Transports and their composition}

\begin{definition}[Galois coupling / transport]\label{def:galois-coupling}
A \emph{Galois coupling} between $M,N\in\vect^\bfP$ consists of Galois insertions
\[
f:\bfX\rightleftarrows \bfP:g,\qquad h:\bfX\rightleftarrows \bfP:i,
\]
a functor $\Gamma\in\vect^\bfX$, and isomorphisms
\[
\alpha: g^\ast\Gamma \xrightarrow{\cong} M,\qquad
\beta: i^\ast\Gamma \xrightarrow{\cong} N.
\]
\[
\begin{tikzcd}
& \bfX \ar[dd,"\Gamma"] \ar[dr, bend left, "h"] \ar[dl, bend right, "f"'] & \\
\bfP \ar[rd,"M"'] \ar[ru, bend right, "g"'] && \bfP \ar[ld,"N"] \ar[lu, bend left, "i"] \\
& \vect &
\end{tikzcd}
\]
\end{definition}

\subsection*{Composition as a pullback in $\vect^\bfZ$.}
Given Galois couplings
\[
(\bfX,f\dashv g,\ h\dashv i,\ \Gamma,\alpha,\beta)\text{ for }(M,N)
\quad\text{and}\quad
(\bfY,f'\dashv g',\ h'\dashv i',\ \Lambda,\alpha',\beta')\text{ for }(N,O),
\]
consider the pull back of $h$ and $f'$ in the category $\mathbf{Pos}$ of posets:
\[
\begin{tikzcd}[column sep=large,row sep=large]
\bfZ \ar[r,"\pi_\bfY"] \ar[d,"\pi_\bfX"'] \ar[dr,phantom,"\lrcorner",very near start]
  & \bfY \ar[d,"f'"] \\
\bfX \ar[r,"h"'] & \bfP
\end{tikzcd},
\]
and set $p:=h\circ\pi_\bfX=f'\circ\pi_\bfY:\bfZ\to \bfP$.

\noindent\textbf{Cospan in $\vect^\bfZ$ (construction of $\sigma,\tau$).}
From the units $\eta^h:\mathrm{id}_\bfX\Rightarrow i\circ h$ and $\eta^{f'}:\mathrm{id}_\bfY\Rightarrow g'\circ f'$,
whisker along the projections to obtain
\[
\pi_\bfX \Rightarrow i\circ p \quad\text{and}\quad \pi_\bfY \Rightarrow g'\circ p,
\qquad p=h\circ\pi_\bfX=f'\circ\pi_\bfY.
\]
Postcompose with $\Gamma$ and $\Lambda$, then use the seam isomorphisms
$\beta:i^\ast\Gamma\cong N$ and $\alpha':g'^{\ast}\Lambda\cong N$ to get
\[
\sigma:\Gamma\circ\pi_\bfX \Rightarrow N\circ p,\qquad
\tau:\Lambda\circ\pi_\bfY \Rightarrow N\circ p,
\]
as the composites
\[
\begin{tikzcd}[column sep=huge,row sep=large]
\Gamma\circ\pi_\bfX
  \ar[r, "{\Gamma(\pi_\bfX \Rightarrow i\circ p)}"]
& \Gamma\circ i\circ p
  \ar[r, "="]
& (i^\ast\Gamma)\circ p
  \ar[r, "{\beta\circ p}", "\cong"']
& N\circ p
\end{tikzcd}
\]
\[
\begin{tikzcd}[column sep=huge,row sep=large]
\Lambda\circ\pi_\bfY
  \ar[r, "{\Lambda(\pi_\bfY \Rightarrow g'\circ p)}"]
& \Lambda\circ g'\circ p
  \ar[r, "="]
& (g'^\ast\Lambda)\circ p
  \ar[r, "{\alpha'\circ p}", "\cong"']
& N\circ p
\end{tikzcd}
\]
Thus, we have the cospan in $\vect^\bfZ$:
\[
\begin{tikzcd}[column sep=huge]
\Gamma\circ\pi_\bfX \ar[dr, "{\sigma}"'] && \Lambda\circ\pi_\bfY \ar[dl, "{\tau}"] \\[-2pt]
& N\circ p &
\end{tikzcd}
\]

\begin{definition}[Composite apex]\label{def:composite-apex}
Define $\Psi\in\vect^\bfZ$ as the pullback of this cospan:
\[
\begin{tikzcd}[column sep=large,row sep=large]
\Psi \ar[r,"\pi_\Lambda"] \ar[d,"\pi_\Gamma"'] \ar[dr,phantom,"\lrcorner",very near start]
  & \Lambda\circ\pi_\bfY \ar[d,"\tau"] \\
\Gamma\circ\pi_\bfX \ar[r,"\sigma"'] & N\circ p
\end{tikzcd}
\]
so $\sigma\circ\pi_\Gamma=\tau\circ\pi_\Lambda$.
\end{definition}

\noindent\textbf{Legs and adjoints.}
With projections $\pi_\bfX,\pi_\bfY:\bfZ\to \bfX,\bfY$, define
\[
\iota_\bfX:\bfX\to \bfZ,\ \ \iota_\bfX(x)=(x,g'(h(x))),\qquad
\iota_\bfY:\bfY\to \bfZ,\ \ \iota_\bfY(y)=(i(f'(y)),y).
\]
Then $\pi_\bfX\dashv\iota_\bfX$ and $\pi_\bfY\dashv\iota_\bfY$, and
\[
F:=f\circ\pi_\bfX \dashv G:=\iota_\bfX\circ g,\qquad
H:=h'\circ\pi_\bfY \dashv I:=\iota_\bfY\circ i',
\]
are Galois insertions ($f\circ g=\id_\bfP$, $h'\circ i'=\id_\bfP$, and $\id_\bfZ\le\iota_\bfX\pi_\bfX$, $\id_\bfZ\le\iota_\bfY\pi_\bfY$).

\medskip
\noindent\emph{Schematics.}
\[
\begin{tikzcd}
    && \bfZ \ar[dr, bend left, "\pi_\bfY"] \ar[dl, bend right, "\pi_\bfX"']  && \\
    & \bfX \ar[dd, "\Gamma", near end] \ar[ur, bend right, "\iota_\bfX"']  
        \ar[dl, bend right, "f"'] \ar[dr, bend left, "h"] & & 
      \bfY \ar[dd, "\Lambda", near end] \ar[ul, bend left, "\iota_\bfY"] 
        \ar[dl, bend right, "f'"'] \ar[dr, bend left, "h'"] & \\
    \bfP \ar[rd, "M"] \ar[ur, bend right, "g"'] && 
    \bfP \ar[ld, "N"] \ar[rd, "N"] \ar[ul, bend left, "i"] 
      \ar[ur, bend right, "g'"'] && 
    \bfP \ar[ld, "O"] \ar[ul, bend left, "i'"] \\
    & \vect & & \vect &
\end{tikzcd}
\]
\vspace{-0.5em}
\[
\begin{tikzcd}
& \bfZ \ar[dd,"\Psi"] \ar[dr, bend left, "H"] \ar[dl, bend right, "F"'] & \\
\bfP \ar[rd,"M"'] \ar[ru, bend right, "G"'] && \bfP \ar[ld,"O"] \ar[lu, bend left, "I"] \\
& \vect &
\end{tikzcd}
\]

\noindent\textbf{Recovery of $M$ and $O$.}
Precomposition preserves limits, so pulling back the defining cospan of $\Psi$ along $G$ and $I$ gives pullback squares in $\vect^\bfP$. In each, one vertical leg is an isomorphism, hence the opposite corner identifies with the pullback.

\medskip
\noindent\emph{Along $G:\bfP\to \bfZ$:}
\[
\begin{tikzcd}[column sep=huge, row sep=large]
G^\ast\Psi \ar[r] \ar[d] \ar[dr,phantom,"\lrcorner",very near start]
  & \Lambda\circ g'\circ h\circ g \ar[d, "{G^\ast\tau\ (\cong)}"'] \\
\Gamma\circ g \ar[r, "{G^\ast\sigma}"'] & N\circ h\circ g
\end{tikzcd}
\]
Thus $G^\ast\Psi\cong \Gamma\circ g \xrightarrow[\alpha]{\cong} M$.

\medskip
\noindent\emph{Along $I:\bfP\to \bfZ$:}
\[
\begin{tikzcd}[column sep=huge, row sep=large]
I^\ast\Psi \ar[r] \ar[d] \ar[dr,phantom,"\lrcorner",very near start]
  & \Lambda\circ i' \ar[d, "{I^\ast\tau}"] \\
\Gamma\circ i\circ f'\circ i' \ar[r, "{I^\ast\sigma\ (\cong)}"'] & N\circ f'\circ i'
\end{tikzcd}
\]
Hence $I^\ast\Psi\cong \Lambda\circ i' \xrightarrow[\beta']{\cong} O$.


%%%%%%%%%%%%%%%%%%%%%%%%%%%%%%%%%%%%%%%%%%%%%%%%%%%%%%%%%%%
\subsection{Cost and the Galois transport distance}
%%%%%%%%%%%%%%%%%%%%%%%%%%%%%%%%%%%%%%%%%%%%%%%%%%%%%%%%%%%

\begin{definition}[Cost]\label{def:coupling-cost}
For a Galois coupling $(\bfX,f\dashv g,h\dashv i,\Gamma)$, we set
\[
\operatorname{cost}(\Gamma):=\max_{x\in \bfX} d_\bfP\big(f(x),h(x)\big).
\]
\end{definition}

\begin{definition}[Galois transport distance]\label{def:GaloisTransport}
For $M,N\in\vect^\bfP$, we define
\[
\dGT(M,N):=\inf\{\operatorname{cost}(\Gamma)\mid (\bfX,f\dashv g,h\dashv i,\Gamma)\ \text{a coupling for }(M,N)\},
\]
with the convention that $\dGT(M,N)=\infty$ if no Galois coupling exists.
\end{definition}

\begin{lemma}\label{lem:GT-basic}
For all $M,N\in\vect^\bfP$, we have the following:
\begin{enumerate}%[label=\textup{(\alph*)}]
\item $\dGT(M,N)=\dGT(N,M)$, \quad
\item If $M\cong N$ then $\dGT(M,N)=0$.
\end{enumerate}
\end{lemma}

\begin{proof}
(1) Swap $(f,g)$ with $(h,i)$ in any coupling. (2) Take $\bfX=\bfP$, $g=i=\id_\bfP$, $f=h=\id_\bfP$, and $\Gamma=M\cong N$.
\end{proof}

\begin{proposition}[Finiteness under agreement at the top]\label{prop:finiteness_infty}
If $\bfP$ has a top element $\infty$ and $M(\infty)\cong N(\infty)$, then $\dGT(M,N)<\infty$.
\end{proposition}

\begin{proof}
Use the standard two-wing construction $\bfX=\bfP_L\sqcup_\infty \bfP_R$, map
$g(a)=a_L$, $i(a)=a_R$, $f(a_L)=a$, $f(a_R)=\infty$, $h(a_R)=a$, $h(a_L)=\infty$,
and glue $\Gamma$ along $M(\infty)\cong N(\infty)$. Then 
$\operatorname{cost}(\Gamma)=\max_{a\in \bfP} d_\bfP(a,\infty)<\infty$.
\end{proof}

\begin{lemma}[Subadditivity under composition]\label{lem:cost-subadd}
If $(\bfX,f\dashv g,h\dashv i,\Gamma)$ is a coupling for $(M,N)$ and
$(\bfY,f'\dashv g',h'\dashv i',\Lambda)$ a coupling for $(N,O)$, and
$(\bfZ,F\dashv G,H\dashv I,\Psi)$ is their composite, then
\[
\operatorname{cost}(\Psi)\ \le\ \operatorname{cost}(\Gamma)+\operatorname{cost}(\Lambda).
\]
\end{lemma}

\begin{proof}
For $z=(x,y)\in \bfZ$ the seam gives $h(x)=f'(y)$, so
\[
d_\bfP\big(F(z),H(z)\big)=d_\bfP\big(f(x),h'(y)\big)
\ \le\ d_\bfP\big(f(x),h(x)\big)+d_\bfP\big(f'(y),h'(y)\big).
\]
Take maxima.
\end{proof}

\begin{proposition}[Triangle inequality]\label{prop:triangle_ineq}
For all $M,N,O\in\vect^\bfP$,
\[
\dGT(M,O)\ \le\ \dGT(M,N)+\dGT(N,O).
\]
\end{proposition}

\begin{proof}
Approximate the infima within $\varepsilon>0$, compose, and apply Lemma~\ref{lem:cost-subadd}; let $\varepsilon\to 0$.
\end{proof}

\begin{corollary}[Extended metric on isomorphism classes]\label{cor:extended-metric}
$\GT$ is an extended pseudo-metric on $\vect^\bfP$ and descends to an extended metric on isomorphism classes.
\end{corollary}

\begin{proof}
Nonnegativity is tautological; symmetry is Lemma~\ref{lem:GT-basic}(a);
the triangle inequality is Proposition~\ref{prop:triangle_ineq}.
If $M\cong N$ then $\dGT(M,N)=0$ by Lemma~\ref{lem:GT-basic}(b).
\end{proof}

%%%%%%%%%%%%%%%%%%%%%%%%%%%%%%%%%%%%%%%%%%%%%%%%%%%%%%%%%%%
\section{Interleaving Distance}
\label{sec:intlv-dist}
%%%%%%%%%%%%%%%%%%%%%%%%%%%%%%%%%%%%%%%%%%%%%%%%%%%%%%%%%%%

The Galois transport distance can be viewed as an adaptation of the classical interleaving distance from persistence theory. In this paper we work with modules over a finite poset $\bfP$, whereas classical persistence uses the totally ordered real line $(\mathbb{R},\le)$. Although persistence modules over $\mathbb{R}$ are not required to be finite, in practice one often restricts attention to a finite totally ordered subposet (e.g.\ determined by selected parameter values or critical times) and extends from there. This places interleavings and Galois transports in a common finite setting for comparison.


\subsection*{Definition}

We recall the standard categorical definition using the shift endofunctor
(see \cite{ChazalCohenSteinerGlisseGuibasOudot2009,Lesnick2015}).

\begin{definition}[Interleaving distance]\label{def:interleaving-distance}
Let $\bfP=(\mathbb{R},\le)$. For $\varepsilon\ge 0$, set
$T_\varepsilon:\vect^\bbR \to\vect^\bbR$ by
$(T_\varepsilon M)(r)=M(r+\varepsilon)$. An \emph{$\varepsilon$-interleaving}
between $M,N\in \vect^\bbR$ is a pair of natural transformations
$\varphi:M\Rightarrow T_\varepsilon N$ and $\psi:N\Rightarrow T_\varepsilon M$
such that the composites $M\to T_{2\varepsilon}M$ and
$N\to T_{2\varepsilon}N$ equal the canonical structure maps.  The
\emph{interleaving distance} is
\[
d_I(M,N):=\inf\{\ \varepsilon\ge 0\mid M,N\ \text{are $\varepsilon$-interleaved}\ \}.
\]
\end{definition}

\subsection*{Two basic facts}

\begin{proposition}[Tail agreement implies finiteness]\label{prop:tail-agree-vanish}
If $M,N \in \vect^\bbR$ satisfy $M(r)=N(r)=0$ for all $r\le b$ and
$M|_{[R,\infty)}\cong N|_{[R,\infty)}$, then $d_I(M,N)\le R-b$.
\end{proposition}

\begin{proof}
Let $\varepsilon=R-b$.  For $r>b$, define $\varphi_r$ by following
$M(r)\to M(r+\varepsilon)\xrightarrow{\theta_{r+\varepsilon}}N(r+\varepsilon)$,
and symmetrically define $\psi_r$.  For $r\le b$ both $M(r)$ and $N(r)$ vanish,
so the maps are zero.  One checks that these assignments are natural, and that
the interleaving equations hold: beyond $R$ the isomorphism $\theta$ cancels,
and before $b$ everything is zero.  Hence $d_I(M,N)\le\varepsilon$.
\end{proof}

\begin{proposition}[Pseudo-metric and separation]\label{prop:dI-pseudometric}
For $M,N,O \in \vect^\bbR$,
\[
d_I(M,N)=d_I(N,M),\qquad
d_I(M,O)\le d_I(M,N)+d_I(N,O),\qquad
d_I(M,M)=0.
\]
Moreover, $d_I(M,N)=0$ iff $M\cong N$. Hence $d_I$ is an extended
metric on isomorphism classes.
\end{proposition}

\begin{proof}
Symmetry is clear by swapping $\varphi,\psi$.  
For the triangle inequality, if $M,N$ are $\varepsilon$-interleaved and $N,O$ are $\delta$-interleaved, then by pasting the data one obtains an $(\varepsilon+\delta)$-interleaving of $M,O$; the details are a standard exercise.  
Finally, $d_I(M,M)=0$ via the identity maps.  
If $d_I(M,N)=0$ then a $0$-interleaving exists, giving mutually inverse isomorphisms $M\cong N$.
\end{proof}

\subsection*{Relating interleavings and Galois couplings on $\mathbb{R}$}

For this subsection fix $\bfP=(\mathbb{R},\le)$ with metric $d(x,y)=|x-y|$.
For $\varepsilon\ge 0$, the translations
$t_\varepsilon(x)=x+\varepsilon$ and $r_\varepsilon(x)=x-\varepsilon$
satisfy $t_\varepsilon\dashv r_\varepsilon$, so $t_\varepsilon\dashv r_\varepsilon$ is a Galois insertion.

\begin{proposition}[Interleaving $\Rightarrow$ coupling]\label{prop:int-to-galois-R}
If $M,N\in \vect^\bbR$ are $\varepsilon$-interleaved, there exists a
Galois coupling $(\bfX,f\dashv g,\ h\dashv i,\ \Gamma)$ for $(M,N)$ with
$\bfX=\mathbb{R}_L\sqcup\mathbb{R}_R$ and $\operatorname{cost}(\Gamma)=\varepsilon$.
\end{proposition}

\begin{proof}
Take two copies of $\mathbb{R}$, ordered so that cross-inequalities encode the shift:
$a_L\le b_R$ iff $a+\varepsilon\le b$, and symmetrically for $R$ to $L$.
Let $f,h:\bfX\to\mathbb{R}$ be the evident projections, with $f(a_L)=a$ and $f(a_R)=a+\varepsilon$, etc.
Define $\Gamma$ by $\Gamma(a_L)=M(a)$ and $\Gamma(a_R)=N(a)$, using the interleaving maps to define $\Gamma$ on cross arrows.  
By construction $g^\ast\Gamma\cong M$, $i^\ast\Gamma\cong N$, and the cost is $\sup|f(x)-h(x)|=\varepsilon$.  
Functoriality of $\Gamma$ on mixed zigzags is ensured by the interleaving identities; the reader can check this directly.
\end{proof}

\begin{proposition}[Coupling $\Rightarrow$ interleaving]\label{prop:coupling-implies-int-R}
Let $(\bfX,f\dashv g,\ h\dashv i,\ \Gamma)$ be a Galois coupling for
$(M,N)\in \vect^\bbR$ with $\sup_{x\in \bfX}|f(x)-h(x)|\le\varepsilon$.
Then $M$ and $N$ are $\varepsilon$-interleaved.
\end{proposition}

\begin{proof}
Fix $r\in\mathbb{R}$.  Using $x=g(r)$, we have $f(x)=r$ and $h(x)\le r+\varepsilon$.
Define $\varphi_r:M(r)\to N(r+\varepsilon)$ by
\[
M(r)\xrightarrow[\cong]{\alpha_r}\Gamma(x)\xrightarrow{\Gamma(\eta^h_x)}\Gamma(i(h(x)))
\xrightarrow[\cong]{\beta_{h(x)}}N(h(x))\to N(r+\varepsilon).
\]
Here $\eta^h$ is the unit of $h\dashv i$.  Similarly, using $y=i(r)$ one defines
\[
\psi_r:\ N(r)\to M(r+\varepsilon).
\]
Naturality follows from functoriality of $\Gamma$ and the seam maps.  
For the interleaving identities, one composes $\varphi$ and $\psi$ and uses the triangle identities of the adjunctions, together with the cost condition ensuring $h(g(r))\le r+\varepsilon$, etc.  
This reduces exactly to the canonical structure maps $M(r)\to M(r+2\varepsilon)$ and $N(r)\to N(r+2\varepsilon)$.  
Routine but slightly tedious diagram chases are left to the reader.
\end{proof}

\subsection*{Equality of distances on $\mathbb{R}$}

\begin{theorem}\label{thm:GT-equals-dI-R}
For all $M,N\in \vect^\bbR$, the Galois transport distance equals
the interleaving distance:
\[
\dGT(M,N)=d_I(M,N).
\]
\end{theorem}

\begin{proof}
By Proposition~\ref{prop:int-to-galois-R}, any $\varepsilon$-interleaving yields a coupling of cost $\varepsilon$, so $\GT\le d_I$.  
Conversely, by Proposition~\ref{prop:coupling-implies-int-R}, any coupling of cost $\varepsilon$ yields an $\varepsilon$-interleaving, so $d_I\le \GT$.  
Thus the two distances coincide.
\end{proof}




\section{Botleneck distance of minimal projective resolutions}

We use the following simple conventions for this section. For $n\in\bbN$ write $[n]=\{1,\dots,n\}$. If $X \cong \bigoplus_{i=1}^{n} X_i$ is a finite direct sum of indecomposables, write $\Summands(X)=\{X_i \mid i\in [n]\}$. For finite sets $A$ and $B$ with the same cardinality, $\Bij(A,B)$ denotes the set of bijections $A\to B$.



%Let $A$ be a finite-dimensional algebra over the field $\Bbbk$.

\begin{notation}
We choose an indecomposable decomposition $M = \Ds_{i=1}^n M_i$ of $M$ once for all $M \in \vect^\bfP$,
and set $|M|:= n$, and $\Summands(M):= \{M_i \mid i \in [n]\}$, which are called the \emph{size} and the \emph{summand set} of $M$, respectively.

Note that $|M|$ is an invariant of $M$ up to isomorphism,
and that the elements of $\Summands(M)$ are pairwise distinct.
\begin{enumerate}
\item
We denote a minimal projective resolution of $M$
by $P \down^M$, which is uniquely determined up to isomorphism of exact sequences.

\item
Denote by $\calP(M)$ the set of all projective resolutions of $M$.

\item
For each $P\down = (P_i, \partial_i)_{i\ge 0} \in \calP(M)$,
we set $|P\down|:= (|P_i|)_{i \ge 0}$, and call it the \emph{size vector} of $P\down$.

\item
Let $(\bfP, d_\bfP)$ be a finite metric poset\footnote{%%%%%%%%%%%%%%%%%%
For example, in the classical case that $\bfP$ is a totally ordered set $\bfP:= \{1 < 2 < \cdots < n\}$ for some $n \ge 1$, we can define
$d_\bfP(x, y):= |x - y|$ for all $x, y \in \bfP$.
}.
Then for any indecomposable projective $\bfP$-modules
$P$ and $Q$, we set the \emph{distance} of $P$ and $Q$ to be
$$
\dist(P, Q):= d_\bfP(x, y),
$$
where $P \iso \bfP(x, \blank)$ and
$Q \iso \bfP(y, \blank)$ with $x, y \in \bfP$.



\end{enumerate}
\end{notation}

\begin{lemma}
\label{lem:all-prj-resol}
The following holds.
$$
\calP(M) =\{P\down^M \ds \Ds_{i \in [n]} \Cone(\id_{P_i})[a_i]
\mid \exists n \in \bbN, \forall i \in [n], P_i \in \prj \bfP, a_i \in \bbN\}.
$$
\end{lemma}

The following definition is crucial for our stability theorem.

\begin{definition}
Let $M, N \in \vect^\bfP$. Then 
\begin{enumerate}
\item
We set
$$
\calP(P\down^M, P\down^N):=
\{(P\down, Q\down) \in \calP(M) \times \calP(N) \mid
|P\down| = |Q\down|\}.
$$

\item
For each $(P\down, Q\down) \in \calP(P\down^M, P\down^N)$, we set
$$
\Match(P\down, Q\down):= \prod_{i\ge 0}
\Bij(\Summands(P_{i}(M)), \Summands(Q_{i}(N))),
$$
the element of which is called a \emph{matching} of $P\down$ and $Q\down$.

\item
For each matching $b = (b_i)_{i\ge 0}$ of
$P\down$ and $Q\down$, we define its \emph{cost}
$\cost(b)$ by
$$
\cost(b):= \sup\{\dist(P, b_i(P)) \mid i \ge 0,
P \in \Summands(P_i^M)\}.
$$

\item
We define the \emph{regular distance}
of $P\down$ and $Q\down$ by
$$
\dist_R(P\down, Q\down):=
\inf \{\cost(b) \mid b \in \Match(P\down, Q\down)\}.
$$

\item
Finally, we define the \emph{bottle neck distance}
of $P\down^M$ and $P\down^N$ by
$$
\dist_B(P\down^M, P\down^N):=
\inf \{\dist_R(P\down, Q\down) \mid 
(P\down, Q\down) \in \calP(P\down^M, P\down^N)\}.
$$
\end{enumerate}
\end{definition}

\begin{remark}
By Lemma \ref{lem:all-prj-resol}, we see that
$\calP(P\down^M, P\down^N) \ne \emptyset$
for all $M, N \in \vect^\bfP$.
\end{remark}

\section{Stability of minimal projective resolutions}

\begin{lemma}
\label{lem:rt-adj-prj-ind}



If $f : \bfX \leftrightarrows \bfP : g$ is a Galois connection of posets $\bfX$ and $\bfP$,
then we have $g^\ast \bfX(x,\blank) \iso \bfP(f(x), \blank)$
in $\vect^\bfP$ for all $x \in \bfX$.
\end{lemma}

\begin{proof}
There exist isomorphisms
$$
(g^\ast\bfX(x,\blank))(y) \iso \bfX(x,g(y)) \iso \bfP(f(x), y)
$$
natural in $y \in \bfP$.
\end{proof}

We now prove the stability theorem for
minimal projective resolutions.

\begin{theorem}
Let $(\bfP, d_\bfP)$ be a finite metric poset.
Then for any $M, N \in \vect^\bfP$ we have
$$
\dist_B(P\down^M, P\down^N) \le \dGT(M, N).
$$
\end{theorem}

\begin{proof}
Let $e:= \dGT(M,N)$ and $\ep > 0$ any positive real number.
Then there exists some Galois coupling $(\bfX, f \dashv g, h \dashv i, \Ga)$
of $M$ and $N$
% with $\al \colon g^\ast \Ga \isoto M$
% and $\be \colon i^\ast \Ga \isoto N$ isomorphisms 
such that $e \le \cost(\Ga) \le e + \ep$.
It is enough to construct some
$(P\down, Q\down) \in \calP(P\down^M, P\down^N)$
such that $\dist_R(P\down, Q\down) \le \cost(\Ga)$.
Indeed, when this is the case,
we have $\dist_B(P\down^M, P\down^N) \le \dist_R(P\down, Q\down) \le e + \ep$,
which shows that $\dist_B(P\down^M, P\down^N) \le e$.
% \todoilgreen{Here give such a construcion from
% the $\ep$-interleaving diagram of $M$ and $N$ as
% in the notes of 2025-07-30.}

Let $R\down = (R_i, \partial_i)_{i \ge 0}$ be a projective resolution of $\Ga$.
Then since $g^\ast$ (resp.\ $i^\ast$) is an exact functor preserving projectivity,
$P\down:= (g^\ast R_i, g^\ast \partial_i)_{i \ge 0}$
(resp.\ $Q\down:= (i^\ast R_i, i^\ast \partial_i)_{i \ge 0}$) is a projective
resolution of $g^\ast \Ga \iso M$ (resp. $i^\ast \Ga \iso N$).
We first show that $(P\down, Q\down) \in \calP(P\down^M, P\down^N)$,
equivalently that $|P\down| = |Q\down|$.
For each $i \ge 0$, there exists a sequence $(x_{i,1}, \dots, x_{i,{n_i}})$ of elements of $\bfX$ (we allow $x_{i,j} = x_{i,k}$ for some $j \ne k$)
with $n_i = |R_i|$
such that $R_i \iso \Ds_{j=1}^{n_i} \bfX(x_{i,j}, \blank)$
by Lemma \ref{lem:ind-proj}.
Then we have indecomposable decompositions
\begin{equation}
\label{eq:ind-dec-P,Q}
P_i \iso \Ds_{j=1}^{n_i} \bfP(f(x_{i,j}), \blank),\ 
Q_i \iso \Ds_{j=1}^{n_i} \bfP(h(x_{i,j}), \blank)
\end{equation}
by Lemma \ref{lem:rt-adj-prj-ind}.
Thus $|P_i| = n_i = |Q_i|$ for all $i \ge 0$.
Hence $|P\down| = |Q\down|$.

By using \eqref{eq:ind-dec-P,Q}, we can define a matching
$b \in \Match(P\down, Q\down)$ by
$b_i(\bfP(f(x_{i,j}),\blank))$\linebreak[3]$:= \bfP(h(x_{i,j}),\blank)$ for all $i \ge 0$, $j = 1, \dots, n_i$.
Then we have
$$
\begin{aligned}
\dist_R(P\down, Q\down) &\le \cost(b)\\
&= \sup\{\dist(\bfP(f(x_{i,j}),\blank), \bfP(h(x_{i,j}),\blank)) \mid i \ge 0, j=1,\dots,n_i\}\\
&= \sup\{d_\bfP(f(x_{i,j}), h(x_{i,j})) \mid i \ge 0, j=1,\dots,n_i\}\\
&\le \sup\{d_\bfP(f(x), h(x)) \mid x \in \bfX\} = \cost(\Ga),
\end{aligned}
$$
as desired.
\end{proof}



%\section{Applications}


%\section*{Acknowledgement}


\bibliographystyle{plain}

\begin{thebibliography}{10}
\bibitem{MR4402576}
Hideto Asashiba, Micka\"{e}l Buchet, Emerson~G. Escolar, Ken Nakashima, and Michio Yoshiwaki.
\newblock On interval decomposability of 2{D} persistence modules.
\newblock {\em Computational Geometry}, 105/106:Paper No. 101879, 33, 2022.
\newblock \url{https://doi.org/10.1016/j.comgeo.2022.101879}.

\bibitem{ASASHIBA2023107397}
Hideto Asashiba, Emerson~G. Escolar, Ken Nakashima, and Michio Yoshiwaki.
\newblock Approximation by interval-decomposables and interval resolutions of persistence modules.
\newblock {\em Journal of Pure and Applied Algebra}, 227(10):107397, 2023.
\newblock \url{https://doi.org/10.1016/j.jpaa.2023.107397}.

\bibitem{ASASHIBA2023100007}
Hideto Asashiba, Emerson~G. Escolar, Ken Nakashima, and Michio Yoshiwaki.
\newblock On approximation of 2{D} persistence modules by interval-decomposables.
\newblock {\em Journal of Computational Algebra}, 6--7:100007, 2023.
\newblock \url{https://doi.org/10.1016/j.jaca.2023.100007}.

\bibitem{aoki2024bipath}
Toshitaka Aoki, Emerson~G. Escolar, and Shunsuke Tada.
\newblock Bipath persistence.
\newblock {\em arXiv preprint}, arXiv:2404.02536, 2024.
\newblock \url{https://doi.org/10.48550/arXiv.2404.02536}.

\bibitem{aoki2023summand}
Toshitaka Aoki, Emerson~G. Escolar, and Shunsuke Tada.
\newblock Summand-injectivity of interval covers and monotonicity of interval resolution global dimensions.
\newblock {\em arXiv preprint}, arXiv:2308.14979, 2023.
\newblock \url{https://doi.org/10.48550/arXiv.2308.14979}.

\bibitem{asashiba2024interval}
Hideto Asashiba, Etienne Gauthier, and Enhao Liu.
\newblock Interval replacements of persistence modules.
\newblock {\em arXiv preprint}, arXiv:2403.08308, 2024.
\newblock \url{https://doi.org/10.48550/arXiv.2403.08308}.

\bibitem{Asashiba2017}
Hideto Asashiba, Ken Nakashima, and Michio Yoshiwaki.
\newblock Decomposition theory of modules: the case of Kronecker algebra.
\newblock {\em Japan Journal of Industrial and Applied Mathematics}, 34(2):489--507, Aug 2017.
\newblock \url{https://doi.org/10.1007/s13160-017-0247-y}.

\bibitem{MR4091895}
Ulrich Bauer, Magnus~B. Botnan, Steffen Oppermann, and Johan Steen.
\newblock Cotorsion torsion triples and the representation theory of filtered hierarchical clustering.
\newblock {\em Advances in Mathematics}, 369:107171, 51, 2020.
\newblock \url{https://doi.org/10.1016/j.aim.2020.107171}.

\bibitem{BBH2024approximations}
Benjamin Blanchette, Thomas Br{\"u}stle, and Eric~J. Hanson.
\newblock Homological approximations in persistence theory.
\newblock {\em Canadian Journal of Mathematics}, 76(1):66--103, 2024.
\newblock \url{https://doi.org/10.4153/s0008414x22000657}.

\bibitem{dey2022fast}
Tamal~K Dey and Tao Hou.
\newblock Fast computation of zigzag persistence.
\newblock In {\em 30th Annual European Symposium on Algorithms (ESA 2022)}. Schloss Dagstuhl-Leibniz-Zentrum f{\"u}r Informatik, 2022.
\newblock \url{https://doi.org/10.4230/LIPIcs.ESA.2022.43}.

\bibitem{dey2023computing}
Tamal~K Dey, Woojin Kim, and Facundo M{\'e}moli.
\newblock Computing generalized rank invariant for 2-parameter persistence modules via zigzag persistence and its applications.
\newblock {\em Discrete {$\&$} Computational Geometry}, pages 1--28, 2023.
\newblock \url{https://doi.org/10.1007/s00454-023-00584-z}.

\bibitem{MR3824276}
Tamal~K. Dey and Cheng Xin.
\newblock Computing bottleneck distance for 2-{D} interval decomposable modules.
\newblock In {\em 34th {I}nternational {S}ymposium on {C}omputational {G}eometry}, volume~99 of {\em LIPIcs. Leibniz Int. Proc. Inform.}, pages Art. No. 32, 15. Schloss Dagstuhl. Leibniz-Zent. Inform., Wadern, 2018.
\newblock \url{https://doi.org/10.4230/LIPIcs.SoCG.2018.32}.

\bibitem{edelsbrunner2010computational}
Herbert Edelsbrunner and John L. Harer.
{\em Computational {T}opology: {A}n {I}ntroduction}. 
American Mathematical Society, Providence, 2010.
\newblock \url{https://doi.org/10.1090/mbk/069}.

\bibitem{MR1949898}
Herbert Edelsbrunner, David Letscher, and Afra Zomorodian.
\newblock Topological persistence and simplification.
\newblock {\em Discrete {$\&$} Computational Geometry}, 28(4):511--533, 2002.
\newblock \url{https://doi.org/10.1007/s00454-002-2885-2}.

\bibitem{gabriel2006auslander}
Peter Gabriel.
\newblock Auslander-Reiten sequences and representation-finite algebras.
\newblock In {\em Representation Theory I: Proceedings of the Workshop on the Present Trends in Representation Theory, Ottawa, Carleton University, August 13--18, 1979}, pages 1--71. Springer, 2006.
\newblock \url{https://doi.org/10.1007/BFb0089778}.

\bibitem{hiraoka2023refinement}
Yasuaki Hiraoka, Ken Nakashima, Ippei Obayashi, and Chenguang Xu.
\newblock Refinement of interval approximations for fully commutative quivers.
\newblock {\em arXiv preprint arXiv:2310.03649}, 2023.
\newblock \url{https://doi.org/10.48550/arXiv.2310.03649}.

\bibitem{JY}
Johnson; Donald Yau:
{\it 2-dimensional categories}%%%%%%% give a complete one

\bibitem{kim2021generalized}
Woojin Kim and Facundo M{\'e}moli.
\newblock Generalized persistence diagrams for persistence modules over posets.
\newblock {\em Journal of Applied and Computational Topology}, 5(4):533--581, 2021.
\newblock \url{https://doi.org/10.1007/s41468-021-00075-1}.

\bibitem{MR3728284}
James R. Munkres.
{\em Topology} second edition. 
Prentice Hall, Inc., Upper Saddle River, NJ, 2000.

\bibitem{MR2121296}
Afra Zomorodian and Gunnar Carlsson.
\newblock Computing persistent homology.
\newblock {\em Discrete $\&$ Computational Geometry}, 33(2):249--274, 2005.
\newblock \url{https://doi.org/10.1007/s00454-004-1146-y}.

\bibitem[CCSG+09]{ChazalCohenSteinerGlisseGuibasOudot2009} Fr{\'e}d{\'e}ric Chazal, David Cohen--Steiner, Marc Glisse, Leonidas J. Guibas, and Steve Y.
Oudot. {\it Proximity of persistence modules and their diagrams}. In Proceedings of the
25th Annual Symposium on Computational Geometry (SoCG), pages 237--246. ACM,
2009.

\bibitem[Cur18]{curry2018}
Justin Michael Curry: {\it Dualities between cellular sheaves and cosheaves}. Journal of
Pure and Applied Algebra, 222(4):966--993, 2018.

\bibitem[EP24]{ElchesenPatel}
Alex Elchesen and Amit Patel:
{\it A categorical approach to m{\"o}bius inversion via de-
rived functors}, 2024.

\bibitem[Les15]{Lesnick2015}
Michael Lesnick:
{\it The theory of the interleaving distance on multidimensional persistence modules}, Foundations of Computational Mathematics, 15(3):613--650, 2015.







\end{thebibliography}

\end{document}